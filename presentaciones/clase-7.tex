%\documentclass[]{beamer}
\documentclass[handout]{beamer}
%\documentclass[handout,draft]{beamer}

% Preambulo
% Paquetes de la ams
\usepackage{amsmath,amsthm,amssymb,amsfonts}
% Posibilidad de mover la pagina
\usepackage[a4paper]{geometry}
% Saco la indentacion en todos los parrafos.
%\usepackage{parskip}
% Codificacion UTF-8
\usepackage[utf8]{inputenc}
% Tablas e imagenes en espaniol
\usepackage[spanish,es-tabla]{babel}
% Mejores graficos
\usepackage{graphicx}
% tablas mas lindas
\usepackage{booktabs}
% Posibilidad de tocar los encabezados
\usepackage{fancyhdr}
%\pagestyle{fancy}
% Posibilidad de meter subfiguras
\usepackage[font=footnotesize, labelfont=it]{subcaption}
% Links a urls
\usepackage{url}
% Linkear referencias en pdfs
\usepackage{hyperref}
% Texto mas lindo para los pie de figura
\usepackage[margin=10pt,font=small,labelfont=bf, labelsep=endash]{caption}
% Mejores autores
\usepackage[affil-it]{authblk}
% Compatibilidad con PDF/A
\usepackage{xmpincl}
% Hoja a4 mas ancha
\usepackage{a4wide}
% Citas
\usepackage[backend=biber,style=ieee]{biblatex}
\addbibresource{biblio.bib}
% Cambio and por y
\renewcommand\Authand{y }
\renewcommand\Authands{, y }

% Codigo
\usepackage{listings}

% Coloreo los links
\usepackage[usenames,dvipsnames]{xcolor}
\hypersetup{colorlinks,
     linkcolor={red!50!black},
     citecolor={blue!50!black},
     urlcolor={blue!80!black} }
% Graficos con tikz
\usepackage{tikz}

% Dir tree
\usepackage{dirtree}

% Configuracion de listings para R
\lstset{%
  language=R,                     % the language of the code
  basicstyle=\footnotesize,       % the size of the fonts that are used for the code
  numbers=left,                   % where to put the line-numbers
  numberstyle=\tiny\color{gray},  % the style that is used for the line-numbers
  stepnumber=1,                   % the step between two line-numbers. If it's 1, each line
                                  % will be numbered
  numbersep=5pt,                  % how far the line-numbers are from the code
  backgroundcolor=\color{white},  % choose the background color. You must add \usepackage{color}
  showspaces=false,               % show spaces adding particular underscores
  showstringspaces=false,         % underline spaces within strings
  showtabs=false,                 % show tabs within strings adding particular underscores
  %frame=single,                   % adds a frame around the code
  rulecolor=\color{black},        % if not set, the frame-color may be changed on line-breaks within not-black text (e.g. commens (green here))
  tabsize=2,                      % sets default tabsize to 2 spaces
  captionpos=b,                   % sets the caption-position to bottom
  breaklines=true,                % sets automatic line breaking
  breakatwhitespace=false,        % sets if automatic breaks should only happen at whitespace
  title=\lstname,                 % show the filename of files included with \lstinputlisting;
                                  % also try caption instead of title
  keywordstyle=\color{blue},      % keyword style
  commentstyle=\color{OliveGreen},   % comment style
  stringstyle=\color{Plum}       % string literal style
}

\definecolor{A11}{HTML}{B2DF8A}
\definecolor{A12}{HTML}{33A02C}
\definecolor{A23}{HTML}{FDBF6F}
\definecolor{A24}{HTML}{FF7F00}
\definecolor{B15}{HTML}{FB9A99}
\definecolor{B16}{HTML}{E31A1C}
\definecolor{B27}{HTML}{A6CEE3}
\definecolor{B28}{HTML}{1F78B4}


\title{Herramientas de Teledetección Cuantitativa\\{\small Clase 6}}
\author{Francisco Nemi\~na}
\institute[Inst.]{\includegraphics[height=1cm]{imagenes/logosopi.png}\phantom{pepe} \includegraphics[height=1cm]{imagenes/2mp.png}\phantom{pepe} \includegraphics[height=1cm]{imagenes/conae.png}}
\date{}
%\titlegraphic{
%\includegraphics[height=1cm]{IMAGENES/minplan.png}\phantom{1}
%\includegraphics[height=1cm]{IMAGENES/conae.png}\phantom{1}
%\includegraphics[height=1cm]{IMAGENES/sopi.png}}

\logo{\includegraphics[height=0.7cm]{imagenes/sopi.png}}

\AtBeginSection[]
{
\begin{frame}
\frametitle{Esquema de presentaci\'on}
\tableofcontents[currentsection]
\end{frame}
}


\begin{document}
\begin{frame}
    \maketitle
\end{frame}

\section{Introducci\'on}
\subsection{Nociones b\'asicas}
\begin{frame}{\subsecname}
\begin{block}{Objetivo de la validaci\'on}
  Lo que esperamos es asignarle a nuestro mapa tem\'atico un cierto grado de confianza a partir de datos medidos en el terreno.
\end{block}
\end{frame}

\begin{frame}{\subsecname}
  \begin{figure}
  \includegraphics[width=0.5\textwidth]{imagenes/valid.png}
  \caption{Ejemplo de datos de referencia contra un mapa tem\'atico.\footfullcite{congalton2008assessing}}
  \end{figure}
\end{frame}
%--- Next Frame ---%

\begin{frame}{\subsecname}
  \begin{figure}
  \includegraphics[width=0.4\textwidth]{imagenes/area-v.png}
  \caption{Comparaci\'on de area total.\footfullcite{congalton2008assessing}}
  \end{figure}
\end{frame}
%--- Next Frame ---%

\begin{frame}{\subsecname}
  \begin{figure}
  \includegraphics[width=0.4\textwidth]{imagenes/area-e.png}
  \caption{Comparacion de espacial.\footfullcite{congalton2008assessing}}
  \end{figure}
\end{frame}
%--- Next Frame ---%

\subsection{Matriz de confusi\'on}
\begin{frame}{\subsecname}
\begin{block}{Definici\'on}
  Lo que esperamos es asignarle a nuestro mapa tem\'atico un cierto grado de presici\'on a partir de datos medidos en el terreno.
\end{block}
\end{frame}

\begin{frame}{\subsecname}
\begin{block}{Definici\'on}
\[
\begin{bmatrix}
      & 1               & 2           &  \dots     & k      &  n_{i+}\\
    1 & n_{11}          & n_{12}      & \dots & n_{1k} &  n_{1+}\\
    2 & n_{21}          & n_{22} & \dots & n_{2k} &  n_{2+}\\
    \vdots  & \vdots & \vdots & \ddots      & \vdots         &  \ddots\\
    k & n_{k1} & n_{k2} & \dots       & n_{kk}       &  n_{k+}\\
    n_{+j} & n_{+1} & n_{+2} & \dots & n_{+k} & N
\end{bmatrix} \]
\end{block}
\end{frame}

\begin{frame}{\subsecname}
\begin{block}{Definici\'on}
  Donde
  $$n_{i+} = \sum_j n_{ij}$$
  $$n_{+j} = \sum_i n_{ij}$$
  y donde $n$ es el n\'umero total de muestras.
\end{block}
\end{frame}

\begin{frame}{\subsecname}
  \begin{exampleblock}{Ejemplo}
    Vamos a tomar s\'olo tres coberturas a modo de ejemplo
    \[
    \begin{bmatrix}
          & A   & S    & V  & \\
        A & 50  & 10   & 20 & 80 \\
        S & 5   & 100  & 15 & 120 \\
        V & 10  & 10   & 80 & 100 \\
          & 65  & 120  & 115& 300
    \end{bmatrix} \]
  \end{exampleblock}
\end{frame}

\begin{frame}{\subsecname}
  \begin{block}{Presici\'on total}
    $$\frac{\sum_i n_{ii}}{n}$$
  \end{block}\pause
  \begin{block}{Presici\'on usuario}
    $$\frac{ n_{ii}}{n_{i+}}$$
  \end{block}\pause
  \begin{block}{Presici\'on productor}
    $$\frac{n_{jj}}{n_{+j}}$$
  \end{block}  \pause
\end{frame}

\begin{frame}{\subsecname}
  \begin{block}{Fracci\'on de la muestra}
    $$p_{ij} = \frac{n_{ij}}{n}$$
  \end{block}\pause

  \begin{block}{Probabilidad de j en los datos de campo}
    $$p_{+j} = \sum_i p_{ij}$$
  \end{block}
  \begin{block}{Probabilidad de i en la clasificaci\'on}
    $$p_{i+} = \sum_j p_{ij}$$
  \end{block}
\end{frame}

\begin{frame}{\subsecname}
  \begin{exampleblock}{Ejemplo}
    \[
    \begin{bmatrix}
          & A   & S        & V    & \\
        A & 0.17  & 0.03   & 0.07 & 0.625 \\
        S & 0.02  & 0.33   & 0.05 & 0.833 \\
        V & 0.03  & 0.03   & 0.27 & 0.800 \\
          & 0.769 & 0.833  & 0.696& 0.767
    \end{bmatrix} \]
  \end{exampleblock}
\end{frame}

\begin{frame}{\subsecname}
  \begin{block}{Matriz de confusi\'on}
    Cualquier an\'alisis sobre el error de una clasificaci\'on parte de la matriz de confusi\'on.
  \end{block}
\end{frame}

\subsection{\'Indice kappa}

\begin{frame}{\subsecname}
  \begin{block}{Definici\'on}
    El \'indice kappa nos permite estimar si dos matrices de confusi\'on son distintas una de la otra o no. \pause
    \\
    Mide cuanto se acerca mi clasificaci\'on a una clasificaci\'on al azar.
  \end{block}
\end{frame}

\begin{frame}{\subsecname}
  \begin{block}{Definici\'on}
    $$\hat{K} = \frac{p_0 - p_c}{1-p_c}$$ donde
    $$p_0 = \sum_i p_{ii}$$ y $$p_c = \sum_i p_{i+}p_{+i}$$
  \end{block}
\end{frame}

\begin{frame}{\subsecname}
  \begin{exampleblock}{Ejemplo}
    En este caso, $p_0 = 0.77$ \pause y $p_c = 0.35$ entonces el \'indice kappa nos queda $$\hat{K} = \frac{0.77-0.35}{1-0.35} = 0.64$$\pause Ahora hay que interpretar esto.
  \end{exampleblock}
\end{frame}

\begin{frame}{\subsecname}
  \begin{block}{Interpretaci\'on}
    Interpretaciones del \'indice kappa hay muchas. Lo mas b\'asico es que cuanto m\'as cerca de cero este el valor, m\'as se parece la clasificacion una clasificaci\'on aleatoria.
  \end{block}
  \begin{block}{Cortes}
    \begin{figure}
      \begin{tabular}{c c c }
        0.0 - 0.4 & 0.4 - 0.8 & 0.8 - 1.0 \\
        Pobre     & Medio     & Bueno
      \end{tabular}
      \caption{Rangos de acuerdo del \'indice kappa \footfullcite{landis1977measurement}}
    \end{figure}
  \end{block}
\end{frame}

\subsection{Muestreo}

\begin{frame}{\subsecname}
  \begin{block}{4 preguntas}
  \begin{enumerate}
    \item ¿Qu\'e categor\'ias tengo?
    \item ¿Qu\'e unidad de muestreo usar?
    \item ¿Cu\'antas muestras tomar?
    \item ¿C\'omo elegir las muestras?
  \end{enumerate}
  \end{block}
\end{frame}

\begin{frame}{\subsecname}
  \begin{block}{¿Que categor\'ias tengo?}
    Las clases tienen que ser \pause
    \begin{itemize}[<+>]
      \item Mutualmente exclusivas
      \item Totalmente exaustivas
    \end{itemize}
    Adem\'as de tener un tamaño m\'inimo para ser considerado de esa clase.
  \end{block}
\end{frame}

\begin{frame}{\subsecname}
  \begin{figure}
  \includegraphics[width=0.8\textwidth]{imagenes/unidad_mapa.png}
  \caption{Clases de muestreo definidas en el terreno.\footfullcite{congalton2008assessing}}
  \end{figure}
\end{frame}
%--- Next Frame ---%

\begin{frame}{\subsecname}
  \begin{block}{¿Qu\'e unidad de muestreo usar?}
    \begin{itemize}[<+>]
      \item Un solo p\'ixel.
      \item Un cl\'uster de p\'ixeles
      \item Un pol\'igono
      \item Un cl\'uster de pol\'igonos
    \end{itemize}
  \end{block}
\end{frame}

\begin{frame}{Muestreo}
  \begin{block}{¿Cu\'antas muestras tomar?}
    $$N = \frac{B}{4b^2}$$ donde $B$ se obtiene a partir de la distribuci\'on $\chi^2$ con un grado de libertad y $b$ es la presici\'on que uno acepta.
  \end{block}
\end{frame}

\begin{frame}{\subsecname}
  \begin{block}{¿Como elegir las muestras?}
    \begin{itemize}[<+>]
      \item Al azar.
      \item Estratificado al azar.
      \item Sistem\'atico.
      \item Clusters
    \end{itemize}
  \end{block}
\end{frame}

\begin{frame}{Muestreo}
  \begin{alertblock}{Logistica}
    Todo lo que vimos va a estar supeditado a mi capacidad de realizar el muestreo.
  \end{alertblock}
\end{frame}

\section{Pr\'actica}

\begin{frame}{Pr\'actica}
  \begin{exampleblock}{Actividades pr\'acticas de la sexta clase}
    \begin{enumerate}
      \item Abrir las im\'agenes clasificadas y fusionadas por el m\'etodo de clasificaci\'on supervisada y no supervisada.
      \item Cargar los pol\'igonos de validaci\'on correspondientes a cada clase.
      \item Calcular al matriz de confusi\'on correspondiente a cada clasificaci\'on.
      \item Obtener la presici\'on global, del usuario, productor y el \'indice kappa.
    \end{enumerate}
  \end{exampleblock}
\end{frame}
%--- Next Frame ---%

\section{Pr\'actica}

\begin{frame}{Pr\'actica}
  \begin{exampleblock}{Actividades pr\'acticas de la sexta clase}
    \begin{enumerate}
      \item Abrir las im\'agenes clasificadas y fusionadas por el m\'etodo de clasificaci\'on supervisada y no supervisada.
      \item Cargar los pol\'igonos de validaci\'on correspondientes a cada clase.
      \item Calcular al matriz de confusi\'on correspondiente a cada clasificaci\'on.
      \item Obtener la presici\'on global, del usuario, productor y el \'indice kappa.
    \end{enumerate}
  \end{exampleblock}
\end{frame}
%--- Next Frame ---%

\end{document}
