%\documentclass[]{beamer}
\documentclass[handout]{beamer}
%\documentclass[handout,draft]{beamer}

% Preambulo
% Paquetes de la ams
\usepackage{amsmath,amsthm,amssymb,amsfonts}
% Posibilidad de mover la pagina
\usepackage[a4paper]{geometry}
% Saco la indentacion en todos los parrafos.
%\usepackage{parskip}
% Codificacion UTF-8
\usepackage[utf8]{inputenc}
% Tablas e imagenes en espaniol
\usepackage[spanish,es-tabla]{babel}
% Mejores graficos
\usepackage{graphicx}
% tablas mas lindas
\usepackage{booktabs}
% Posibilidad de tocar los encabezados
\usepackage{fancyhdr}
%\pagestyle{fancy}
% Posibilidad de meter subfiguras
\usepackage[font=footnotesize, labelfont=it]{subcaption}
% Links a urls
\usepackage{url}
% Linkear referencias en pdfs
\usepackage{hyperref}
% Texto mas lindo para los pie de figura
\usepackage[margin=10pt,font=small,labelfont=bf, labelsep=endash]{caption}
% Mejores autores
\usepackage[affil-it]{authblk}
% Compatibilidad con PDF/A
\usepackage{xmpincl}
% Hoja a4 mas ancha
\usepackage{a4wide}
% Citas
\usepackage[backend=biber,style=ieee]{biblatex}
\addbibresource{biblio.bib}
% Cambio and por y
\renewcommand\Authand{y }
\renewcommand\Authands{, y }

% Codigo
\usepackage{listings}

% Coloreo los links
\usepackage[usenames,dvipsnames]{xcolor}
\hypersetup{colorlinks,
     linkcolor={red!50!black},
     citecolor={blue!50!black},
     urlcolor={blue!80!black} }
% Graficos con tikz
\usepackage{tikz}

% Dir tree
\usepackage{dirtree}

% Configuracion de listings para R
\lstset{%
  language=R,                     % the language of the code
  basicstyle=\footnotesize,       % the size of the fonts that are used for the code
  numbers=left,                   % where to put the line-numbers
  numberstyle=\tiny\color{gray},  % the style that is used for the line-numbers
  stepnumber=1,                   % the step between two line-numbers. If it's 1, each line
                                  % will be numbered
  numbersep=5pt,                  % how far the line-numbers are from the code
  backgroundcolor=\color{white},  % choose the background color. You must add \usepackage{color}
  showspaces=false,               % show spaces adding particular underscores
  showstringspaces=false,         % underline spaces within strings
  showtabs=false,                 % show tabs within strings adding particular underscores
  %frame=single,                   % adds a frame around the code
  rulecolor=\color{black},        % if not set, the frame-color may be changed on line-breaks within not-black text (e.g. commens (green here))
  tabsize=2,                      % sets default tabsize to 2 spaces
  captionpos=b,                   % sets the caption-position to bottom
  breaklines=true,                % sets automatic line breaking
  breakatwhitespace=false,        % sets if automatic breaks should only happen at whitespace
  title=\lstname,                 % show the filename of files included with \lstinputlisting;
                                  % also try caption instead of title
  keywordstyle=\color{blue},      % keyword style
  commentstyle=\color{OliveGreen},   % comment style
  stringstyle=\color{Plum}       % string literal style
}

\definecolor{A11}{HTML}{B2DF8A}
\definecolor{A12}{HTML}{33A02C}
\definecolor{A23}{HTML}{FDBF6F}
\definecolor{A24}{HTML}{FF7F00}
\definecolor{B15}{HTML}{FB9A99}
\definecolor{B16}{HTML}{E31A1C}
\definecolor{B27}{HTML}{A6CEE3}
\definecolor{B28}{HTML}{1F78B4}


\title{Herramientas de Teledetección Cuantitativa\\{\small Clase 4}}
\author{Francisco Nemi\~na}
\institute{Unidad de Educación y Formación Masiva \\ Comisión Nacional de
Actividades Espaciales}
%\institute[Inst.]{\includegraphics[height=1cm]{Figures/logosopi.png}\phantom{pepe} \includegraphics[height=1cm]{Figures/2mp.png}\phantom{pepe} \includegraphics[height=1cm]{Figures/conae.png}}
\date{}
%\titlegraphic{
%\includegraphics[height=1cm]{IMAGENES/minplan.png}\phantom{1}
%\includegraphics[height=1cm]{IMAGENES/conae.png}\phantom{1}
%\includegraphics[height=1cm]{IMAGENES/sopi.png}}

\logo{\includegraphics[height=0.7cm]{imagenes/sopi.png}}

\AtBeginSection[]
{\begin{frame}
\frametitle{Esquema de presentación}
\tableofcontents[currentsection]
\end{frame}
}


\begin{document}
\begin{frame}
    \maketitle
\end{frame}

\section{Transformaciones}
\subsection{Motivación}

\begin{frame}{Motivación}
  \begin{center}
      \resizebox{0.4 \linewidth}{!}{%
        \begin{tikzpicture}[node distance = 2cm, auto]
          \node[block]                                (init) {Firma Espectral};\pause
          \node[block, below= of init]                (resp) {Reflectancia Espectral Efectiva};
          \path[line] (init) --          (resp);
          \pause
          \node[decision, below= of resp]             (ques) {????};
          \path[line] (resp) --          (ques);\pause
        \end{tikzpicture}%
      }%
    \end{center}
\end{frame}
%--- Next Frame ---%

\begin{frame}{Motivación}
  \begin{block}{Técnicas de reducción de la dimensionalidad}
    \begin{itemize}[<+->]
      \item Índices
      \item Rotaciones
      \item Clasificaciones
    \end{itemize}
    Sigamos con la segunda.
  \end{block}
\end{frame}
%--- Next Frame ---%

\subsection{Matemática}


\begin{frame}{Matemática}
  \begin{block}{Definición:}
    Un vector es un objeto de la forma $$\left(
    \begin{array}{c}
          v_1 \\
          \vdots \\
          v_n
        \end{array}
        \right) $$
  \end{block}\pause
  \begin{block}{Propiedades}
    Con dos operaciones
    $$ \begin{array}{c} v+w \\ \alpha v \end{array}  $$
    \pause
    y viven en un lugar que se llama espacio vectorial
  \end{block}
\end{frame}
%--- Next Frame ---%

\begin{frame}{Matemática}
  \begin{block}{Definición:}
    Las matrices se pueden pensar como transformaciones que convierten a un vector en otro. \pause
    $$ Av = w$$
  \end{block}\pause
  \begin{alertblock}{Propiedad}
    Como las transformaciones que utilizaremos son lineales, con sólo definirlas en unos pocos valores alcanza. \pause Elegir bien los vectores para definir la transformación es útil.
  \end{alertblock}
\end{frame}
%--- Next Frame ---%

\begin{frame}{Matemática}
  \begin{figure}
    \includegraphics[width=0.6\textwidth]{imagenes/elandsat.png}
    \caption{Comparación entre firma espectral y valores medidos para un píxel\footfullcite{richards2013remote}}
  \end{figure}
\end{frame}
%--- Next Frame ---%

\begin{frame}{Matemática}
  \begin{figure}
    \includegraphics[width=0.6\textwidth]{imagenes/vector.png}
    \caption{Píxeles en $R^2$.\footfullcite{richards2013remote}}
  \end{figure}
\end{frame}
%--- Next Frame ---%

\begin{frame}{Matemática}
  \begin{alertblock}{Respuesta efectiva como vector}
    A la respuesta espectral efectiva la puedo pensar como un vector de reflectancias
      $$\left(\begin{array}{c}
          \rho_1 \\
          \vdots \\
          \rho_n
        \end{array}
    \right) $$
  \end{alertblock}
  \pause
\end{frame}
\begin{frame}{Matemática}
  \begin{exampleblock}{Ejemplo:}
    $$v_{(x,y)}=\left(\begin{array}{c}
      0.03 \\
      0.08 \\
      0.04 \\
      0.40 \\
      0.20 \\
      0.15
    \end{array}
    \right),
    a_{(x,y)}=\left(\begin{array}{c}
      0.05 \\
      0.03 \\
      0.01 \\
      0.01 \\
      0.00 \\
      0.00
    \end{array}
    \right),
    s_{(x,y)}=\left(\begin{array}{c}
      0.08 \\
      0.10 \\
      0.15 \\
      0.20 \\
      0.25 \\
      0.30
    \end{array}
    \right) $$
  \end{exampleblock}
\end{frame}
%--- Next Frame ---%

\begin{frame}{Matemática}
  \begin{alertblock}{Motivación}
    Podemos pensar a una imagen como vectores en un espacio vectorial. El número de bandas es la dimensión de ese espacio.
  \end{alertblock}
\end{frame}
%--- Next Frame ---%

\section{Rotaciones}
\subsection{Idea}
\begin{frame}{Idea}
  Empecemos con un ejemplo para una imagen de dos bandas
  \begin{figure}
  \centering
  \includegraphics[width=0.6\textwidth]{imagenes/nir-red.png}
  \caption{Imagen de dos bandas.}
  \end{figure}
\end{frame}
%--- Next Frame ---%

\begin{frame}{Idea}
  \begin{figure}
  \centering
  \includegraphics[width=0.8\textwidth]{imagenes/nir-red2.png}
  \caption{Imagen de dos bandas en el espacio vectorial.}
  \end{figure}
\end{frame}
%--- Next Frame ---%

\begin{frame}{Idea}
  \begin{alertblock}{Transformación}
    Una combinación obvia es $$ \rho_d = 0.5\rho_n-0.5\rho_r$$
    y
    $$ \rho_s = 0.5\rho_n+0.5\rho_r $$
  \end{alertblock}
\end{frame}
%--- Next Frame ---%


\begin{frame}{Idea}
  \begin{alertblock}{Importante}
    No siempre más bandas significa mas información.
  \end{alertblock}
\end{frame}
%--- Next Frame ---%

\subsection{Componentes principales}

\begin{frame}{Componentes principales}
  \begin{block}{Idea}
    Queremos ver si un set bandas está correlacionadas o no.
  \end{block}
\end{frame}
%--- Next Frame ---%

\begin{frame}{Componentes principales}
  \begin{figure}
  \centering
  \includegraphics[width=0.8\textwidth]{imagenes/corr.png}
  \caption{Datos correlacionados y no correlacionados\footfullcite{richards2013remote}}
  \end{figure}
\end{frame}
%--- Next Frame ---%

\begin{frame}{Componentes principales}
  \begin{block}{Matriz de correlación}
    Tiene en sus componentes las funciones de correlación entre cada banda\pause
    \[
    A = \begin{bmatrix}
        corr_{11}       & corr_{12} & corr_{13} & \cdots & corr_{1n} \\
        corr_{21}       & corr_{22} & corr_{23} & \cdots & corr_{2n} \\
        \vdots          & \vdots    & \vdots    & \ddots & \vdots \\
        corr_{n1}       & corr_{d2} & corr_{n3} & \cdots & corr_{nn}
    \end{bmatrix} \]
  \end{block}
\end{frame}
%--- Next Frame ---%

\begin{frame}{Componentes principales}
  \begin{block}{Observaciones}
      Queremos que la correlación cruzada entre bandas sea cero.
      \pause\@ Matemáticamente lo pedimos como
        $$Av=\lambda v$$
      Y nos quedamos como vectores útiles a los que cumplan esto.
  \end{block}
\end{frame}
%--- Next Frame ---%

\begin{frame}{Componentes principales}
  \begin{block}{Matriz de correlación}
    La forma de la matriz va a depender de las combinaciones lineal que haga
      entre los vectores \pause\@
    \[\begin{bmatrix}
        \lambda{1}       & 0 & 0 & \cdots & 0 \\
        0       & \lambda_{2} & 0 & \cdots & 0 \\
        \vdots & \vdots & \vdots & \ddots & \vdots \\
        0       & 0 & 0 & \cdots & \lambda_{n}
    \end{bmatrix} \]
    \pause\@
    donde son los autovectores $$\lambda_{1}  > \lambda_{2} > \cdots > \lambda_{n}$$
  \end{block}
\end{frame}
%--- Next Frame ---%

\begin{frame}{Componentes principales}
  \begin{block}{Observaciones}
    \begin{itemize}[<+>]
      \item $\frac{\lambda_i}{\sum_i \lambda_i}$ me habla de cuanto me explica ese vector sobre la variabilidad de la imagen
      \item $(v_1 , \dots , v_n)$ el autovector que me representa la combinación de bandas de un autovalor dado.
      \item Estas combinación lineal de bandas tienen la información más relevante.
    \end{itemize}
  \end{block}
\end{frame}
%--- Next Frame ---%

\begin{frame}{Componentes principales}
  \begin{exampleblock}{Ejemplo}
    Volviendo al ejemplo de antes
    \[
    \begin{bmatrix}
        1       & 0.329127 \\
        0.329127 & 1
    \end{bmatrix} \]
  \end{exampleblock}
\end{frame}
%--- Next Frame ---%

\begin{frame}{Componentes principales}
  \begin{exampleblock}{Ejemplo}
    Al diagonalizar me queda
    \[
    \begin{bmatrix}
        1.343685       & 0 \\
        0       & 0.656315
    \end{bmatrix} \]
    con autovectores $$0.707107\, \rho_n-0.707107 \, \rho_r$$  y $$0.707107 \,
      \rho_n+0.707107\, \rho_r$$ \pause\@
    Acá el primer vector explica el el 67\% de la variabilidad de la imagen y el segundo del 33\%.
  \end{exampleblock}
\end{frame}
%--- Next Frame ---%

\begin{frame}{Componentes principales}
  \begin{figure}
  \centering
  \includegraphics[width=0.6\textwidth]{imagenes/pca1.png}
  \caption{Ejemplo con las bandas nir-rojo en la imagen.}
  \end{figure}
\end{frame}
%--- Next Frame ---%

\begin{frame}{Componentes principales}
  \begin{figure}
  \centering
  \includegraphics[width=0.8\textwidth]{imagenes/pca2.png}
  \caption{Ejemplo con las bandas nir-rojo en el espacio vectorial.}
  \end{figure}
\end{frame}
%--- Next Frame ---%

\subsection{Transformada tasseled-cap}

\begin{frame}{Transformada tasseled-cap}
  \begin{block}{Utilidad}
    La utilidad de esto no suele ser con dos bandas, si no con muchas más.
  \end{block}\pause
  \begin{block}{Problema}
    Acá es mas fácil darse cuenta que brinda mas información, el tema es interpretar esa información.
  \end{block}\pause
  \begin{block}{Idea}
    Encontrar alguna transformación que me permita descartar bandas pero que tengan relación con distintos comportamientos biofísicos.
  \end{block}
\end{frame}
%--- Next Frame ---%

\begin{frame}{Transformada tasseled-cap}
  \begin{figure}
  \centering
  \includegraphics[width=0.8\textwidth]{imagenes/tc.png}
  \caption{Movimiento asociado al comportamiento fenológico de un píxel de vegetación en el espacio vectorial.\footfullcite{richards2013remote}}
  \end{figure}
\end{frame}
%--- Next Frame ---%


\begin{frame}{Transformada tasseled-cap}
    \begin{figure}
      \begin{tabular}{l c c c c c c }
        Combinación  & Azul & Verde & Rojo & nir & swir 1 & swir 2\\
        Brillo &  0.30  & 0.27  & 0.47  & 0.55  & 0.50  & 0.18\\
        Verdor & -0.29  &-0.24  &-0.54  & 0.72 & 0.07  &-0.16\\
        Humedad&  0.15  & 0.19  & 0.32  & 0.34  &-0.71  &-0.45
      \end{tabular}
      \caption{Transformada tasseled-cap para landsat 8 \footfullcite{baig2014derivation}}
    \end{figure}
\end{frame}
%--- Next Frame ---%

\begin{frame}{Transformada tasseled-cap}
  \begin{block}{Idea}
    Todo esto logra hacer que el número de bandas que utilizo sea menor que el
      número de bandas inicial
  \end{block}
\end{frame}
%--- Next Frame ---%
\section{Práctica}

\begin{frame}{Práctica}
  \begin{exampleblock}{Actividades prácticas de la tercer clase}
    \begin{enumerate}
        \item Realice la transformada por componentes principales sobre la
            imagen de NDVI de modis.
        \item Realice la transformada por componentes principales sobre la 
            serie de imágenes Landsat 8.
        \item Realice la transformada tasseled-cap sobre la imagen Landsat 8.
        \item Realice la transformada por componentes principales sobre la
            imagen Landsat 8.
    \end{enumerate}
  \end{exampleblock}
\end{frame}
%--- Next Frame ---%

\end{document}
