\documentclass[]{beamer}
%\documentclass[handout]{beamer}
%\documentclass[handout,draft]{beamer}

% Preambulo
% Paquetes de la ams
\usepackage{amsmath,amsthm,amssymb,amsfonts}
% Posibilidad de mover la pagina
\usepackage[a4paper]{geometry}
% Saco la indentacion en todos los parrafos.
%\usepackage{parskip}
% Codificacion UTF-8
\usepackage[utf8]{inputenc}
% Tablas e imagenes en espaniol
\usepackage[spanish,es-tabla]{babel}
% Mejores graficos
\usepackage{graphicx}
% tablas mas lindas
\usepackage{booktabs}
% Posibilidad de tocar los encabezados
\usepackage{fancyhdr}
%\pagestyle{fancy}
% Posibilidad de meter subfiguras
\usepackage[font=footnotesize, labelfont=it]{subcaption}
% Links a urls
\usepackage{url}
% Linkear referencias en pdfs
\usepackage{hyperref}
% Texto mas lindo para los pie de figura
\usepackage[margin=10pt,font=small,labelfont=bf, labelsep=endash]{caption}
% Mejores autores
\usepackage[affil-it]{authblk}
% Compatibilidad con PDF/A
\usepackage{xmpincl}
% Hoja a4 mas ancha
\usepackage{a4wide}
% Citas
\usepackage[backend=biber,style=ieee]{biblatex}
\addbibresource{biblio.bib}
% Cambio and por y
\renewcommand\Authand{y }
\renewcommand\Authands{, y }

% Codigo
\usepackage{listings}

% Coloreo los links
\usepackage[usenames,dvipsnames]{xcolor}
\hypersetup{colorlinks,
     linkcolor={red!50!black},
     citecolor={blue!50!black},
     urlcolor={blue!80!black} }
% Graficos con tikz
\usepackage{tikz}

% Dir tree
\usepackage{dirtree}

% Configuracion de listings para R
\lstset{%
  language=R,                     % the language of the code
  basicstyle=\footnotesize,       % the size of the fonts that are used for the code
  numbers=left,                   % where to put the line-numbers
  numberstyle=\tiny\color{gray},  % the style that is used for the line-numbers
  stepnumber=1,                   % the step between two line-numbers. If it's 1, each line
                                  % will be numbered
  numbersep=5pt,                  % how far the line-numbers are from the code
  backgroundcolor=\color{white},  % choose the background color. You must add \usepackage{color}
  showspaces=false,               % show spaces adding particular underscores
  showstringspaces=false,         % underline spaces within strings
  showtabs=false,                 % show tabs within strings adding particular underscores
  %frame=single,                   % adds a frame around the code
  rulecolor=\color{black},        % if not set, the frame-color may be changed on line-breaks within not-black text (e.g. commens (green here))
  tabsize=2,                      % sets default tabsize to 2 spaces
  captionpos=b,                   % sets the caption-position to bottom
  breaklines=true,                % sets automatic line breaking
  breakatwhitespace=false,        % sets if automatic breaks should only happen at whitespace
  title=\lstname,                 % show the filename of files included with \lstinputlisting;
                                  % also try caption instead of title
  keywordstyle=\color{blue},      % keyword style
  commentstyle=\color{OliveGreen},   % comment style
  stringstyle=\color{Plum}       % string literal style
}

\definecolor{A11}{HTML}{B2DF8A}
\definecolor{A12}{HTML}{33A02C}
\definecolor{A23}{HTML}{FDBF6F}
\definecolor{A24}{HTML}{FF7F00}
\definecolor{B15}{HTML}{FB9A99}
\definecolor{B16}{HTML}{E31A1C}
\definecolor{B27}{HTML}{A6CEE3}
\definecolor{B28}{HTML}{1F78B4}


\title{Herramientas de Teledetección Cuantitativa\\{\small Clase 1}}
\author{Francisco Nemiña \and Diego Schell \and Laura Rouco}
\institute{Unidad de Educación y Formaci\'on Masiva \\ Comisi\'on Nacional de
Actividades Espaciales}
%\institute[Inst.]{\includegraphics[height=1cm]{Figures/logosopi.png}\phantom{pepe} \includegraphics[height=1cm]{Figures/2mp.png}\phantom{pepe} \includegraphics[height=1cm]{Figures/conae.png}}
\date{}
%\titlegraphic{
%\includegraphics[height=1cm]{imagenes/minplan.png}\phantom{1}
%\includegraphics[height=1cm]{imagenes/conae.png}\phantom{1}
%\includegraphics[height=1cm]{imagenes/sopi.png}}

\logo{\includegraphics[height=0.7cm]{imagenes/sopi.png}}

\AtBeginSection[]
{\begin{frame}
\frametitle{Esquema de presentación}
\tableofcontents[currentsection]
\end{frame}
}

\begin{document}
\begin{frame}
    \maketitle
\end{frame}

\section{Introducción}
\subsection{Organización del curso}
\begin{frame}
  \frametitle{Objetivos del curso}
  \begin{itemize}
    \item<1-> Poder analizar en detalle una {\only<|2->{\color{red}}firma espectral.}
    \item<2> Familiarizarse con el concepto de reflectancia bidireccional.
    \item<3> Conocer las distintas fuentes de distorsión radiométrica.
    \item<4> Comprender el concepto de dimensionalidad y como reducir la misma.
    \item<5> Poder realizar clasificaciones supervisadas y no supervisadas comprendiendo los fundamentos matemáticos detrás de las mismas.
    \item<6> Poder realizar validaciones de clasificaciones.
    \item<7> Realizar estudios de series temporales.
  \end{itemize}
\end{frame}

\begin{frame}[fragile]{Organización del curso}
  \begin{block}{Plataforma de Educación a Distancia}
    \verb+https://sopi.conae.gov.ar/aulavirtual+\\
    \verb+Contraseña: benteveo2016+
  \end{block}\pause
  \begin{block}{Aprobación}
    \begin{enumerate}
        \item \emph{75\% de asistencia.}
        \item sumar 60 puntos entre
            \begin{itemize}
                \item 7 cuestionarios teórico-prácticos sobre las clases
                \item 1 trabajo final integrador
            \end{itemize}
    \end{enumerate}\pause
  \end{block}
\end{frame}
%--- Next Frame ---%

\begin{frame}[fragile]{Organización del curso}
  \begin{block}{Cronograma}
    \begin{itemize}[<+>]
      \item 8/4 Conceptos básicos y firmas espectrales.
      \item 15/4 Correcciones radiométricas.
      \item 22/4 Dimensionalidad 
      \item 29/4 \'Indices.
      \item 6/5 Clasificaciones no supervisadas.
      \item 13/5 Clasificaciones supervisadas.
      \item 20/5 Validación de datos satelitales.
      \item 27/5 Clase de consulta
      \item 3/6  Clase de consulta
      \item 10/5 Entrega del trabajo final.
    \end{itemize}
  \end{block}
\end{frame}

\begin{frame}{Métodos cuantitativos}
  \begin{block}{Definición:}
    Hablamos de \emph{métodos cuantitativos en teledetección \'optica} cuando queremos cuantificar los datos disponibles en una imagen para poder extraer informaci\'on de las mismas utilizando las longitudes de onda de $0.4\mu m$ a $14 \mu m$.
  \end{block}
\end{frame}
%--- Next Frame ---%

\begin{frame}{Métodos cuantitativos}
  \begin{enumerate}
    \item<1-3> Tipos de modelos
    \begin{enumerate}
      \item<2-3> estad\'isticos
      \item<3> biofísicos
    \end{enumerate}
    \item<4-6> Tipos de variables
    \begin{enumerate}
      \item<5-6> continuas
      \item<6> categóricas
    \end{enumerate}
  \end{enumerate}
\end{frame}
%--- Next Frame ---%

\section{Conceptos básicos}
\subsection{Radiancia}
\begin{frame}{Radiancia}
  \begin{block}{Definición:}
    $$dE = L_{\lambda}(\theta,\phi) \cos(\theta) d\Omega dA dt d\lambda$$
    Potencia radiante por unidades de área y \'angulo sólido.
  \end{block}\pause
  \begin{alertblock}{Importante:}
    \begin{itemize}[<+>]
      \item $[L_{\lambda}] = \frac{W}{m^2 sr nm}$
      \item Es una de las dos magnitudes más relevantes.
    \end{itemize}
  \end{alertblock}
\end{frame}
%--- Next Frame ---%

\begin{frame}{Radiancia}
  \begin{figure}
  \centering
  \includegraphics[width=0.8\textwidth]{imagenes/solid_angle.png}
  \caption{\'Angulo sólido $\Omega$ y los ángulos asociados $\theta$ y $\phi$.\footfullcite{liang2005quantitative}}
  \end{figure}
\end{frame}
%--- Next Frame ---%

\begin{frame}{Radiancia}
  \begin{block}{Definición}
    Definimos la irradiancia como
    $$E=\int L(\theta,\phi) \cos(\theta) d\Omega$$\pause
    para el caso de que la luz se emita sólo en uno de los hemisferios
    $$E=\int_0^{2\pi}\int_0^{\pi/2} L(\theta,\phi) \cos(\theta) \sin(\theta) d\theta d\phi$$
  \end{block}
\end{frame}
%--- Next Frame ---%

\begin{frame}{Radiancia}
  \begin{figure}
  \centering
  \includegraphics[width=0.8\textwidth]{imagenes/espectrum.png}
  \caption{Espectro electromagnético.\footfullcite{espectrum}}
  \end{figure}
\end{frame}
%--- Next Frame ---%

\begin{frame}{Radiancia}
  \begin{figure}
  \centering
  \includegraphics[width=0.8\textwidth]{imagenes/blackpercent.png}
  \caption{Irradiancia medida sobre la superficie terrestre.\footfullcite{blackpercent}}
  \end{figure}
\end{frame}
%--- Next Frame ---%

\begin{frame}{Radiancia}
  \begin{block}{Curva de irradiancia}
    Cálculo de la irradiancia de un cuerpo negro
    $$ E(\lambda,T) = \frac{2hc^2}{\lambda^5}\frac{1}{e^{\frac{hc}{\lambda k_B T}}-1}$$
  \end{block}
\end{frame}
%--- Next Frame ---%

\begin{frame}{Radiancia}
  \begin{figure}
  \centering
  \includegraphics[width=0.8\textwidth]{imagenes/blackbody.png}
  \caption{Curva de irradiancia para un cuerpo negro.\footfullcite{blackbody}}
  \end{figure}
\end{frame}
%--- Next Frame ---%

\begin{frame}{Radiancia}
  \begin{block}{Cálculo de la irradiancia solar}
    Cálculo de la irradiancia solar
    $$S_0 = \int_0^\infty E_0(\lambda) d\lambda$$
    su valor aproximado es
    $$ S_0 = 1369 W/m^2$$
    \pause es la cantidad de luz que llega del sol.
  \end{block}
\end{frame}
%--- Next Frame ---%

\begin{frame}{Valores tipos de L}
  \begin{exampleblock}{Valores tipos de E para Landsat}
    En $[L_{\lambda}] = \frac{W}{m^2 \mu m}$
    \begin{figure}
      \begin{tabular}{l c c c}
          Banda & ETM+  & TM    &  OLI \\
          1     & 1970  & 1954  & 1925 \\
          2     & 1843  & 1826  & 1826 \\
          3     & 1555  & 1558  & 1574 \\
          4     & 1047  & 1047  & 955  \\
          5     & 227.1 & 217.2 & 242 \\
          7     & 80.53 & 80.29 & 82.5\\
      \end{tabular}
    \end{figure}
  \end{exampleblock}
\end{frame}
%--- Next Frame ---%

\subsection{Reflectancia}

\begin{frame}{Reflectancia}
  \begin{figure}
  \centering
  \includegraphics[width=0.8\textwidth]{imagenes/brdf.png}
  \caption{Irradiancia incidente y reflejada por una cobertura.\footfullcite{brdf}}
  \end{figure}
\end{frame}
%--- Next Frame ---%

\begin{frame}{Reflectancia}
  \begin{block}{Definición:}
    Definimos la BRDF (espectral bidirectional reflectance distribution function) como:
    $$ f(\theta_i, \phi_i, \theta_r, \phi_r) = \frac{dL(\theta_i, \phi_i, \theta_r, \phi_r)}{dE(\theta_i, \phi_i)}$$
  \end{block}
  \pause
  \begin{block}{Definición:}
    Defininimos la reflectancia direccional como:
    $$ R(\theta_i, \phi_i, \theta_r, \phi_r) = \frac{\pi L(\theta_i, \phi_i, \theta_r, \phi_r)}{\cos(\theta_i) E_0} = \pi f(\theta_i, \phi_i, \theta_r, \phi_r)$$
  \end{block}
\end{frame}
%--- Next Frame ---%

\begin{frame}{Reflectancia}
  \begin{figure}
  \centering
  \includegraphics[width=0.8\textwidth]{imagenes/difusa.png}
  \caption{Distintos casos de reflectancia direccional.\footfullcite{jensen2007remote}}
  \end{figure}
\end{frame}
%--- Next Frame ---%

\begin{frame}{Reflectancia}
  \begin{block}{Aproximación lambertiana}
    Hablamos de la aproximación lambertiana cuando la reflectancia no depende del ángulo reflejado
    $$\rho = \frac{\pi L}{\mu_i E_0}$$
    donde tomamos $\mu=cos(\theta)$
  \end{block}
\end{frame}
%--- Next Frame ---%

\section{Firma espectral}
\subsection{Medición}
\begin{frame}{Medición}
  \begin{block}{Definición:}
    La distribución de la reflectancia es funci\'on de la longitud de onda nos habla de la caracter\'isticas intr\'insecas de la cobertura. Es su firma espectral $\rho_\lambda$.
  \end{block}
\end{frame}
%--- Next Frame ---%

\begin{frame}{Medición}
  \begin{block}{Respuesta espectral}
    Podemos pensar a la respuesta de un sensor como una integral
      $$\rho_{j} =\frac{\int s_j(\lambda) \rho d\lambda}{\int s_j(\lambda) d\lambda}$$
    donde si pensamos a la respuesta como una distribución podemos definir $\lambda_c$ y $\Delta \lambda$ el centro de la adquisici\'on y ancho de banda efectivo.
  \end{block}\pause
  \begin{alertblock}{Importante}
    Desde el punto de vista espectral, las resoluciones espectral y radiométrica, nos permiten distinguir distintas cosas de la firma espectral.
  \end{alertblock}
\end{frame}
%--- Next Frame ---%

\begin{frame}{Medición}
  \begin{figure}\centering
    \includegraphics[width=0.4\textwidth]{imagenes/ebaja.png}\phantom{F}
    \includegraphics[width=0.4\textwidth]{imagenes/ealta.png}
    \caption{Espectral separa.}
  \end{figure}
\end{frame}
%--- Next Frame ---%

\begin{frame}{Medición}
  \begin{figure}\centering
    \includegraphics[width=0.4\textwidth]{imagenes/rbaja.png}\phantom{F}
    \includegraphics[width=0.4\textwidth]{imagenes/ralta.png}
    \caption{Resolución radiométrica.}
  \end{figure}
\end{frame}
%--- Next Frame ---%

\begin{frame}{Medición}
  \begin{figure}
  \centering
  \includegraphics[width=0.8\textwidth]{imagenes/respacial.png}
  \caption{Respuesta espacial de un sensor en ambas direcciones.\footfullcite{liang2005quantitative}}
  \end{figure}
\end{frame}
%--- Next Frame ---%

\begin{frame}{Medición}
  \begin{block}{Respuesta espacial}
    \begin{itemize}
      \item La resolución espacial sale de esta funci\'on.
      \item Es importante por que nos permite comprender la formación de un p\'ixel.
    \end{itemize}
  \end{block}
  \begin{block}{Formación de un p\'ixel}
    El valor de reflectancia para un p\'ixel vale
    $$\rho_{pix} = \sum_i w_i \rho_i$$
    donde $w_i$ corresponde a la distinta cobertura de cada p\'ixel.
  \end{block}
\end{frame}

\subsection{Modelado}

\begin{frame}{Modelado}
  \begin{figure}
  \centering
  \includegraphics[width=0.8\textwidth]{imagenes/puras.png}
  \caption{Firmas espectrales de vegetación y suelo desnudo.\footfullcite{clark2007usgs}}
  \end{figure}
\end{frame}
%--- Next Frame ---%

\begin{frame}{Modelado}
  \begin{figure}
  \centering
  \includegraphics[width=0.8\textwidth]{imagenes/mezcla.png}
  \caption{Mezcla de firmas espectrales para un gradiente de coberturas.\footfullcite{clark2007usgs}}
  \end{figure}
\end{frame}
%--- Next Frame ---%

\begin{frame}{Modelado}
  La vegetación tiene 3 zonas del espectro principales que modelar
  \begin{itemize}
    \item<1> Visible
    \item<2> Infrarrojo cercano
    \item<3> Infrarrojo de onda media
  \end{itemize}
\end{frame}
%--- Next Frame ---%

\begin{frame}{Modelado}
    \begin{figure}
    \centering
    \includegraphics[width=0.8\textwidth]{imagenes/clorovar.png}
    \caption{Variaciones de la firma espectral con el contenido de clorofila.\footfullcite{liang2005quantitative}}
    \end{figure}
\end{frame}
%--- Next Frame ---%

\begin{frame}{Modelado}
    \begin{figure}
    \centering
    \includegraphics[width=0.8\textwidth]{imagenes/vwvar.png}
    \caption{Variaciones de la firma espectral con el contenido de agua.\footfullcite{liang2005quantitative}}
    \end{figure}
\end{frame}
%--- Next Frame ---%

\begin{frame}{Modelado}
    \begin{figure}
    \centering
    \includegraphics[width=0.8\textwidth]{imagenes/leafvar.png}
    \caption{Variaciones de la firma espectral con el área foliar.\footfullcite{asner1998biophysical}}
    \end{figure}
\end{frame}
%--- Next Frame ---%

\begin{frame}{Modelado}
    \begin{figure}
    \centering
    \includegraphics[width=0.8\textwidth]{imagenes/vivomuerto.png}
    \caption{Firma espectral de la vegetación en diferentes estados.\footfullcite{asner1998biophysical}}
    \end{figure}
\end{frame}
%--- Next Frame ---%

\begin{frame}{Modelado}
    \begin{figure}
    \centering
    \includegraphics[width=0.8\textwidth]{imagenes/waterm.png}
    \caption{Firma espectral de agua con distinto contenido de arcilla disuelta.\footfullcite{clark2007usgs}}
    \end{figure}
\end{frame}
%--- Next Frame ---%

\begin{frame}{Modelado}
    \begin{figure}
    \centering
    \includegraphics[width=0.6\textwidth]{imagenes/soilvar.png}
    \caption{Firma espectral del suelo con distintos contenidos de humedad.\footfullcite{liang2005quantitative}}
    \end{figure}
\end{frame}
%--- Next Frame ---%

\section{Práctica}

\begin{frame}{Práctica}
  \begin{exampleblock}{Actividades prácticas de la primera clase}
    \begin{enumerate}
      \item Abrir imágenes Landsat 8 y familiarizarse con el SoPI.
      \item Digitalizar coberturas uniformes dentro de la imagen.
      \item Extraer la firma espectral de las coberturas digitalizadas.
      \item Reescalar las firmas obtenidas y compararlas para dos imágenes distintas.
    \end{enumerate}
  \end{exampleblock}
\end{frame}
%--- Next Frame ---%

\end{document}
