\documentclass[a4paper,12pt]{article}

% Paquetes para usar bien el idioma español
\usepackage[spanish,es-tabla]{babel}
\selectlanguage{spanish}
\usepackage[utf8]{inputenc}

% Paquetes para usar mejores imagenes
\usepackage{graphicx}

% Paquetes para links y tabla de contenidos en el PDF
\usepackage{hyperref}
\hypersetup{colorlinks=true,allcolors=blue}
%\usepackage{hypcap}

% Paquetes para mejores tablas
\usepackage{booktabs}

% Mejor matematica
\usepackage{amsmath}

% Fuentes de las imagenes
\usepackage[absolute,overlay]{textpos}

% Paquete captions
\usepackage[justification=centering,labelformat=empty,labelsep=none]{caption}

% Opciones para ticks
\usepackage{tikz}
\usetikzlibrary{shapes,arrows,positioning}

\tikzstyle{decision} = [diamond, draw, fill=blue!20, text width=4em, text badly centered, node distance=2cm, inner sep=0pt,on grid]
\tikzstyle{block} = [rectangle, draw, fill=blue!20, text width=8em, text centered, rounded corners, minimum height=2em,on grid]
\tikzstyle{line} = [draw, -latex]

% Citas bibliograficas
\usepackage[backend=biber]{biblatex}
\renewcommand{\footnotesize}{\tiny}
\addbibresource{biblio.bib}

% Mejoro las captions
\setbeamertemplate{caption}{\raggedright\insertcaption\par}

\setbeamertemplate{caption}{%
\begin{beamercolorbox}[wd=0.85\paperwidth, sep=.2ex]{block body}\insertcaption%
\end{beamercolorbox}%
}


% Sacar barra de navegacion
\setbeamertemplate{navigation symbols}{}%remove navigation symbols

% Transparencias en items
\setbeamercovered{transparent}

% Estilo de diapositivas
% \usetheme{Boadilla}
\usecolortheme{whale}
\usecolortheme{orchid}

\newtheorem{dex}{Definicion}[section]
\newtheorem{exa}{Ejemplo}[section]
\newtheorem*{obs}{Observacion}

\title{Curso SoPI 2: Herramientas de teledetección cuantitativa}
\author{Francisco Nemiña}

\begin{document}
\maketitle
\section*{Introducción}

Buen día. Este es el apunte del curso SoPI 2 de herramientas de teledetección
cuantitativa. La idea es que este apunte reproduzca el material de las clases
basado en el curso del segundo cuatrimestre de 2016.

El apunte va a ser lo mas incluso posible pero claramente incompleto por lo
tanto es bueno complementarlo con otros libros/guías/papers/etc. Al final de
cada capítulo citaremos las fuentes que sean releveantes para el mismo. Sin
embargo, citar todas las fuentes es claramente imposible por lo tanto
incentivamos a los alumnos a llevar adelante su propia busqueda y ser crítico de
las mismas.

El curso va a trabajar utilizando el concepto de \emph{vector pixel} para
estudiar distintas herramientas de teledeteccion cuantitativa. La forma de tocar
estas herramientas es desde un punto de vista introductorio. Intentando siempre
motivar el porque se introducen y que aplicacion tienen, pero no siempre
demostrando su validez matematica.

A grandes rasgos el curso se divide en dos partes. La primera llamada
\emph{transformaciones en el dominio espectral} incluye los capitulos 1 a 4 y
toca los conceptos basicos de la teledeteccion, firmas espectrales, vectores
pixeles, correcciones radimetricas y atmosfericas, indices espectrales y
transformaciones en el espacio espectral. La segunda llama \emph{clasificaciones
supervisadas en la practica} incluye los capitulos 5 a 7 y cubre distintos
metodos de clasificacion supervisada con su correspondiente validacion.

Durante el curso que funciono de manera teorico-practica se trabajo con un caso
de estudio de la region de Triple Frontera, en Misiones. Al final de cada
capitulo se incluira las actividades sugeridas para el mismo mas algunos
ejercicios que pueden aportar a la compresion de los conceptos espuestos.

Como concentrarnos en un tema permite siplificar las cosas, en el curso
trabajamos en la parte optica del espectro. Es decir, de los $0.4\mu mu$ a $14.0
\mu m$ incluyendo el visible, infrarrojo cercano, medio y termico. Las imagenes
que se utilizaron fueron en su mayoria Landsat 8 del sensor OLI, pero los mismos
conceptos pueden aplicarse a imagenes de otras fuentes.

Como cualquier producción de un texto, este no va a estar libre de defectos. Les
pedimos por favor que ante cualquier problema escriban a
\url{fnemina@conae.gov.ar} con los comentarios pertinentes para dar una solución
lo mas pronta posible al problema.

\section{Un viaje del sol a los píxeles}

\subsection{Introducción}
Este capitulo responde a los contenidos de la primer clase del curso. La idea es
brindar un recorrido que parta de la energía proveniente del sol y termine
definiendo el concepto de vector píxel. El mismo supone algunos conocimientos
basicos sobre fisica y matematica pero no muchos mas que los correspondientes a
un graduado universitario. Los conceptos de calor, energia y optica sera de
utilidad.

Ademas seria buena idea repasar los conceptos de algebra lineal -muchas veces
llamado geometria analitica o calculo en algunas universidades/carreras- pues
los mismos seran de utilidad para todo el curso.

\subsection{Ondas electromagneticas}

Comencemos definiendo algunos conceptos importantes. El primero de los mismos
sera el de energia. Dar una definicion concreta de este concepto puede llevar
bastante tiempo\footnote{la carrera de fisica dura 5 años} por lo tanto daremos
una definicion amplia.
\begin{dex}
  Diremos que un cuerpo posee una energia $E$ si el cuerpo tiene una capacidad de
  ejercer un trabajo $W$.
\end{dex}

\begin{obs}
  En realidad esta es la definicion en realidad es mas cercana a la de energia
  libre.
\end{obs}

Como dijimos antes estas definiciones no son ciento por ciento correctas. Y en
realidad hablar de la energia que posee un cuerpo, en abstracto, no tiene mucho
sentido. Sin embargo si sera interesante las formas de un cuerpo de intercambiar
energia. Las mismas son basicamente dos

\begin{itemize}
  \item Trabajo
  \item Calor
\end{itemize}

la primera de ella nos permite intercambiar energia entre cuerpo por medio de
fuerzas. La segunda, nos permite enmascarar formas de transferencia de energia
para las cuales no conocemos en detalle las fuerzas intervinientes.

Dentro del calor, hay tres nuevas clasificaciones para la transferencia de
energia. Conveccion, cuando hay movimiento de materia, conduccion, cuando hay
contacto entre los cuerpos pero no movimiento de materia, y radiacion, cuando no
hay contacto entre los cuerpos.

\begin{exa}
  Ejemplos de transferencia de calor en la vida cotidiana hay varios y la cocina
  se presentan varios. Al cocinar un asado, por ejemplo, la energia se trasmite
  transmite por conveccion a travez del aire hacia el asado, de radiacion entre
  el carbon al rojo vivo y conduccion, entre las barritas de hierro y el asado.
\end{exa}

De las tres la mas importante para la teledeteccion sera la transferencia de
energia por radiacion. En particular de ondas electromagneticas. La descripcion
de onda del electromagnetismo es una de las dos que existen en la actualidad
para describir a la luz. En esta descripcion para describir al campo
electromagnetico se describe por una onda transversal donde el campo electrico y
magnetico son perpendiculares. Como toda onda que se propaga la misma esta
descripta por su \emph{amplitud}, su \emph{frecuencia} y su \emph{longitud de
onda}.

\begin{equation}
  E(x,t) = A \times \cos(2\pi \nu t - 2\pi x / \lambda)
\end{equation}

donde la frecuencia y la longitud de onda se relacionan por la llamada
\emph{relacion de dispersion}

\begin{equation}
  c = \lambda \nu
\end{equation}

siendo $c\sim 3\times 10^8 m/s$ la velocidad de la luz en el vacio. Definimos
entonces 

\begin{dex}
  Hablamos de frecuencia cuando hablamos de la cantidad de oscilaciones que
  realiza una onda en un determinado periodo de tiempo. Las unidades de la
  frecuencia en el Sistema Internacional son Hz.
\end{dex}

\begin{dex}
  Hablamos de longitud de onda cuando hablamos de la distancia entre dos maximos
  para una onda. La unidad de longitud de onda en el Sistema Internacional es m.
\end{dex}

\begin{dex}
  Hablamos de amplitud cuando hablamos del maximo valor que toma el campo
  electrico o magnetico en un punto. La unidad dependera de si hablamos de campo
  electrico o magnetico
\end{dex}

Esta descripcion tiene varias particularidades interesantes que nos resultaran
utiles en teledeteccion.

La primera es que la energia que transfiere la onda sera proporcional a su
amplitud al cuadrado. Por lo tanto dentro de nuestro trabajo medir la amplitud o
medir la energia sera escencialmente lo mismo.

Por otro lado, podremos clasificar a las ondas en funcion de su longitud de
onda\footnote{O su frecuencia. Las mismas estan ligadas por la relacion de
dispersion por lo tanto hablar de una o la otra sera lo mismo.}. A distintas
longitudes de onda le corresponderan distintas propiedades segun como sea
generada y como interactue con la materia. Es comun clasificarlas entonces
dentro del llamado espectro electromagnetico. En el mismo se agrupan las ondas
electromagneticas segun propiedades comunes en rangos de longitudes de onda.

Nosotros en el curso nos centraremos solamente en dos de estos rangos. El
visible que se encuentra entre los $0.45 \mu m$, correspondiendo al extremo azul
del espectro, y los $0.75 \mu m$ correspondiente al extremo rojo del expectro. Y
el infrarrojo, hubicado entre los $0.75 \mu m$ y los $1000 \mu m$
aproximadamente. 

\begin{obs}
  Los nombres de las zonas del espectro electromagnetico fueron dados en funcion
  de propiedades en la generacion o medicion de las mismas. El visible por ser
  la zona donde el ojo humano es capaz de captar luz. El infrarrojo y el
  ultravioleta mas alla del rojo y el violeta respectivamente. Los rayos X por
  ser desconocido su origen en su momento. Las ondas de radio por ser las que se
  utilizaron para la radio. Y asi cada una.
\end{obs}

Durante el curso nos centraremos en la zona que va de los $0.4 \mu m$ a los $2.7
\mu m$. Es decir el visible, el infrarrojo cercano y el infrarrojo de onda
media.

\begin{obs}
  La otra descripcion de la radiacion electromagnetica es la de particula. En la
  misma se trabaja con particulas llamadas \emph{fotones} como mediadoras del
  campo electromagnetico. No utilizaremos dicha descripcion en este curso pero
  es interesante si nos interesa conocer como interactura la luz con la materia
  mas en detalle.
\end{obs}

La energia como magnitud a medir en teledeteccion presentara una serie de
problemas que intentaremos solucionar a continuacion. Pero es el punto de
partida para entender mucho de lo que viene atras.

\subsection{Radiancia}

Veamos entonces como solucionar estos problemas. En primer lugar, la cantidad de
energia recibida por el sensor\footnote{Ya sea satelital u de otro origen} es
proporcional al tiempo en que la recibe. Si tomamos la energia como unidad
fundamental para nuestra descripcion entonces un sensor con un tiempo de
integracion mayor nos dara valores del doble. Para sacarnos este problema de
encima trabajemos de aqui en adelante con la potencia en lugar de la energia.

Para obtener la \emph{potencia} solo debemos calcular el cociente entre la
cantidad de energia recibida en un cierto lapso de tiempo y dicho lapso. Es
decir

\begin{equation}
  P = \frac{\Delta E}{\Delta t}
\end{equation}

La unidad de potencia en el sistema internacional es el $W$.

\begin{obs}
  De aqui en adelante consideraremos siempre un $\Delta$ lo suficientemente chico
  como para que valgan las relaciones. Haciendo honor a la verdad la definicion
  deberia tomarse en el limite de $\Delta$ muy chico.
\end{obs}

La potencia sera por ahora una caracteristica de la fuente. Como segunda
preocupacion es de esperar que si miro un area mas grande la potencia que la
atraviesa sera mayor. Por lo tanto un sensor que mire un area mayor o menor vera
cosas distintas. Para sacarnos este segundo problema de encima calcularemos la
\emph{densidad de potencia}. En este caso lo que miraremos es cual es la
potencia por unidad de area como vemos a continuacion

\begin{equation}
  p = \frac{P}{A} = E
\end{equation}

en el sistema internacional, la unidad de densidad de potencia es $W/m^2$.

Para el caso de una fuente isotropa\footnote{Cuya distribucion no depende del
angulo} la densidad de potencia decaera con el cuadrado de la distancia. Es
decir que si duplicamos la distancia a la fuenta la potencia se vera disminuida
a la cuarta parte.

\begin{exa}
  Veamos un caso concreto. La densidad de potencia del sol en la superficie
  terrestre sera de $1.353 kW/m^2$. Si pudieramos medir la densidad de potencia
  en marte la misma estaria dada por
  $$ p_{marte} = p_{tierra} \frac{4\pi r_{tierra}^2}{4\pi r_{marte}^2}= 1.353
  kW/m^2 \frac{1^2}{1.5^2} = 601 W/m^2$$
  tomando las distancias entre la tierra y el sol en unidades astronomicas.
\end{exa}

\begin{obs}
  La densidad de potencia es un nombre que no suele usarse en teledeteccion.
  Hablaremos en general de intensidad o de \emph{irradiancia}.
\end{obs}

Esta distribucion dependera ademas de que zona del espectro electromagnetico
estemos mirando. No sera lo mismo la irradiancia recibida si nuestro sensor mida
la cantidad de luz que llega en el visible que la que llega en el infrarrojo.

En el caso del sol, la energia proveniente del mismo esta modelada bastante bien
por la curva de radiacion de un cuerpo negro con una temperatura superficial de
$5500 K$. En dicha curva vemos que el maximo se encuentra cerca de la longitud
de onda correspondiente al color amarillo y que decae mas rapidamente hacia el
sectol del ultravioleta que hacia el infrarrojo.

\begin{obs}
  Al mirar esta curva es interesante notar porque nos centramos en medir estas
  zonas del espectro electromagnetico y no otras.
\end{obs}

Esta curva estara afectada ademas por el medio interestelar y la atmosfera
terrestre dando iluminaciones muy distintas en cada zona. Para sacarnos entonces
de arriba el problema de la dependencia espectral definiremos

\begin{dex}
  Llamamos irradiancia espectral a la funcion de distribucion de irradiancia
  segun la longitud de onda. En el sistema internacional la misma se medira en
  $W m^{-2} \mu m^{-1}$
\end{dex}

Vemos en la siguiente tabla algunos valores de irradiancia espectral que, como
se espera, decae a medida que nos vamos hacia la zona del infrarrojo.

TABLA

Finalmente, y mas importante de todo, no todas las fuentes seran isotropas. Para
explicar su comportamiento podemos ver como varia la irradiancia en funcion del
angulo solido que estemos mirando entorno a la fuente. Al definir la irradiancia
espectral por unidad de angulo solido terminamos teniendo una magnitud que es
independiente del tiempo de integracion del sensor, el area que estemos
sensando, la longitud de onda donde el sensor tome sus mediciones  y el angulo
que estamos mirando. Dicha magnitud sera la que mas usaremos en teledeteccion.

Definiendola propiamente tenemos

\begin{dex}
  La irradiancia espectral es la radiancia espectral por unidad de angulo solido
  \begin{equation}
    L_\lambda = \frac{p_\lambda}{\Delta \Omega \cos \theta_z}
  \end{equation}
  En el sistema internacional la irradiancia espectral se mide en $W m^{-2} \mu
  m^{-1} sr^{-1}$.
\end{dex}

donde $\theta_z$ es el angulo zenital\footnote{medido con respecto a la
vertical}. 

\begin{obs}
  Alternativamente podemos definir a la radiancia espectral como $L_\lambda$ tal
  que
  \begin{equation}
    dQ = L_\lambda(\theta,\phi) \cos \theta_z d\Omega dA dt d\lambda
  \end{equation}
\end{obs}

\subsection{Reflectancia}
Tenemos una magnitud ya digna de ser medida por un satelite. Sin embargo la
radiancia presenta otro problema. Al cambiar la fuente de iluminacion cambiamos
la radiancia espectral recibida como se ve en la figura.

Queremos por lo tanto poder independizarnos de la fuente de iluminacion. Para
esto diremos que la radiancia reflejada por un cuerpo se relacionada con la
irradiancia por una funcion que depende solo de los angulos y el cuerpo en
cuestion.

\begin{dex}
  Llamamores distribucion de reflectancia bidirectional espectral (BRDF) a la
  funcion $f(\theta_i, \phi_i, \theta_r, \phi_r)$ tal que la radiancia reflejada
  y la irradiancia incidente estan relacionadas por
  \begin{equation}
    f dE = dL
  \end{equation}
\end{dex}

Esta funcion dependera solamente del cuerpo en cuestion y los angulos de
incidencia y refleccion, pero no de la intensidad de la iluminacion. Si
suponemos que la intensidad de la iluminacion es constante podemos definir a
partir de la misma la \emph{reflectancia direccional} como 

\begin{equation}
  R = \frac{\pi L}{\cos\theta_i E_0} = \pi f
\end{equation}

\begin{exa}
  Veamos un ejemplo interesante de reflectancia bidireccional antes de abandonarla
  por el resto del curso.

  Se puede apreciar en la siguiente figura un parque donde el color del cesped
  varia por franjas. Esto se debe a que la funcion de reflectancia direccional
  del mismo no es constante. Por lo tanto al pasar una maquina que corta el
  pasto en un sentido y en el otro termina orientado de forma distintas y
  viendose de distintos colores.

  Si medidos la funcion de reflectancia bidireccional como funcion del angulo
  obtenemos la siguiente

  vemos que la misma es bastante complicada en general pero existe una magnitud
  remarcable. Para un angulo en particular la reflectancia es mucho mayor que
  para todo los demas. Dicho angulo se corresponde con el angulo de reflectancia
  especular para el sol y suele pasar en la mayoria de las superficies. Muchas
  veces se lo llama efecto de nucleo espectral.

  La otra superficie corresponde a un espectralon. Superficie para la cual la
  funcion de reflectancia direccion depende poco del angulo.
\end{exa}

Podemos entonces clasificar a nuestras superficies segun que tan fuertemente
dependar del angulo de incidencia. Dos casos extremos se ven en la siguiente
figura

En el primero vemos que el angulo de refleccion es igual al angulo de
incidencia. Hablamos en este caso de una refleccion especular. En el ultimo
vemos que la reflectancia no depende del angulo de incidencia. Hablamos en este
caso de una superficie \emph{lambertiana}. 

\begin{dex}
  Hablaremos de una superficie lambiertiene cuando la funcion de reflectancia
  direccional sea constante. Es decir, que la radiancia reflejada no dependa del
  angulo. En este caso
  \begin{equation}
    \rho = \frac{\pi}{\cos \theta_z E_0}
  \end{equation}
\end{dex}

Durante el resto del curso trabajaremos solamente con reflectancias
lambertianas\footnote{Es importante conocer que esto es una aproximacion. Si no
lo hacemos podemos encontrarnos frente a errores en nuestra interpretacion de
las imagenes que no sabemos de donde provienen.}.

\subsection{Firmas espectrales}

Una vez definida la reflectancia ya podemos empezar a estudiar como es la
reflectancia para distintos cuerpos. Para hacer esto introduciremos un concepto
nuevo que sera unificador para el trabajo en teledeteccion optica.

\begin{dex}
  Llamaremos \emph{firma espectra} a la funcion $f(\lambda)$ que nos dice la
  reflectancia de una cobertura o cuerpo para cada longitud de onda y la
  notaremos $\rho_\lambda$.
\end{dex}

Es importante remarcar que distintos cuerpos tendran distintas firmas
espectrales y que las mismas dependeran del comportamiento biofisico y geofisico
del mismo. Es decir, cambiando magnitudes como la consentracion de minerales,
estructura foliar, contenido de humedad, etc.\ cambiaremos la firma espectral.

\begin{obs}
  Ademas de la reflectancia hay otras dos magnitudes que describen el
  comportamiento de un cuerpo: la emitancia y la transmitacia. No estudiaremos
  los mismos en este curso.
\end{obs}

\begin{obs}
Nos queda entonces estudiar firmas estectrales para distintas coberturas del
suelo teniendo como objetivo conocerlas para luego estudiar lo que le pasa al
mismo. Durante el curso, haremos dicho analisis de forma cualitative. Sin
embargo es posible modelar desde primeros principios el comportamiento de una
firma espectral y luego, invirtiendo este comportamiento, obtener variables
cuantitativas sobre el terreno. Esto excede los objetivos del curso pero pidan y
bibliografia les sera dada.
\end{obs}

\subsubsection{Vegetacion}
Para la firma espectral de la vegetacion veremos la variacion en 3 zona del
espectro.

Para la parte visible observamos en la figura que a medida que aumenta el
contenido de clorofila a y b la reflectancia se vuelve cada vez menor. Este
comportamiento es esperable pues la clorofila absorve en la longitudes de onda
del visible cercanas al rojo y al azul.

Ademas vemos que la variacion en el contenido de clorofila\footnote{o pigmentos
en general} no afecta a la firma espectral en el resto del esctro por lo tanto
si nuestra imagen presenta variaciones solamente en la zona del visible sabremos
que se debe a esto.

En seguno lugar vemos que el contenido de agua de la vegetacion hacie que la
reflectancia cambie solo en la zona del infrarrojo medio. A medida que aumenta
dicho cntenido de agua la reflectancia disminuye y viceversa.

Ademas, como en el caso anterior, el cambio de contenido de agua no afecta a la
zona del infrarrojo cercano y el visible. Por lo tanto si la vegetacion se
encuentra bajo stress hidrico veremos el cambio primero en la zona del
infrarrojo medio antes que en el infrarrojo cercano.

Finalmente, los cambios en el area foliar de la vegetacion se ven reflejados
principalmente en la zona del infrarrojo cercano y medio. Aumentando la
reflectancia en el infrarrojo cercano a medida que aumenta el area foliar.

Estos comportamientos generales se reproducen para todas las especies de
vegetacion, cambiando la dependencia con los parametros pero no la forma
funcional.

Finalmente, como vemos en esta imagen. La vegetacion que pasa de estar
fotosinteticamente activa a estar muerta va cambiando lentamente su firma
espectral entre la curva verde y la curva roja. Por lo tanto estudiar como
cambia la firma espectral de la vegetacion en el tiempo nos hablara de su estado
fenologico.

\subsubsection{Suelo}

En el caso de la firma de suelo nos centraremos en que pasa con el contenido de
humedad.

Vemos en este caso que la firma espectral del suelo cambia en todo el espectro a
medida que variamos el contenido de humedad. Sin embargo los cambios principales
se dan en la zona del infrarrojo cercano.

En particular, las bandas de absorcion del agua y los cationes hidroxiles
cambian muy fuertemente con el contenido de humedad. Por lo tanto estudiando
dichas zonas podemos sacar de forma cuantitativa cual es el contenido de humedad
superficial del suelo.

\begin{obs}
  Vale hacer la aclaracion que todas las magnitudes medidas en teledeteccion
  optica son superficiales. Uno puede hacer inferencias sobre el comportamiento
  debajo de la superficie, pero no mas que eso.
\end{obs}

\subsubsection{Agua}
Para el caso del agua la firma espectral de agua pura sera casi cero para todo
el espectro. Esto es especialmente cierto para la region del infrarrojo cercano
y medio del mismo. Por lo tanto las variaciones en su firma espectral vendran
asociadas a cambios en el contenido de sedimentos y el tipo de los mismos como
se ve en la figura.

Vemos que la reflectancia de la misma aumenta a medida que disolvemos mas
arcilla hasta el punto de saturacion. Nuevamente, este comportamiento nos
permite realizar inferencias sobre la composicion de la solucion de agua en
cuestion y el tipo de sedimentos presenten\footnote{Hacerlo efectivamente es
otro tema.}.

Como ultimo ejempo de firma espectral mostramos al agua en otros estados de
agregacion. En particular vemos que tanto para las nubes como para la nueve la
reflectancia en la zona del espectro visible es siempre alta.

Para el caso de la nieve podemos distinguir la granularidad de la misma mirando
como es la reflectancia en la region del infrarrojo cercano en relacion al
visible. Notamos en este caso que a medida que el tamaño del mismo aumenta la
reflectancia en el infrarrojo cercano disminuye. 

Ademas observar lo que sucede en el infrarrojo medio nos permite separar nubes
de nieve en una imagen notando que las nubes tienen siempre una reflectancia
mayor que la nieve.

\subsection{Respuesta espectral}

Por lastima los sensores que uno usara habitualmente no mediran todo el espectro
electromagnetico. Tampoco es que esto tenga sentido, como vimos arriba hay zonas
de la atmosfera donde la iluminacion solar es cero y por lo tanto la radiancia
en dichas zonas sera cero.

En general los sensores mediran una zona acotada del espectro en funcion de cual
sea el objetivo de la medicion de interes. En nuestro caso, hablaremos de la
respuesta espectral de un sensor segun como pese la radiancia recibida en
funcion de la longitud de onda.

Como vemos en el siguiente caso para Landsat 8, la respuesta espectral de un
sensor esta definida principalmente por dos cosas. Por un lado el centro de
banda que nos dice donde esta midiendo el sensor en el espectro. Por otro el
ancho de banda que nos dice cuanto hacia los lados el sensor esta midiendo.
Cuanto menor sea el ancho de banda de un sensor, mayor su resolucion espacial y
por lo tanto mas sensible a los cambios en una determinada longitud de donda

\begin{obs}
  En funcion de la cantidad de bandas de las que disponga un sensor para medir
  en el espectro hablaremos de sensores multiespectrales o hiperespectrales.
\end{obs}

Matematicamente podemos describir al sensor como una distribucion de pesos
$s_j(\lambda)$ donde lo que calculamos al realizar la medicion es el valor
esperado de la firma espectral bajo esa distribucion $s_j$. 

\begin{obs}
  Hablaremos de resolucion espectral de un sensor cuando queramos hablar de que
  tan bien separa el mismo zonas del espectro.
\end{obs}  

De esta manera estamos discretizando a la firma espectral y pasamos de una
funcion continua a una serie de valores discretos\footnote{Lo mismo pasara
para la descripcion espacial, temporal y radiometrica. Cada una con su
correspondiente resolucion espectral.}.

\subsection{Vector pixel}

Al discretizar de esta forma el espectro combiene introducir una notacion
matematica para describir a los pixeles. En el caso de que el satelite tenga una
sola banda hablaremos de un pixel como un pixel escalar.

Para el caso de satelites de mas de una banda, hablaremos de un pixel como un
\emph{vector pixel} dado por

\begin{equation}
  \rho = (\rho_1, \ldots, \rho_N)
\end{equation}

donde cada componente corresponde al valor de reflectancia de una banda en
concreto.

Como vemos en el siguiente caso para landsat 7 la firma queda discretizada en 7
valores donde cada uno esta dado por el valor esperado de la reflectancia con
respecto al filtro de cada sensor. Valores mas altos de reflectancia en una zona
de la firma espectral se convertiran en valores mas altores de reflectancia en
la componente del vector pixel.

Al hacer esto, es importante remarcar que estamos perdiendo datos. Segun la
aplicacion de interes dichos datos pueden ser relevantes o no. 

Definimos este concepto pues nos permite comenzar a pensar a los valores de los
pixeles de otra forma. Como vemos en este ejemplo con solo dos bandas

los pixeles pertenecientes a coberturas similares se agrupan en regiones del
espacio que llamaremos espacio espectral.

Ademas, al cambiar una cobertura su firma espectral veremos que el pixel se
desplaza dentro de este espacio y en funcion de como es dicho desplazamiento
podremos extraer informacion sobre la cobertura como se ven a continuacion.

A medida que pase el curso veremos como a partir del concepto de espacio
espectral podremos encontrar nuevas formas de interpretar la informacion y de
extraer datos de las imagenes.
\end{document}
