\chapter{Cronograma}\label{chap:cronograma}

Cronograma del curso.

\begin{itemize}
  \item[25/4] Instalar el QGIS y el R. Ver apendice \ref{chap:instalacion}.
  \item[26/4] \nameref{viaje}
  \begin{itemize}
    \item Clase te\'orica: radiancia y reflectancia. Firmas espectrales y espacio espectral.
    \item Clase pr\'actica: capitulo \ref{viaje} de la guida pr\'actica.
    \item Lectura recomendada: Remote sensing digital Image Analysis - John A. Richards. Capitulos 1 y 3
  \end{itemize}
  \item[2/5] Entrega del cuestionario 1. Encuesta de inicio del curso.
  \item[3/5] \nameref{rebotando}
  \begin{itemize}
    \item Clase te\'orica: Transferencia radiativa. M\'etodos estad\'isticos y modelado de la atm\'osfera.
    \item Clase pr\'actica: capitulo de \ref{rebotando} la guia pr\'actica.
    \item Lectura recomendada: Remote sensing digital Image Analysis - John A. Richards. Capitulos 2
  \end{itemize}
  \item[9/5] Entrega del cuestionario 2.
  \item[10/5] \nameref{abaco}
  \begin{itemize}
    \item Clase te\'orica: C\'alculo de \'indices espectrales
    \item Clase pr\'actica: capitulo de \ref{abaco} la guia pr\'actica.
    \item Lectura recomendada: Quantitative Remote Sensing - ShunLin Liang. Capitulos 8
  \end{itemize}
  \item[16/5] Entrega del cuestionario 3.
  \item[17/5] \nameref{rotaciones}
  \begin{itemize}
    \item Clase te\'orica: Transformada Tasseled Cap y An\'alisis por componentes principales.
    \item Clase pr\'actica: capitulo de \ref{rotaciones} la guia pr\'actica.
    \item Lectura recomendada: Remote sensing digital Image Analysis - John A. Richards. Capitulos 6
  \end{itemize}
  \item[23/5] Entrega del cuestionario 4.
  \item[24/5] Clase de consulta.
  \item[30/5] Entrega del trabajo pr\'actico 1.
  \item[31/5] \nameref{otrolado}
  \begin{itemize}
    \item Clase te\'orica: M\'etodos de clasificaci\'on no supervisados.
    \item Clase pr\'actica: capitulo de \ref{otrolado} la guia pr\'actica.
    \item Lectura recomendada: Remote sensing digital Image Analysis - John A. Richards. Capitulos 9
  \end{itemize}
  \item[6/6] Entrega del cuestionario 5.
  \item[7/6] \nameref{educando}
  \begin{itemize}
    \item Clase te\'orica: M\'etodos de clasificaci\'on supervisados.
    \item Clase pr\'actica: capitulo de \ref{educando} la guia pr\'actica.
    \item Lectura recomendada: Remote sensing digital Image Analysis - John A. Richards. Capitulos 8
  \end{itemize}
  \item[13/6] Entrega del cuestionario 6.
  \item[14/6] \nameref{pos}
  \begin{itemize}
    \item Clase te\'orica: Tecnicas pos-clasificaci\'on.
    \item Clase pr\'actica: capitulo de \ref{pos} la guia pr\'actica.
    \item Lectura recomendada: Making better use of accuracy data in land change studies: Estimating accuracy and area and quantifying uncertainty using stratified estimation - Olofsson et al.
  \end{itemize}
  \item[20/6] Entrega del cuestionario 7.
  \item[21/6] Clase de consulta.
  \item[27/6] Entrega del trabajo pr\'actico 2.
  \item[30/6] Encuesta de fin del curso.
\end{itemize}

\chapter{Categor\'ias de uso y cobertura del suelo}\label{apcate}
Categor\'ias de uso y cobertuar segun el esquema LCCS2 de la FAO\@. Los colores son sugerencias por categor\'ia.
\begin{table}[hbt]
    \centering
    \begin{tabular}{p{11cm}cc}
        \toprule
        Nombre & Codigo & Color \\
        \midrule
        Áreas terrestres cultivadas y manejada & A11 & \textcolor{A11}{$\blacksquare$}\texttt{\#b2df8a}
        \\
        Vegetación natural y semi-natural & A12 & \textcolor{A12}{$\blacksquare$}\texttt{\#33a02c}\\
        Áreas acuáticas o regularmente inundadas cultivadas & A23  &
        \textcolor{A23}{$\blacksquare$}\texttt{\#fdbf6f}\\
        Vegetación natural y semi-natural acuática o
	regularmente inundadas & A24 & \textcolor{A24}{$\blacksquare$}\texttt{\#ff7f00}\\
        Superficies artificiales y áreas asociadas & B15  &
        \textcolor{B15}{$\blacksquare$}\texttt{\#fb9a99}\\
        Áreas descubiertas o desnudas & B16 & \textcolor{B16}{$\blacksquare$}\texttt{\#e31a1c}\\
        Cuerpos artificiales de agua, nieve y hielo & B27 &
        \textcolor{B27}{$\blacksquare$}\texttt{\#a6cee3}\\
        Cuerpos naturales de agua, nieve y hielo & B28&
        \textcolor{B28}{$\blacksquare$}\texttt{\#1f78b4}\\
        \bottomrule
    \end{tabular}
\caption{\label{tabla1}Categorias usos del suelo segun el esquema LCCS2 de la
FAO.}
\end{table}

\chapter{Instalacion de qgis y R} \label{chap:instalacion}

Veamos como instalar Q-GIS y R en Windows 10 y Ubuntu 16.04. De la misma forma deberia poder instalarse en otro sistemas operativos (Windows 7, 8 y 8.1) y GNU/Linux.

\section{Instalaci\'on en Windows 10}

Para realizar la instalacion en Windows 10, descargamos e instalamos en el siguiente orden cada uno de los programas a utilizar.

\begin{enumerate}
  \item Q-GIS 2.14 o 2.18 de la pagina oficial de QGIS \url{http://www.qgis.org/es/site/}.
  \item R 3.3 o 3.4 de la pagina de oficial del proyecto R \url{https://cran.r-project.org/bin/windows/base/}.
  \item R-Studio en su version gratuita de la pagina oficial \url{https://www.rstudio.com/products/rstudio/download/}.
\end{enumerate}

Siempre en sus instalaciones por defecto y debemos instalarlas en el orden  indicado.

Una vez finalizada, abrimos R-Studio y ejecutamos el siguiente comando en la consola.

\begin{lstlisting}
  install.packages(c("raster","lattice","RStoolbox","rasterVis","rgal","e1071","randomForest","kernlab"))
\end{lstlisting}

El mismo descargara todos los paquetes que utilizaremos durante el curso y los instalara. Si la instalaci\'on termina sin ningun error, estamos listos para comenzar a trabajar.

\section{Instalaci\'on GNU/Linux}

Para instalar los programas necesarios en linux debemos seguir los siguientes tutoriales para instalas Q-GIS, R y R-Studio.

\begin{enumerate}
  \item Q-GIS seguimos las instrucciones para instalar Q-GIS 2.14 o 2.18 \url{http://www.qgis.org/es/site/forusers/alldownloads.html#linux}.
  \item R segun nuestra distribucion. Para el caso de Ubuntu podemos utilizar \texttt{sudo apt install r-base} y para Fedora podemos utilizar \texttt{yum install R}. En caso de tener otra distribucion preguntar en el foro.
  \item R-Studio en su version gratuita de la pagina oficial \url{https://www.rstudio.com/products/rstudio/download/}.
\end{enumerate}

Una vez finalizada la instalacion, abrimos R-Studio y ejecutamos el siguiente comando en la consola.

\begin{lstlisting}
  install.packages(c("raster","lattice","RStoolbox","rasterVis","rgal","e1071","randomForest","kernlab"))
\end{lstlisting}

En este caso descargara y recompilara cada uno de los paquetes necesarios para nuestra distribuci\'on.

\section{M\'aquina virtual}

En caso de no querer realizar la instalaci\'on, es posible utilizar la m\'aquina virtual provista por el \href{www.beeoda.org}{Boston Education in Earth Observation Data Analysis} siguiendo las instrucciones en su repositorio de github \url{https://github.com/beeoda/opengeo-vm}.
