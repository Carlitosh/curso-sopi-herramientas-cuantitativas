\chapter{Instalacion de qgis y R}

Veamos como instalar Q-GIS y R en Windows 10 y Ubuntu 16.04. De la misma forma deberia poder instalarse en otro sistemas operativos (Windows 7, 8 y 8.1) y GNU/Linux.

\section{Instalacion en Windows 10}

Para realizar la instalacion en Windows 10, descargamos e instalamos en el siguiente orden cada uno de los programas a utilizar.

\begin{enumerate}
  \item Q-GIS 2.14 o 2.18 de la pagina oficial de QGIS \url{http://www.qgis.org/es/site/}.
  \item R 3.3 o 3.4 de la pagina de oficial del proyecto R \url{https://cran.r-project.org/bin/windows/base/}.
  \item R-Studio en su version gratuita de la pagina oficial \url{https://www.rstudio.com/products/rstudio/download/}.
\end{enumerate}

Siempre utilizaremos sus instalaciones por defecto y debemos instalarlass en el orden arriba indicado.

Una vez finalizada la instalacion, abrimos R-Studio y ejecutamos el siguiente comando en la consola.

\begin{lstlisting}
  install.packages(c("raster","lattice","RStoolbox","rasterVis","rgal"))
\end{lstlisting}

El mismo descargara todos los paquetes que utilizaremos durante el cursso y los instalara para que posterior utilizacion. Si la instalacion termina sin ningun error, estamos listos para comenzar a trabajar.

\section{Instalacion GNU/Linux}

Para instalar los programas necesarios en linux debemos seguir los siguientes tutoriales para instalas Q-GIS, R y R-Studio.

\begin{enumerate}
  \item Q-GIS seguimos las instrucciones para instalar Q-GIS 2.14 o 2.18 \url{http://www.qgis.org/es/site/forusers/alldownloads.html#linux}.
  \item R segun nuestra distribucion. Para el caso de Ubuntu podemos utilizar \texttt{sudo apt install r-base} y para Fedora podemos utilizar \texttt{yum install R}. En caso de tener otra distribucion preguntar en el foro.
  \item R-Studio en su version gratuita de la pagina oficial \url{https://www.rstudio.com/products/rstudio/download/}.
\end{enumerate}

Una vez finalizada la instalacion, abrimos R-Studio y ejecutamos el siguiente comando en la consola.

\begin{lstlisting}
  install.packages(c("raster","lattice","RStoolbox","rasterVis","rgal"))
\end{lstlisting}

En este caso descargara y recompilara cada uno de los paquetes necesarios para nuestra distribucion.

\section{Maquina virtual}

En caso de no querer realizar la instalacion, es posible utilizar la maquina virtual provista por el \href{www.beeoda.org}{Boston Education in Earth Observation Data Analysis} siguiendo las instrucciones en su repositorio de github \url{https://github.com/beeoda/opengeo-vm}.

\chapter{Categorias de uso y cobertura del suelo}\label{apcate}
Categorias de uso y cobertuar segun el esquema LCCS2 de la FAO\@. Los colores son
sugerencias por categoria.
\begin{table}[hbt]
    \centering
    \begin{tabular}{p{11cm}cc}
        \toprule
        Nombre & Codigo & Color \\
        \midrule
        Áreas terrestres cultivadas y manejada & A11 & \textcolor{A11}{$\blacksquare$}\texttt{\#b2df8a}
        \\
        Vegetación natural y semi-natural & A12 & \textcolor{A12}{$\blacksquare$}\texttt{\#33a02c}\\
        Áreas acuáticas o regularmente inundadas cultivadas & A23  &
        \textcolor{A23}{$\blacksquare$}\texttt{\#fdbf6f}\\
        Vegetación natural y semi-natural acuática o
	regularmente inundadas & A24 & \textcolor{A24}{$\blacksquare$}\texttt{\#ff7f00}\\
        Superficies artificiales y áreas asociadas & B15  &
        \textcolor{B15}{$\blacksquare$}\texttt{\#fb9a99}\\
        Áreas descubiertas o desnudas & B16 & \textcolor{B16}{$\blacksquare$}\texttt{\#e31a1c}\\
        Cuerpos artificiales de agua, nieve y hielo & B27 &
        \textcolor{B27}{$\blacksquare$}\texttt{\#a6cee3}\\
        Cuerpos naturales de agua, nieve y hielo & B28&
        \textcolor{B28}{$\blacksquare$}\texttt{\#1f78b4}\\
        \bottomrule
    \end{tabular}
\caption{\label{tabla1}Categorias usos del suelo segun el esquema LCCS2 de la
FAO.}
\end{table}
