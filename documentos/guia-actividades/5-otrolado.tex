En la quinta clase del curso, trabajaremos con clasificaci\'on no supervisada
de im\'agenes satelitales. Son nuestros objetivos:

\begin{itemize}
  \item Usar la herramienta de clasificaci\'on no supervisada.
  \item Aprender a indentificar clases espectrales y asignarlas a categor\'ias de uso y cobertura.
  \item Incorporar informaci\'on no espectral a las clasificaciones como
  pueden ser datos temporales o informaci\'on espacial.
  \item Aplicar la transformada por componentes principales para reducir la dimensionalidad y seleccionar los datos mas relevantes para realizar las clasificaciones.
\end{itemize}

\section{Clasificaci\'on mediante el m\'etodo k-means}

Cargaremos primero la imagen Landsat 8 y habilitaremos la opci\'on para escribir
el header de ENVI\@.

\begin{lstlisting}
    rasterOptions(addheader = "ENVI")
    xml.2016 <- readMeta("raster_data/LC82240782016304/LC82240782016304LGN00.xml")
    ref.2016 <- stackMeta(xml.2016, quantity = "sre")
    scaleF <- getMeta(ref.2016,xml.2016, what = "SCALE_FACTOR")
    ref.2016 <- ref.2016 * scaleF
    ref.2016 <- ref.2016[[-1,]]
    names(ref.2016) <- c("blue","green","red","nir","swir1","swir2")
\end{lstlisting}

Veamos como clasificar una imagen usando el m\'etodo k-means en R. Vamos a usar
los paquetes \texttt{raster} y \texttt{RStoolbox}

\begin{exa}
    Elegimos la semilla para el geneador de n\'umeros aleatorios con el comando \texttt{set.seed(42)}. De esta forma, la serie de n\'umeros aleatorios
    es la misma para todos.

    Luego, clasificamos la imagen y la guardamos como vimos en las clases anteriores.

    \begin{lstlisting}
    set.seed(42)
    kmeans.2016 <- unsuperClass(ref.2016, nClasses = 5, nStarts = 100,
                                nSamples = 100)
    writeRaster(kmeans.2016$map, "raster_data/processed/kmeans2016",
                datatype="INT1U")
    \end{lstlisting}

    En este caso estamos usando solamente 5 clases espectrales. Podemos ahora graficar por separado cada una de las clases (figura \ref{fig:clases5}).

    \begin{lstlisting}
        clases.2016 <- layerize(kmeans.2016$map)
        plot(clases.2016)
    \end{lstlisting}

    \begin{figure}[h!]
      \centering
      \includegraphics[width=0.7\textwidth]{clases_5.png}
      \caption{Clases generadas por el algoritmo kmeans para 5 clases.}
      \label{fig:clases5}
    \end{figure}

    Abrimos la imagen en el QGIS e identificamos cada una de las clases. Para esto primero vamos al men\'u \menu{propiedades de la imagen, Estilo, Tipo de renderizacion, Unibanda pseudocolor}. Elegimos de modo \menu{Intervalo Igual} y en n\'umero de clases ponemos con el m\'inimo en 1 y el m\'aximo en 5. En estilo de color elegimos colores aleatorios.

    Construimos una tabla como la siguiente en un editor de texto.

    \begin{verbatim}
        id  class
        1   2
        2   7
        3   1
        4   1
        5   1
    \end{verbatim}

  La guardamos en un archivo de texto con el nombre \file{class.txt}.

  Una vez conocidas las categor\'ias de uso y cobertura correspondientes a cada
  clase espectral podemos combinarlas

  \begin{lstlisting}
      clases.2016 <- read.delim("aux_data/class.txt")
      reclas.2016 <- subs(kmeans.2016$map, clases.2016)
  \end{lstlisting}

  y graficarlas (figura \ref{fig:kmean5})

  \begin{lstlisting}
    colores = c('#b2df8a','#33a02c',
                '#fdbf6f','#ff7f00',
                '#fb9a99','#e31a1c',
                '#a6cee3','#1f78b4')
    plot(reclas.2016, col=colores, zlim=c(1,8))
  \end{lstlisting}
  \begin{figure}[h!]
    \centering
    \includegraphics[width=0.7\textwidth]{kmeans-colores.png}
    \caption{Imagen clasificada por el m\'etodo kmeans con 5 clases espectraales.}
    \label{fig:kmean5}
  \end{figure}
\end{exa}

\begin{exa}
  Hagamos un an\'alisis espectral de las im\'agenes clasificadas antes y despu\'es de la fusi\'on.  Utilizaremos a la imagen clasificada como mascara para asignar los colores  al scatterplot.

  Agregamos la imagen clasificada a las bandas de la imagen en reflectancia y luego hacemos los scatterplot seg\'un la clase espectral o de informaci\'on como

  \begin{lstlisting}
    stack.2016 <- stack(ref.2016, reclas.2016)
    xyplot(nir+swir1~red, groups=MC_ID, data=stack.2016)
  \end{lstlisting}

  Vemos que las clases espectrales pueden unirse en clases de informaci\'on y que algunas clases de informaci\'on no se separan en distintas clases espectrales (figuras \ref{fig:esk}).

  \begin{figure}[h!]
    \centering
    \includegraphics[width=0.7\textwidth]{kmeans-es.png}
    \caption{Espacio espectral clasificado por kmeans.}
    \label{fig:esk}
  \end{figure}

\end{exa}

\begin{exa}
  Repitamos este an\'alisis pero con 100 clases espectrales. El algor\'itmo es el mismo pero ahora vamos a usar el parametro \texttt{nClasses} igual a 100.

  \begin{lstlisting}
  set.seed(42)
  kmeans.2016b <- unsuperClass(ref.2016, nClasses = 100, nStarts = 100,
                              nSamples = 1000)
  writeRaster(kmeans.2016b$map, "raster_data/processed/kmeans2016b",
              datatype="INT1U", overwrite=TRUE)
  clasesb.2016 <- read.delim("aux_data/class100.txt")
  reclasb.2016 <- subs(kmeans.2016b$map, clasesb.2016)
  plot(reclasb.2016, col=colores, zlim=c(1,8))
  \end{lstlisting}

  Comparamos las clasificaciones obtenidas con 5 y 100 classes espectrales (figura \ref{fig:5v100})

  \begin{figure}[h!]
    \centering
    \includegraphics[width=0.7\textwidth]{5v100.png}
    \caption{Comparaci\'on entre la clasificaci\'on con 5 clases y 100 clases espectrales.}
    \label{fig:5v100}
  \end{figure}

  Comparamos nuevamente los espacios de fase entre la clasificaciones obtenidas a partir de 5 y 100 clases espectrales

  \begin{lstlisting}
    stackb.2016 <- stack(ref.2016, reclasb.2016)
    xyplot(nir+swir1~red, groups=MC_ID, data=stack.2016)
    xyplot(nir+swir1~red, groups=MC_ID, data=stackb.2016)
  \end{lstlisting}

\end{exa}

\begin{act}
  Repita la clasificaci\'on para la imagen landsat 7 del año 2000.
\end{act}


\section{Informaci\'on espacial y temporal}

Con el fin de mejorar la clasificaci\'on podemos incorporar informaci\'on espacial y temporal a la informaci\'on radiom\'etrica. De esta forma, al momento de clasificar los p\'ixeles, ya no estaremos trabajando en un espacio espectral, sino en uno m\'as amplio. Veamos como hacerlo.

\begin{exa}
  Para incorporar informaci\'on espacial sobre el contexto del p\'ixel, podemos
  hacerlo antes o despu\'es de la clasificaci\'on. Aqu\'i nos concentraremos en el primer caso

  Calculamos primero la variabilidad local de brillo para la banda pancrom\'atica de la imagen Landsat 8 como

  \begin{lstlisting}
    pan.2016 <- raster("raster_data/LC82240782016304LGN00/LC82240782016304LGN00_B8.TIF")
    window <- matrix(1,nrow=5, ncol=5)
    sd.2016<-focal(pan.2016,w=window,fun=sd)
    plot(log(sd.2016,base = 10), zlim=c(1,4))
  \end{lstlisting}

  Vemos que las zonas con mayor presencia de urbanizacion, presentan mayor variabilidad que aquellas con menor (figura \ref{fig:pansd}).

  \begin{figure}[h!]
    \centering
    \includegraphics[width=0.7\textwidth]{pan-sd.png}
    \caption{Desv\'io standar para una ventana de 5 p\'ixeles por 5 p\'ixeles en escala logar\'itmica.}
    \label{fig:pansd}
  \end{figure}

  Una vez obtenida la banda de desvio standar, podemos agregarla a las demas, luego de remuestrearla haciendo

  \begin{lstlisting}
    sd.2016 <- aggregate(sd.2016, fact=2, fun=mean)
    stack(ref.2016, sd.2016)
    pca.2016 <- rasterPCA(stack(ref.2016, sd.2016), spca = TRUE)
  \end{lstlisting}

  \begin{Verbatim}[fontsize=\small]
  Importance of components:
                            Comp.1    Comp.2    Comp.3     Comp.4      Comp.5 ...
  Standard deviation     2.1592402 1.1821570 0.8392878 0.40888073 0.196591748 ...
  Proportion of Variance 0.6660455 0.1996422 0.1006292 0.02388335 0.005521188 ...
  Cumulative Proportion  0.6660455 0.8656876 0.9663168 0.99020013 0.995721318 ...
  \end{Verbatim}

\end{exa}

\begin{act}
  Repita el ejemplo para la imagen landsat 7 del año 2000. Clasifique por el metodo de kmeans las imagenes de los años 2000 y 2016.
\end{act}

\begin{exa}
  Para incorporar informaci\'on del contexto temporal, agregaremos el promedio y el desv\'io estandar del NDVI a lo largo del 2016. Primero cargamos las im\'agenes MODIS del 2016

  \begin{lstlisting}
    list.2016 <- list.files("raster_data/MOD13Q1/NDVI/", pattern = "MOD13Q1.A2016+.*tif", full.names = TRUE)
    ndvi.2016 <- stack(list.2016)/1e4
    ndvi.2016 <- approxNA(ndvi.2016)
    sd.2016 <- calc(ndvi.2016, fun = sd)
    me.2016 <- calc(ndvi.2016, fun = mean)
    plot(stack(me.2016,sd.2016))
  \end{lstlisting}

  Una vez hecho esto, resampleamos el promedio y el desv\'io de la serie temporal y lo unimos a las im\'agenes en reflectancia, sobre las cuales calcularemos la transformada por componentes principales.

  \begin{lstlisting}
    sd.2016 <- resample(sd.2016, ref.2016, method="ngb")
    me.2016 <- resample(me.2016, ref.2016, method="ngb")
    pca.2016 <- rasterPCA(stack(ref.2016, me.2016, sd.2016), spca = TRUE)
    summary(pca.2016$model)
  \end{lstlisting}
\end{exa}

\begin{act}
  Repita el ejemplo para la imagen Landsat 7 del año 2000. Clasifique por el metodo de kmeans las imagenes de los años 2000 y 2016.
\end{act}
