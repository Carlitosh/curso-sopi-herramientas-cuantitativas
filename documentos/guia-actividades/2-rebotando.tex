
En esta segunda actividad pr\'actica nos centraremos en la correcci\'on radiom\'etrica
de im\'agenes satelitales. Son nuestros objetivos:

\begin{itemize}
    \item Abrir una imagen satelital desde el metadato.
    \item Convertir los valores de la imagen a reflectancia a tope de la
        atm\'osfera.
    \item Corregir la imagen satelital utilizando los m\'etodos de \emph{dos} y
        \emph{cost}
    \item Corregir la imagen satelital utilizando el \emph{6S} en su versi\'on web.
\end{itemize}

\section{C\'alculo de reflectancia a tope de la atm\'osfera}

Para poder convertir una imagen a reflectancia a tope de la atm\'osfera vamos a
necesitar la informaci\'on adicional que hallaremos en su metadato.
Para abrirla desde el metadato utilizaremos las funciones
disponibles en la libreria \texttt{RStoolbox}.

\begin{exa}
    Abramos  la imagen Landsat 7
    del año 2000 desde el metadato y la mostraremos en combinaci\'on color real
    y analicemos sus propiedades.
    \begin{lstlisting}
    meta.2000 <- readMeta("raster_data/LE72240782000188EDC00/LE72240782000188EDC00_MTL.txt")
    \end{lstlisting}
    Podemos mostrar las distintas variables incluidas en el objeto usando el
    signo \$ y su nombre. Por ejemplo, \verb|meta.2000$SOLAR_PARAMETERS|
    da como resultado
    \begin{Verbatim}[fontsize=\small]
     azimuth elevation  distance
    37.38251  31.14409   1.01670
    \end{Verbatim}
    Usando el metadato podemos cargar la imagen completa con el comando
    \texttt{stackMeta}. Eliminamos, en este caso, las bandas 6 y 7 por
    ser t\'ermicas.
    \begin{lstlisting}
    dn.2000 <- stackMeta(meta.2000)
    dn.2000 <- dn.2000[[-6:-7,]]
    dn.2000
    \end{lstlisting}
    El resultado es un objeto \emph{raster stack} como el que sigue
    \begin{Verbatim}[fontsize=\small]
    class       : RasterStack
    dimensions  : 2412, 1834, 4423608, 6  (nrow, ncol, ncell, nlayers)
    resolution  : 30.00402, 30.00265  (x, y)
    extent      : 731118.6, 786146, 7101531, 7173897  (xmin, xmax, ymin, ymax)
    coord. ref. : +proj=utm +zone=21 +south +datum=WGS84 +units=m +no_defs
                  +ellps=WGS84 +towgs84=0,0,0
    names       : B1_dn, B2_dn, B3_dn, B4_dn, B5_dn, B7_dn
    min values  :     0,     0,     0,     0,     0,     0
    max values  :   255,   255,   255,   255,   255,   255
    \end{Verbatim}
    Mostramos la imagen en cobinaci\'on color real con \texttt{plotRGB(dn.2000, r=3, g=2, b=1, stretch="lin")}
    (Figura \ref{fig:dn-l7-rgb}).
     \begin{figure}[h!]
     \begin{center}
         \includegraphics[scale=0.6]{dn-l7-rgb}
     \end{center}
     \caption{Imagen en combinaci\'on de bandas color real de la zona de inter\'es.}
     \label{fig:dn-l7-rgb}
     \end{figure}
\end{exa}

De esta forma tenemos el archivo en n\'umero digital con todos sus metadatos
para convertirlo a reflectancia y realizar las distintas correcciones.

Existen dos formas de convertirla a reflectancia a tope de la atm\'osfera: a mano,
utilizando las herramientas algebraicas de R o con la funci\'on
especifica de \texttt{RStoolbox}.

\begin{exa}
    C\'alculo de reflectancia a tope de la atm\'osfera
    utilizando el metadato paso por paso

    \begin{lstlisting}
    dn2ref.2000 <- meta.2000$CALREF[1:6,]
    elev.2000 <- pi*meta.2000$SOLAR_PARAMETERS['elevation']/180
    \end{lstlisting}
    extraemos primero del metadato los par\'ametros de calibraci\'on en
    reflectancia y el \'angulo de elevaci\'on solar. Convertimos luego la imagen
    a reflectancia y la dividimos por el seno del \'angulo solar.
    Luego cambiamos los nombres de las bandas

    \begin{lstlisting}
    toam.2000 <- (dn.2000*dn2ref.2000$gain+dn2ref.2000$offset)/sin(elev.2000)
    names(toam.2000) <- c("blue","green","red","nir","swir1","swir2")
    \end{lstlisting}

    Otra forma de realizarlo es utilizando la funci\'on
    \texttt{radCor}. En este caso, debemos tener la imagen en DN, el metadato y
    cual es la cantidad que queremos calcular.

    \begin{lstlisting}
    toa.2000 <- radCor(dn.2000, metaData = meta.2000, method = "apref")
    \end{lstlisting}

    Podemos comparar los resultados de ambos m\'etodos inspeccionando los objetos
    \texttt{toam.2000} y \texttt{toa.2000}.
    \begin{Verbatim}[fontsize=\small]
    class       : RasterBrick
    dimensions  : 2412, 1834, 4423608, 6  (nrow, ncol, ncell, nlayers)
    resolution  : 30.00402, 30.00265  (x, y)
    extent      : 731118.6, 786146, 7101531, 7173897  (xmin, xmax, ymin, ymax)
    coord. ref. : +proj=utm +zone=21 +south +datum=WGS84 +units=m +no_defs
                  +ellps=WGS84 +towgs84=0,0,0
    data source : in memory
    names       :        blue,       green,         red,         nir,    swir1,       ...
    min values  : -0.01976113, -0.02181530, -0.02029439,  0.01934678,    -0.02781926, ...
    max values  :   0.6106812,   0.5609009,   0.6079443,   0.8696885,    0.8640919,   ...
    \end{Verbatim}

    y

    \begin{Verbatim}[fontsize=\small]
    class       : RasterStack
    dimensions  : 2412, 1834, 4423608, 6  (nrow, ncol, ncell, nlayers)
    resolution  : 30.00402, 30.00265  (x, y)
    extent      : 731118.6, 786146, 7101531, 7173897  (xmin, xmax, ymin, ymax)
    coord. ref. : +proj=utm +zone=21 +south +datum=WGS84 +units=m +no_defs
                  +ellps=WGS84 +towgs84=0,0,0
    names       :     B1_tre,     B2_tre,     B3_tre,     B4_tre,     B5_tre, ...
    min values  : 0.00000000, 0.00000000, 0.00000000, 0.01934678, 0.00000000, ...
    max values  :  0.6106812,  0.5609009,  0.6079443,  0.8696885,  0.8640919, ...
    \end{Verbatim}

    \end{exa}

\begin{act}
    Inspeccione la reflectancia a tope de la atm\'osfera para todas las bandas.
    Para esto realice los histogramas, graficos de dispersi\'on, calcule la media,
    el desv\'io standar y cualquier otra medida estad\'istica que desee.
\end{act}
\section{C\'alculo de reflectancia corregida atmosf\'ericamente por m\'etodos
            estad\'isticos}

La funci\'on \texttt{radCor} dispone de un par\'ametro para hacer distintos
tipos de correcciones atmosf\'ericas. Ya utilizamos \emph{apref}, que nos permiti\'o
calcular la reflectancia a tope de la atm\'osfera, veamos ahora como aplicar el m\'etodo
de substracci\'on de cuerpo obscuro.

\begin{exa}
    Apliquemos el metodo de \emph{simple dos} para corregir la imagen. En este caso
    solamente restaremos el m\'inimo en cada banda a la imagen para las bandas
    donde existe haze, es decir en la zona del visible y del infrarrojo cercano.

    Estimamos el haze primero y luego corregimos la imagen haciendo
    \begin{lstlisting}
    haze.2000 <- estimateHaze(dn.2000,darkProp = 0.01, hazeBands = 1:4, plot=TRUE)
    sdos.2000 <- radCor(dn.2000, metaData = meta.2000,
                 hazeValues = haze.2000,
                 hazeBands = c("B1_dn","B2_dn","B3_dn","B4_dn"),
                 method="sdos")
    \end{lstlisting}
    en este caso los valores de haze estimados son
    \begin{Verbatim}[fontsize=\small]
    B1_dn B2_dn B3_dn B4_dn
       41    27    20    15
    \end{Verbatim}
    Para hacer un an\'alisis de lo que pasa, vamos a graficar los
    histogramas de cada banda para la imagen en reflectancia TOA y corregida por
    el m\'etodo \emph{simple dos}. Usaremos el paquete \texttt{rasterVis}
    \begin{lstlisting}
    B1 <- densityplot(~B1_tre+B1_sre, data=toa.boa, xlab="Reflectancia",
                      ylab="", main="Banda azul", plot.points=FALSE, xlim=c(0,0.3),
                      key=simpleKey(text=c("Tope de la atmosfera",
                                           "Correccion Simple DOS"),
                                           lines=TRUE, points=FALSE))
    B2 <- densityplot(~B2_tre+B2_sre, data=toa.boa, xlab="Reflectancia",
                      ylab="", main="Banda verde", plot.points=FALSE, xlim=c(0,0.3),
                      key=simpleKey(text=c("Tope de la atmosfera",
                                           "Correccion Simple DOS"),
                                           lines=TRUE, points=FALSE))
    B3 <- densityplot(~B3_tre+B3_sre, data=toa.boa, xlab="Reflectancia",
                      ylab="", main="Banda roja", plot.points=FALSE, xlim=c(0,0.3),
                      key=simpleKey(text=c("Tope de la atmosfera",
                                           "Correccion Simple DOS"),
                                           lines=TRUE, points=FALSE))
    B4 <- densityplot(~B4_tre+B4_sre, data=toa.boa, xlab="Reflectancia",
                      ylab="", main="Banda nir", plot.points=FALSE, xlim=c(0,0.3),
                      key=simpleKey(text=c("Tope de la atmosfera",
                                           "Correccion Simple DOS"),
                                           lines=TRUE, points=FALSE))
     print(B1,split = c(1, 1, 2, 2),more=TRUE)
     print(B2,split = c(2, 1, 2, 2),more=TRUE)
     print(B3,split = c(1, 2, 2, 2),more=TRUE)
     print(B4,split = c(2, 2, 2, 2),more=FALSE)
    \end{lstlisting}
    las primeras 4 funciones crean los histogramas para cada banda
    corregida, mientras que las \'ultimas 4 l\'ineas los imprimen en una grilla (Figura \ref{fig:simpledos}).
    \begin{figure}[h!]
    \begin{center}
        \includegraphics[scale=0.4]{simpledos.png}
    \end{center}
    \caption{Graficos de densidad para las distintas bandas donde se
        muestra el nivel de correcci\'on en cada una.}
    \label{fig:simpledos}
    \end{figure}
Notamos en este caso que la correcci\'on se vuelve menos importante a medida
que crece la longitud de onda. Adem\'as, la correcci\'on solo cambia la
posici\'on de la distribuci\'on y no su forma.

\end{exa}
\begin{act}
    Analice los valores de haze obtenidos por la funci\'on stimate haze y graf\'iquelos
    como funci\'on de la longitud de onda en escala logar\'itmica. ¿Qu\'e observa?
\end{act}

\begin{act}
    Utilice el m\'etodo \emph{costz} para corregir la imagen a reflectancia a tope
    de la superficie. Puede ayudarse con el comando \texttt{?radCor}
\end{act}

\begin{act}
    Guarde el archivo raster generado por cada uno de los m\'etodos de
    correcci\'on. Abralos en QGIS y comp\'arelos visualmente. Obtenga firmas
    espectrales con los distintos m\'etodos de correcci\'on.
\end{act}


\section{6S}
\label{sub:corr:6S}

Veamos ahora como operar con el 6S para obtener una estimaci\'on de los par\'ametros
atmosf\'ericos. Para ello utilizaremos la versi\'on web del 6S que se encuentra
disponible en \url{http://6s.ltdri.org/pages/run6SV.html}.

Ingresaremos a la p\'agina y haremos click en el bot\'on
\menu{Submit query}. Iremos luego configurando paso a paso nuestro modelo de la
atm\'osfera, haciendo siempre click en el bot\'on \menu{submit query} para
ir al paso siguiente.

Los parametros para nuestro modelo son

\begin{enumerate}
    \item Geometrical conditions
        \begin{itemize}
            \item TM (Landsat)
            \item Month: 4, Day:13, GTM decimal hour: 13.60, Longitude:
                -63.8606, Latitude: -24.9937.
        \end{itemize}
    \item Atmospheric Model
        \begin{itemize}
            \item Select Atmospheric Profile: Mid latitude summer
            \item Select aerosol model: Continental Model
            \item Visibility: 60
        \end{itemize}
    \item Target \& sensor altitude
        \begin{itemize}
            \item Select targe altitude: sea level
            \item Select sensor altitude: satellite level
        \end{itemize}
    \item Spectral conditions
        \begin{itemize}
            \item Select spectral conditions: choose band
            \item Select band: 1st band of tm (landat 5)
        \end{itemize}
    \item Ground reflectance
        \begin{itemize}
            \item Ground reflectance type: homogeneous surface
            \item Directional effect: no directional effect
            \item Specify surface reflectance: input constant value of ro
            \item input constant value for ro: 0
        \end{itemize}
    \item Signal
        \begin{itemize}
            \item Atmospheric correction mode: no atmospheric correction
        \end{itemize}
\end{enumerate}

Todos estos valores los encontramos en el metadato de la imagen.

En \menu{7.Results} podemos ver el resultado haciendo click en \emph{Output
file}. Extraemos los valores de
 \begin{itemize}
     \item \texttt{global gas. trans. - total}
     \item \texttt{total sca. trans. - total}
     \item \texttt{spherical albedo - total}
     \item \texttt{reflectance I - total}
 \end{itemize}

Una vez ejecutado el proceso puede usarse el siguiente c\'odigo para corregir
todas las bandas utilizando R.

\begin{lstlisting}
    a <- c(0.98,0.90,...) # Global gas transmitance
    b <- c(0.81,0.90,...) # Total scatering transmitance
    g <- c(0.15,0.10,...) # Spherical albedo
    r <- c(0.08,0.05,...) # Reflectance I
    sss.2000 <- (toa.2000/(a*b)-r/b)/(1+g*(toa.2000/(a*b)-r/b))
\end{lstlisting}


\begin{act}
    Realice una extracci\'on de firmas espectrales para distintas coberturas de
    cada uno de los archivos raster obtenidos y graf\'iquelos juntos.
    Compare los resultados con la firma espectral realizada a partir de la imagen corregida
    por el USGS.
\end{act}

\begin{act}
    Haga un gr\'afico de densidades que muestre los distintos m\'etodos de
    correcci\'on atmosf\'ericos para cada banda.
\end{act}

\begin{act}
    Para cada banda calcule la diferencia promedio entre las imagenes en
    reflectancia a tope de la atm\'osfera, las distintas correcciones
    y la imagen en reflectancia entregada por el USGS.
\end{act}
