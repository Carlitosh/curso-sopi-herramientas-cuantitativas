Veamos ahora algunas tecnicas de que permiten mejorar las clasificaciones y nos
ayudaran a validar y extraer datos de las imagenes clasificadas.

Comenzamos viendo como aplicar un filtro a una imagen. Comenzamos cargando las
librerias utilizar.

\begin{lstlisting}
    library(RStoolbox)
    library(rgdal)
    library(raster)
    library(rasterVis)
\end{lstlisting}

Para aplicar un filtro a una imagen monobanda, debemos primero definir cual es
la ventana en la que trabajaremos y luego cual es la operacion que desamos
realizr en dicha ventana.

\begin{lstlisting}
    window <- matrix(1,nrow=3, ncol=3)
    clasification.3x3<-focal(clasification.2016,w=window,fun=modal)
\end{lstlisting}

En el caso de un filtro por moda, estaremos dejando el mayor que mas veces
aparezca entre los que rodean al pixel.

\begin{act}
    Aplique filtros de 5x5 y 7x7 para filtrar la imagen. Que problemas
    desaparecen? que dificultad introducen.
\end{act}

\begin{act}
    Aplique el filtro de 3x3 a la imagen correspondiente al año 2000.
\end{act}

\begin{act}
    Utilice la funcion raster sieve de qgis para realizar un filtrado espacial.
    Que diferencias encuentra.
\end{act}

Una vez filtrada la imagen podemos obtener de la misma la matriz de confusion.
Para esto debemos cargar el poligono de validacion y la calculamos con la
funcion \texttt{validateMap}.

\begin{lstlisting}
    valid.2016 <- readOGR(dsn="vector_data/", layer="validacion")
    val.unsup.2016 <- validateMap(sup.2016$map,valData = valid, 
                              responseCol = "MC_ID")
\end{lstlisting}

\begin{act}
    Construya la matriz de confusion y obtenga la presicion global para todas
    las clasificaciones. Que algoritmo funciona mejor con la imagen?
\end{act}

Otra forma de construir incorporar contexto espacial a las clasificaciones es
contruir una capa de textura. Veamos como construir una banda de textura
utilizando la banda pancromatica de Landsat degradada a 30m.

\begin{exa}
    Comenzamos cargando la imagen pancromatica
    \begin{lstlisting}
        pan.2016 <- raster("raster_data/LE72240782000188EDC00/LE72240782000188EDC00_B8.TIF")
    \end{lstlisting}
    Una vez cargada podemos visualizarla como 
    \begin{lstlisting}
        plot(pan.2016)
    \end{lstlisting}
    calculamos ahora el estimador mas sencillo de la textura mediante el calculo
    del desvio standar en una ventana de 3x3
    \begin{lstlisting}
        windows <- matrix(1,nrow=3,ncol=3)
        sd.2016 <- focal(pan.2016,window,fun=sd)
    \end{lstlisting}
    desvio que pondemos graficar como
    \begin{lstlisting}
        plot(sd.2016)
    \end{lstlisting}
    finalmente, degradamos el mapa obtenido a 30m para poder utilizarlo con la
    imagen multiespectral.
    \begin{lstlisting}
        sd.agregate.2016 <- aggregate(sd.2016,fact=2,fun=mean)
    \end{lstlisting}
    finalmente usamos la funcion \texttt{stact} para juntar todas las bandas y
    proceder a la clasificacion.
    \begin{lstlisting}
        context.2016 <- stack{ref.2016,sd.agregate.2016}
        class.2016 <- supClas{}
    \end{lstlisting}
\end{exa}
