Veamos ahora como realizar operaciones sencillas entre las bandas de una imagen.
Usaremos en esta practica los siguientes paquetes

\begin{lstlisting}
    library(raster)
    library(RStoolbox)
    library(RColorBrewer)
    library(rgdal)
    library(ggplot2)
    libyrary(GGally)
\end{lstlisting}

Comenzamos primer cargando la imagen desde el metadato y convirtiendola a
reflectancia como hicimos en la clase anterior

\begin{lstlisting}
    xml.2016 <- readMeta("raster_data/LC.../LC....xml")
    ref.2016 <- stackMeta(xml.2016, quantity = "sre")
    scaleF <- getMeta(ref.2015,xml.2016, what = "SCALE_FACTOR")
    ref.2016 <- ref.2016 * scaleF
    ref.2016 <- ref.2016[[-1,]]
    names(ref.2016) <- c("blue","green","red","nir","swir1","swir2")
\end{lstlisting}

una vez cargada la imagen podemos realizar operaciones entre las bandas llamando
a cada una por separado. Veamos como ejemplo el calculo de NDVI\@.

\begin{exa}
    Calculo de NDVI a mano y grafico del mismo
    \begin{lstlisting}
        ndvi.2016 <- (ref.2016$nir-ref.2016$red)/(nir.2016$nir+ref.2016$ref)
        cols = colorRampPalette(brewer.pal(9,"YlGn"))(16)
        plot(ndvi.2016, col=cols, zlim = c(0,1))
    \end{lstlisting}
    obteniendo una imagen como la que se ve debajo.
\end{exa}

El paquete \texttt{RStoolbox} tiene varias herramientas que nos ayudan a
calcular los indices espectrales. Veamos por ejemplo como calcular el NDVI y el
EVI utilizando dicho paquete

\begin{exa}
    Para calcular los indices mediante la funcion spectralIndices debemos
    especificar con que raster trabajamos y que bandas corresponden a cada
    longitud de onda
    \begin{lstlisting}
    indices.2016 <- spectralIndices(ref.2016, 
                                    blues="blue", red="red", nir="nir", 
                                    indices=c("NDVI","EVI"))
    plot(indices.2016,col=cols, zlim=c(0,1))
    \end{lstlisting}
    obtenemos una imagen como se muestra debajo.
\end{exa}

\begin{act}
    Calcule el NDVI para el año 2000 utilizando la imagen landsat 7.
\end{act}

\begin{act}
    Calcule y grafique todos los indices posibles que involucren a las bandas
    roja y nir de landsat 8. 
\end{act}


\begin{exa}
    Veamos ahora como calcular el tSAVI utilizando la linea de suelo obtenida a
    partir de la imagen. Para esto necesitaremos enmascarar las zonas con
    cobertura de agua y nubes. Veamos primer como hacer esto.
    \begin{lstlisting}
        mask.2016 <- raster("raster/.../...cfmask.tif")
        masked.2016 <- mask(ref.2016, mask=mask.2016, inverse=TRUE,
                            maskvalue=0, updatevalue=255)
        masked.2016[masked.2016<=0] <- 255
    \end{lstlisting}
    de esta forma enmascaramos todos los valores con nubes, agua y donde la
    reflectancia obtenida es cero.
    Calculamos ahora la linea de suelo y la mostramos en un scatterplot
    \begin{lstlisting}
        bsl.2016 <- BSL(as.matrix(masked.2016$red), as.matrix(masked.2016$nir),
                        method="quantile", ulimimt=0.99, llimit=0.001)
        plot(ref.2016$red, ref.2016$nir)
        abline(bsl.2016$BSL,col="red")
    \end{lstlisting}
\end{exa}

\begin{act}
    Calcule el tSAVI utilizando la linea de suelo obtenida arriba.
\end{act}

\begin{act}
    Vuelva a obtener la linea de suelo sin enmascarar la imagen y dibujo el
    scatterplot con la misma y la anterior. Que problema encuentra.
\end{act}

Finalmente, veamos como se puede obtener datos biofisicos a partir de los
indices de vegetacion calculados. De esta forma podremos generar mapas de
porcentaje de cobertura, productividad, etc.

\begin{act}
    Cargue la capa vectorial del muestreo de variables biofisicas
    \texttt{muestreo.shp} y haga una extraccion de los valores de NDVI
    correspondientes a dichos puntos. Guarde estos valores en un dataframe
    llamado \texttt{muestreo}.
\end{act}

\begin{exa}
    Veamos como ajustar con R un modelo lineal a nuestro modelo. Para esto
    comencemos haciendo un analisis visual con la funcion \texttt{ggpairs}.
    \begin{lstlisting}
        ggpairs(muestreo,diag=list(continuous="barDiag"))
    \end{lstlisting}
    Obtendremos un grafico que presenta los scatterplots entre las bandas, su
    correlacion e histogramas.
    Veamos en el mismo que la superficie cubierta por vegetacion varia
    linealmente con el NDVI\@. Por lo tanto utilizaremos estos para hacer un
    ajuste de nuestro modelo.
    \begin{lstlisting}
        lm.2016 <- lm(fcover~ndvi, data=muestreo)
        plot(muestreo$ndvi, mustreo$fcover)
        abline(lm.2016, col="red")
        summary(lm.2016)
    \end{lstlisting}
    de esta forma veremos los parametros de nuestro ajuste, y graficaremos al
    mismo en un scatterplot.

    Para aplicar el modelo a nuestro raster hacemos
    \begin{lstlisting}
        fcover.2016 <- predict(ndvi.2016,lm.2016)
        plot(fcover.2016)
    \end{lstlisting}
    Obteniendo el mapa de abajo.
\end{exa}

\begin{act}
    Genere los modelos de lai, fapar y fcover para el año 2016 y con los mismos
    realice mapas de dichas variables.
\end{act}

\begin{act}
    Utilizando los modelos obtenidos para 2016 aplique los mismos para obtener
    los mapas de lai, fapar y fcover del año 2000. Que suposicion esta
    haciendo?
\end{act}

\begin{act}
    * Utilizando la funcion spectralIndices y ggpairs, analice si hay otro indice
    que ajuste que correlacione mejor con las alguna de las varibles biofisicas
    medidas a campo.
\end{act}

