\documentclass[a4paper]{book}

% Preambulo por defecto
% Paquetes de la ams
\usepackage{amsmath,amsthm,amssymb,amsfonts}
% Posibilidad de mover la pagina
\usepackage[a4paper]{geometry}
% Saco la indentacion en todos los parrafos.
%\usepackage{parskip}
% Codificacion UTF-8
\usepackage[utf8]{inputenc}
% Tablas e imagenes en espaniol
\usepackage[spanish,es-tabla]{babel}
% Mejores graficos
\usepackage{graphicx}
% tablas mas lindas
\usepackage{booktabs}
% Posibilidad de tocar los encabezados
\usepackage{fancyhdr}
%\pagestyle{fancy}
% Posibilidad de meter subfiguras
\usepackage[font=footnotesize, labelfont=it]{subcaption}
% Links a urls
\usepackage{url}
% Linkear referencias en pdfs
\usepackage{hyperref}
% Texto mas lindo para los pie de figura
\usepackage[margin=10pt,font=small,labelfont=bf, labelsep=endash]{caption}
% Mejores autores
\usepackage[affil-it]{authblk}
% Compatibilidad con PDF/A
\usepackage{xmpincl}
% Hoja a4 mas ancha
\usepackage{a4wide}
% Citas
\usepackage[backend=biber,style=ieee]{biblatex}
\addbibresource{biblio.bib}
% Cambio and por y
\renewcommand\Authand{y }
\renewcommand\Authands{, y }

% Codigo
\usepackage{listings}

% Coloreo los links
\usepackage[usenames,dvipsnames]{xcolor}
\hypersetup{colorlinks,
     linkcolor={red!50!black},
     citecolor={blue!50!black},
     urlcolor={blue!80!black} }
% Graficos con tikz
\usepackage{tikz}

% Dir tree
\usepackage{dirtree}

% Configuracion de listings para R
\lstset{%
  language=R,                     % the language of the code
  basicstyle=\footnotesize,       % the size of the fonts that are used for the code
  numbers=left,                   % where to put the line-numbers
  numberstyle=\tiny\color{gray},  % the style that is used for the line-numbers
  stepnumber=1,                   % the step between two line-numbers. If it's 1, each line
                                  % will be numbered
  numbersep=5pt,                  % how far the line-numbers are from the code
  backgroundcolor=\color{white},  % choose the background color. You must add \usepackage{color}
  showspaces=false,               % show spaces adding particular underscores
  showstringspaces=false,         % underline spaces within strings
  showtabs=false,                 % show tabs within strings adding particular underscores
  %frame=single,                   % adds a frame around the code
  rulecolor=\color{black},        % if not set, the frame-color may be changed on line-breaks within not-black text (e.g. commens (green here))
  tabsize=2,                      % sets default tabsize to 2 spaces
  captionpos=b,                   % sets the caption-position to bottom
  breaklines=true,                % sets automatic line breaking
  breakatwhitespace=false,        % sets if automatic breaks should only happen at whitespace
  title=\lstname,                 % show the filename of files included with \lstinputlisting;
                                  % also try caption instead of title
  keywordstyle=\color{blue},      % keyword style
  commentstyle=\color{OliveGreen},   % comment style
  stringstyle=\color{Plum}       % string literal style
}

\definecolor{A11}{HTML}{B2DF8A}
\definecolor{A12}{HTML}{33A02C}
\definecolor{A23}{HTML}{FDBF6F}
\definecolor{A24}{HTML}{FF7F00}
\definecolor{B15}{HTML}{FB9A99}
\definecolor{B16}{HTML}{E31A1C}
\definecolor{B27}{HTML}{A6CEE3}
\definecolor{B28}{HTML}{1F78B4}

\usepackage{fancyvrb}
% Ejemplos, observaciones y teorema
\theoremstyle{definition}
\newtheorem{exa}{Ejemplo}[section]
\newtheorem{obs}{Observación}[section]
\newtheorem{act}{Actividad}[section]


% Referencias menu
\newtoggle{FirstOne}%
\newcommand*{\menu}[1]{%
\toggletrue{FirstOne}%
\foreach \x in {#1} {%
\iftoggle{FirstOne}{}{${}\rightarrow{}$}%
\emph{\x}%
\global\togglefalse{FirstOne}%
}%
}%

% Referencias archivo
\newtoggle{SecondOne}%
\newcommand*{\file}[1]{%
\toggletrue{SecondOne}%
\foreach \x in {#1} {%
\iftoggle{SecondOne}{}{${}/{}$}%
\texttt{\x}%
\global\togglefalse{SecondOne}%
}%
}%




\title{{Nivel 2: Herramientas de teledetecci\'on cuantitativa}\\
\emph{Gu\'{\i}a de actividades: Uso del suelo en el departamento de Iguaz\'u,
provincia de Misiones}}
\date{2017-1}
\author{Francisco Nemiña\thanks{\texttt{fnemina@conae.gov.ar}}}
\affil{Unidad de Educacion y Formacion Masiva \\ Comisi\'on Nacional de Actividades Espaciales}

\graphicspath{{./figs/}}



\begin{document}
\frontmatter
\maketitle

\tableofcontents
\mainmatter
\chapter{Introducción}

\label{sec:intro}

La utilización de imágenes satelitales permite analizar grandes extensiones del
territorio, contando con un registro histórico con el cual realizar
comparaciones.

En la provincia de Misiones, el departamento de Iguaz\'u es lindante a Brasil y
Paraguay siendo parte de la zona conocida como triple frontera
perteneciente a la ecorregi\'on conocida como \emph{selva paranaense} (Figura \ref{parque}). Dentro
del mismo podemos encontrar la Represa de Urugua-\'{\i} y el Parque Nacional
Iguaz\'u. El departamento tiene un area de $2.736 km^2$ y una poblaci\'on de
$82.227$ seg\'un el \'u;timo senso del INDEC.

\begin{figure}[h!]
  \centering
  \includegraphics[width=0.8\textwidth]{triple.png}
  \caption{Departamento de Iguaz\'u, en rosa, y parque natural Iguaz\'u, en celeste.}
  \label{parque}
\end{figure}

Tomaremos entonces al departamento como \'area de estudio durante este curso
con el objetivo de obtener un mapa de uso y cobertura dentro del mismo que nos
permita estimar y validar  las \'areas correspondients a los mismos.

Utilizaremos para esto imagenes satelitales de los satelites Landsat 8, Landsat 7 y
el producto de MOD13Q1 obtenido de los satelites TERRA y AQUA obtenidas durante
el periodo 2000-2016.

\section{Organizacion del curso}

El curso se divide en dos partes. En la primera trabajaremos con la compresion
del espacio espectral y el uso de las imagenes satelitales para extraer valores
continuos de las variables biofisicas.

En el capitulo \ref{viaje}, \nameref{viaje}, estudiaremos la firma espectral de
distintas coberturas, veremos como se relacionan con las propiedades biofisicas
de ellas y por ultimo introduciremos el concepto de espacio espectral como el
lugar natural donde realizar el analisis en teledeteccion.

En el capitulo \ref{rebotando}, \nameref{rebotando}, estudiaremos distintas
formas de obtener la reflectancia de las coberturas a partir de los datos obtenidos
por un satelite. En el estudiaremos metodos de correccion atmosferica basados
en propiedades estadisticas de las imagenes y en el modelado de la atmosfera.

En el capitulo \ref{abaco}, \nameref{abaco}, veremos como a partir de los valores
de reflectancia para una cobertura y operaciones matematicas entre los mismos,
podemos obtener los valores de variables biofisicas continuass como pueden ser
el contenido de clorofila o humedad.

En la segunda parte del curso vamos a trabajar mas en detalle con el espacio
espectral y vamos a usar sus propiedades para extraer informacion categorica de
las imagenes.

En el capitulo \ref{rotaciones}, \nameref{rotaciones}, veremos como utilizar
herramientas geometricas en el espacio de reflectancias para resaltar distintas
propiedades de las imagenes y poner en evidencia cuales son las zonas del espectro
que mas informacion aportan sobre nuesta zona de interes. Comenzaremos en este
capitulo tambien a analizar el contexto temporal para nuestras imagenes.

En el capitulo \ref{otrolado}, \nameref{otrolado}, veremos como utilizar herramientas
que no requieren de conocimiento previo del area de estudio para realizar segmentaciones
en el espacio de fases de la imagen, que luego podremos utilizar para obtener mapas
de uso y cobertura. Comenzaremos en este capitulo tambiena analizar el contexto
espacial de cada pixel.

En el capitulo \ref{educando}, \nameref{educando}, veremos otros metodos para extraer
informacion categorica sobre las imagenes satelitales al estudiar distintas
formas de clasificacion supervisada. Comenzaremos en este capitulo tambien a
analizar mas en detalle cuales son las coberturas que mayores problemas generan al
momento de realizar clasificaciones

En el capitulo \ref{pos}, \nameref{pos}, veremos algunas tecnicas de postprocesamiento
que nos permitiran analizar el contexto espacial de nuestras clasificaciones y
comenzar a calcular areas de cobertura para nuestras imagenes con su correspondiente incerteza.

\subsection{Cronograma}

Duranto el curso trabajaremos con el siguiente cronograma

\begin{itemize}
  \item[25/4] Instalar el QGIS y el R. Ver apendice.
  \item[26/4] \nameref{viaje}
  \begin{itemize}
    \item Clase teorica: radiancia y reflectancia. Firmas espectrales y espacio espectral.
    \item Clase practica: capitulo \ref{viaje} de la guida practica.
    \item Lectura recomendada: Remote sensing digital Image Analysis - Jogn A. Richards. Capitulos 1 y 3
  \end{itemize}
  \item[2/5] Entrega del cuestionario 1.
  \item[3/5] \nameref{rebotando}
  \begin{itemize}
    \item Clase teorica: Transferencia radiativa. Metodos estadisticos y modelado de la atmosfera.
    \item Clase practica: capitulo de \ref{rebotando} la guia practica.
    \item Lectura recomendada: Remote sensing digital Image Analysis - Jogn A. Richards. Capitulos 2
  \end{itemize}
  \item[9/5] Entrega del cuestionario 2.
  \item[10/5] \nameref{abaco}
  \begin{itemize}
    \item Clase teorica: C\'alculo de indices espectrales
    \item Clase practica: capitulo de \ref{abaco} la guia practica.
    \item Lectura recomendada: Quantitative Remote Sensing - ShunLin Liang. Capitulos 8
  \end{itemize}
  \item[16/5] Entrega del cuestionario 3.
  \item[17/5] Clase de consulta.
  \item[23/5] Entrega del trabajo practico 1.
  \item[24/5] \nameref{rotaciones}
  \begin{itemize}
    \item Clase teorica: Transformada Tasseled Cap y Analisis por componentes principales.
    \item Clase practica: capitulo de \ref{rotaciones} la guia practica.
    \item Lectura recomendada: Remote sensing digital Image Analysis - Jogn A. Richards. Capitulos 6
  \end{itemize}
  \item[30/5] Entrega del cuestionario 4.
  \item[31/5] \nameref{otrolado}
  \begin{itemize}
    \item Clase teorica: Metodos de clasificacion no supervisados.
    \item Clase practica: capitulo de \ref{otrolado} la guia practica.
    \item Lectura recomendada: Remote sensing digital Image Analysis - Jogn A. Richards. Capitulos 9
  \end{itemize}
  \item[6/6] Entrega del cuestionario 5.
  \item[7/6] \nameref{educando}
  \begin{itemize}
    \item Clase teorica: Metodos de clasificacion supervisados.
    \item Clase practica: capitulo de \ref{educando} la guia practica.
    \item Lectura recomendada: Remote sensing digital Image Analysis - Jogn A. Richards. Capitulos 8
  \end{itemize}
  \item[13/6] Entrega del cuestionario 6.
  \item[14/6] \nameref{pos}
  \begin{itemize}
    \item Clase teorica: Tecnicas post clasificacion.
    \item Clase practica: capitulo de \ref{pos} la guia practica.
    \item Lectura recomendada: Making better use of accuracy data in land change studies: Estimating accuracy and
area and quantifying uncertainty using stratified estimation - Olofsson et al.
  \end{itemize}
  \item[20/6] Entrega del cuestionario 7.
  \item[21/6] Clase de consulta. Arreglar horario.
  \item[27/6] Entrega del trabajo practico 2.
  \item[30/6] Entrega de certificados.
\end{itemize}

\subsection{Forma de aprobacion}
Para aprobar el curso se deben juntar al menos 100 puntos entre las distintas actividades.
Adem\'as se deber\'a completar un cuestionario \emph{obligatorio} sobre el curso
al finalizar el mismo.

La nota final del curso estara dada por la siguiente

\begin{itemize}
\item 0 - 99 - No aprobó
\item 100-109 - Seis
\item 110-129 - Siete
\item 130-169 - Ocho
\item 170-189 - Nueve
\item 190-200 - Diez
\end{itemize}

para obtener puntos puede realizarse de la siguiente manera

\begin{itemize}
  \item Cada cuestionario: 0 a 10 puntos. Máximo 70.
  \item Cada tarea: 0 a 50 puntos. Máximo 100.
  \item Participar en la plataforma: 0 a 10 puntos. Sin máximo.
\end{itemize}

\section{Materiales del curso}
Todos los materiales del curso, con sus correspondientes editables, pueden
encontrarse en el repositorio de github \url{https://github.com/fnemina/curso-sopi-herramientas-cuantitativas}.

En caso de encontrar cualquier problema en los materiales puede reportarlo ahi y sera
subsanado en el menor tiempo posible.

\part{Variables continuas}

\chapter{Un viaje del sol a los p\'ixeles.}
\label{viaje}
En esta primera práctica nos familiarizaremos con las interfaces gráficas del
qgis y de R-studio. Para esto analizaremos la imagen Landsat 8 de noviembre de
2016desde el punto de vista espectral. Son nuestros objetivos

\begin{itemize}
    \item Abrir una imagen en qgis.
    \item Crear archivos vectoriales y digitalizar coberturas en qgis.
    \item Abrir un archivos raster y vectoriales en R.
    \item Realizar un análisis estadistico de la imagen y de las
        distintan coberturas digitaliza en R.
\end{itemize}
\subsection{Exploración de imagenes con el qgis}

Comenzamos abriendo la imagen \file{LC82240782016304LGN00.vrt} que se encuentra
en la carpeta \file{raster\_data,LC82240782016304}. Esta image  corresponde al
departamento de Iguazu en la provincia de Misiones. Esta fue obtenida por
el satelite Landsat 8 durante el mes de noviembre de 2016.

Para esto vamos al menú \menu{Capa, Añadir capa, Añadir capa ráster}. Navegamos
hasta la carpeta \file{raster\_data/LC8224078201630} y abrimos el archivo
\file{LC82240782016304LGN00.vrt}. Una vez abierto el mismo podremos encontrarlo
en el \menu{Panel de capas} de q-gis desde donde podremos cambiar las opciones
de visualizacion y estudiar sus propiedades.

\begin{figure}[h!]
\begin{center}
    \includegraphics[scale=0.5]{move.png}
\end{center}
\caption{Herramientas para moverse dentro de la imagen. De izquierda a derecha:
    1. Desplazar mapa, 2. Desplazar mapa a la seleccion, 3. Acercar zum, 4.
    Alejar zum, 5. Zum a la resolucion nativa, 6. Zum general, 7. Zum a la
    seleccion, 8. Zum a la capa, 9. Zum anterior, 10. Zum siguiente, 11.
    Actualizar.}
\label{fig:move}
\end{figure}

Para realizar cambios en la visualizacion y explorar las propiedades de una
capa, hacemos click derecho sobre ella y luego seleccionamos la opcion
\menu{Propiedades}. Dentro de las propiedades podemos ir a la pestaña
\menu{General} para ver datos como el nombre de la capta\footnote{Es un buen
momento para ponerle uno mas sencillo}, la cantidad de filas y columnas del
archivo, el valor digital no valido, el sistema de referencias de coordenadas
entro otros.

\begin{figure}[h!]
\begin{center}
    \includegraphics[scale=0.3]{general.png}
\end{center}
\caption{Pestaña general de propiedades de una capa. En la misma se pueden ver
    los datos mas importantes sobre la misma como la cantidad de filas y
    columnass, el nombre y el sistema de referencia de coordenadas.}
\label{fig:general}
\end{figure}

Podemos ir luego a la pestaña de \menu{Estilo} para cambiar la visualizacion de
la capa. All\'i podemos elegir de que color mostraremos cada una de las
bandas adem\'as de cambiar el realce. Una vez elegidas las bandas debemos hacer
click en el boton \menu{Cargar} para seleccionar los valores maximos y minimos
de las bandas para el realce.

\begin{figure}[h!]
\begin{center}
    \includegraphics[scale=0.3]{estilo.png}
\end{center}
\caption{Estilos de visualizacion de una capa raster. Los estilos posibles son:
    1. Color de multibanda, 2. En paleta, 3. Unibanda gris, 4. Unibanda
    pseudocolor. Puede explorar cada uno por separado ya que todos tendran
    distintas utilidades.}
\label{fig:estilo}
\end{figure}

La herramienta \menu{Identificar un objeto espacial} nos permite extraer valores
de una coordenada espacial de nuestra imagen. Al habilitarla y
hacer click sobre un punto de la imagen veremos datos de la misma como por
ejemplo los valores de reflectancia del pixel seleccionado. Dichos valores
pueden mostrarse como Arbol, Tabla o Grafo segun como sea mas util.

\begin{figure}[h!]
\begin{center}
    \includegraphics[scale=0.3]{grafo.png}
\end{center}
\caption{Identificacion de un pixel correspondiente a la selva paranaense
    mostrada como grafo. }
\label{fig:grafo}
\end{figure}

\begin{act}
    Cambie la combinación de bandas de la imagen L8 a color real y explorela.
    Identifique zonas de coberturas uniformes. Pruebe cambiar la
    combinacion de bandas y decida si dichas zonas siguen siendo uniformes
    despues de cada cambio.
\end{act}

\begin{act}
    Encuentre el sistema de coordenadas en el cual se encuentra la imagen.
    ¿Cuantas filas y columns tiene la imagen?
\end{act}

\begin{act}
   Utilizando la herramienta identificar objetos espaciales encuentre los
   valores de reflectancia de distintas coberturas. Grafique estos  valores en
   una firma espectral y en el espacio de fases nirrojo.
\end{act}

\subsection{Creacion de capas vectoriales}

Veamos ahora como crear capas vectoriales. Nos van a ser principalmente de
de utilidad para extrar datos cuantitativos de las capas raster.

Con la herramienta \menu{nueva capa de archivo shape} es posible crear una nueva
capa vectorial. Para esto hacemos click en el boton del mismo nombre que se
encuentra en el panel lateral. Podemos agregar los campos que sean necesarios
para nuestra capa vectorial. En este caso crearemos lso campos MC\_ID como
entero de longitud 1 y Comment como texto de 80 caracteres. Elegimos el sistema
de coordenadas correspondiente a la imagen anterior. La guardamos en la carpeta
\file{vector\_data/} con el nombre \file{firmas.shp}.

\begin{figure}[h!]
\begin{center}
    \includegraphics[scale=0.3]{new_shape.png}
\end{center}
\caption{Creacion de una nueva capa vectorial. Se agregan campos que seran de
    interes para comparar las firmas espectrales. }
\label{fig:newshape}
\end{figure}


Una vez creada la nueva capa podemos utilizar la barra de herramientas de qgis
para agregar nuevas geometrias a la misma. Para esto hacemos click en el boton
de agregar geometrica y digitalizamos una zona uniforme dentro de la imagen.
\begin{figure}[h!]
\begin{center}
    \includegraphics[scale=0.5]{shapetool.png}
\end{center}
\caption{Herramientas de edición vectorial. De izquierda a derecha: 1. Conmutar
    edicion, 2. Guardar cambios a la capa, 3. Añadir objeto espacial, 4. Añadir
    cadena circular, 5. Mover objeto espacial, 6. Herramienta de nodos, 7.
    Borrar lo seleccionado, 8. Cortar objetos espaciales, 9. Copiar objetos
    espaciales, 10. Pegar objetos espaciales.}
\label{fig:shapetool}
\end{figure}

Al terminar de acerlo qgis pedira un numero de ID para la capa que debe ser
correlativo. Además podremos ingresar en este momento los valores del resto de
los campos de nuestro objeto espacial.

Es importante recordar que debemos estar en el modo de edicion para poder hacer
esto y salir de al terminarla.

\begin{figure}[h!]
\begin{center}
    \includegraphics[scale=0.3]{new_poli.png}
\end{center}
    \caption{Valores de los campos del nuevo poligono creado.}
    \label{fig:newpoli}
\end{figure}

\begin{act}
   Digitalize coberturas uniformes dentro de la imagen. Recuerde obtener al
   menos una por cada categoria de uso y cobertura presente dentro de la misma.
\end{act}

En caso de desear cambiar la visualizacion de la capa vectorial, podemos entrar
a las propiedades de la misma\footnote{Pueded utilizar el estilo precargado
ubicado en la carpeta \file{aux\_data}}. Ademas podemos acceder a la tabla de
datos de la capa vectorial haciendo click derecho sobre la misma y eligiendo la
opcion \menu{Abrir tabla de atributos}.

\subsection{Exploracion raster en R}
Veamos como abrir y trabajar con las imagenes satelitales en R. La forma de
realizar operaciones es escribir comandos en la consola de R-studio y
ejecutarlos de a uno. Para trabajar con imagenes satelitales debemos utilizar
algunas librerias adicionales. Para cargarlas usamos el comando
\texttt{library(raster)}. De esta forma agregamos funciones a las basicas de
R que nos facilitaran el trabajo raster.

Además, deberemos situar nuestra carpeta de trabajo donde se encuentran las
carpetas que descargamos. Para esto nos movemos en el explorar de archivos
hasta la misma y hacemos click en usar la carpeta como carpeta de trabajo.

\begin{figure}[h!]
\begin{center}
    \includegraphics[scale=0.3]{setwd.png}
\end{center}
\caption{Configuracion del directorio de trabajo desde la interfaz grafica.}
\label{fig:setwd}
\end{figure}

Tambien podemos utilizar el comando \texttt{setwd(.)} para configurar el
directorio de trabajo.

Una vez en dicha carpeta, existen varias maneras de abrir una imagen segun
queramos hacerlo solo para una banda, varias bandas en archivos separados o un
solo archivo multibanda.

Los comandos para esto son \texttt{raster}, para abrir una unica banda,
\texttt{brick}, para abrir un archivo multibanda, y \texttt{stack} para abrir
distinas bandas por separado. Veamos algunos ejemplo de esto:

\begin{exa}
    Abrimos la imagen completa del archivo de Landsat 8 y consultamos sus
    propiedades.
    \begin{lstlisting}
    ref.2016 <- brick("raster_data/LC82240782016304/LC82240782016304LGN00.vrt")
    ref.2016
    \end{lstlisting}
    obtenemos de resultado el siguiente text
    \begin{Verbatim}[fontsize=\small]
    class       : RasterBrick
    dimensions  : 2412, 1834, 4423608, 6  (nrow, ncol, ncell, nlayers)
    resolution  : 30.00402, 30.00265  (x, y)
    extent      : 731118.6, 786146, 7101531, 7173897  (xmin, xmax, ymin, ymax)
    coord. ref. : +proj=utm +zone=21 +south +datum=WGS84 +units=m +no_defs
                  +ellps=WGS84 +towgs84=0,0,0
    data source : ./material/raster_data/LC82240782016304/LC82240782016304LGN00.vrt
    names       : LC82240782016304LGN00.1, LC82240782016304LGN00.2, ...
    min values  :                     -33,                     192, ...
    max values  :                    2774,                    3265, ...
    \end{Verbatim}
    En el podemos ver la clase a la que corresponde el archivo abierto, en este
    caso un \emph{RasterBrick}, las dimensiones, el tamaño de pixel, extension
    de la capa, proyeccion, cual es la ruta al archivo que abrimos, las bandas y
    sus valores maximos y minimos.

    Trabajemos ahora con este raster. Vamos a cambiarle el nombre a las bandas
    y convertirla a reflectancia entre 0 y 1.

    \begin{lstlisting}
    ref.2016 <- brick(filename)
    names(ref.2016) <- c("blue","gree","red","nir","swir1","swir2")
    ref.2016 <- ref.2016/1e4
    rasterOptions(addheader = "ENVI")
    writeRaster(ref.2016,"raster\_data/processed/ref2016")
    \end{lstlisting}

    Analicemos el codigo linea por linea.
    \begin{itemize}
    \item La primera de ellas abre la imagen como  un raster de multiples bandas.
    \item La segunda, cambia los nombres de cada banda a los que figuran en la
          lista entre parentesis. Es importante resaltar que el numero de nombres
          debe ser el mismo que el de bandas.
    \item En tercer lugar convertimos el archivo de numeros enteros entre 0 y
          10000 a numeros entre 0 y 1.
    \item La cuenta linea es necesaria correrla una sola vez por sesion. La misma
          agrega el header de ENVI a nuestro output para poder abrir el archivo
          desde qgis
      \item La sexta linea guarda el archivo raster con el nombre \file{ref2016}
          . En este caso estamos usando el formato nativo de R.
    \end{itemize}
    podemos ademas graficar tanto una combinacion de bandas en qgis
    \begin{lstlisting}
    plotRGB(ref.2016,r=4,g=5,b=3, stretch='lin')
    \end{lstlisting}
    Obtenemos como resultado
    \begin{figure}[h!]
    \begin{center}
        \includegraphics[scale=0.3]{plot453.png}
    \end{center}
    \caption{Combinacion de bandas nir-swir1-red en R.}
    \label{fig:}
    \end{figure}
    como tambien todas las bandas por separado
    \begin{lstlisting}
    plotRGB(ref.2016)
    \end{lstlisting}
    obtenemos como resultado
    \begin{figure}[h!]
    \begin{center}
        \includegraphics[scale=0.3]{plotband.png}
    \end{center}
    \caption{Grafico de bandas con realce automatico para cada una.}
    \label{fig:plotband}
    \end{figure}

\end{exa}

\begin{act}
   Abra el archivo vrt en qgis y vuelva a mirar la firma espectral para
   distintas coberturas. ¿Entre que valores se encuentra ahora las mismas?
\end{act}

\begin{exa}

    Hagamos un poco de analisis ahora sobre la imagen. En primer lugar podemos
    calcular la estadistica basica sobre la imagen. Para ello ejecutamos el
    comando \texttt{summary(ref.2016)}
    obtenemos como resultado
    \begin{Verbatim}[fontsize=\small]
               blue   gree    red     nir   swir1   swir2
    Min.    -0.0278 0.0000 0.0000 -0.0128 -0.0069 -0.0038
    1st Qu.  0.0128 0.0328 0.0184  0.2763  0.1198  0.0493
    Median   0.0138 0.0362 0.0203  0.3287  0.1365  0.0572
    3rd Qu.  0.0170 0.0450 0.0329  0.3557  0.1644  0.0749
    Max.     0.5548 0.8257 0.8034  0.7542  0.9181  0.9446
    NA's     0.0000 0.0000 0.0000  0.0000  0.0000  0.0000
    \end{Verbatim}
    Para comenzar podemos calcular los histogramas de todas las bandas con el
    comando \texttt{hist(ref.2016)} y el scatter plot entre dos bandas como
    \texttt{plot(l8$red, l8$blue)}

    En caso de querer todos los scatterplots e histogramas en un solo grafico
    podemos hacerlo con el comando \texttt{pairs(l8)}.
    \end{exa}


\subsection{Manejo vectorial en R}

Hasta ahora estamos analizando la imagen completa. Podemos sin embargo analizar
solo sectores concretos de la imagen muestreandola en funcion de un archivo
vectorial. Tambien sera posible muestrar la imagen pos zonas definidas por otro
raster pero veremos esto mas adelante.

Para poder trabajar con vectores en R utilizaremos la libreria
\texttt{library(rgal)}.

\begin{exa}
    Veamos como realizar el an\'alisis b\'asico de un vector en R. Comenzamos
    leyendolo

    \begin{lstlisting}
    firmas <- readOGR(dsn="vector\_data/", layer="firmas")
    \end{lstlisting}

    Notamos en este caso que debemos indicar por separado la carpeta que
    contiene al shapefilee en \emph{dsn} y el nombre de la capa que queremos
    abrir como \emph{layer}.

    Podemos mostrar las propiedades del vector llamando a la variable
    \texttt{firmas} obteniendo como resultado
    \begin{Verbatim}[fontsize=\small]
    class       : SpatialPolygonsDataFrame
    features    : 8
    extent      : 738692.8, 767774.6, 7133396, 7165265  (xmin, xmax, ymin, ymax)
    coord. ref. : +proj=utm +zone=21 +south +datum=WGS84 +units=m +no_defs
                  +ellps=WGS84 +towgs84=0,0,0
    variables   : 3
    names       : id, MC_ID,       Comment
    min values  :  0,     1,          Alto
    max values  :  9,     8, Suelo desnudo
    \end{Verbatim}

    Podemos graficar los vectores obtenidos en R junto aa la imagen de base como

    \begin{lstlisting}
    plotRGB(ref.2016, stretch="lin")
    plot(firmas,add=TRUE,col='red')
    \end{lstlisting}
    donde la primera linea grafica la imagen de fondo y la segunda agrega el
    el shapefile sobre la misma.
\end{exa}

\begin{act}
    Muestre las propiedades de la capa raster y el vector abiertos y verifique
    que los mismos se encuentren en el mismo sistema de coordenadas.
\end{act}

Por ultimo mostremos como extraer datos de un archivo raster y veamos un par de
ejemplo concretos. La funcion que nos permite extrar datos de un raster segun
un vector es \texttt{extract} que toma dos argumentos, el vector que queremos
utilizar y la capa raster sobre la cual hacer la consulta.

Veamos algunos ejemplos

\begin{exa}
    Graficar en un scatterplot de dos bandas mostrando la zona del espacio
    ocupada por una cobertura.
    \begin{lstlisting}
    datos <- extract(ref.2016,firmas)
    \end{lstlisting}
    de esta forma realizamos la extraccion de todos los datos de la imagen a una
    lista
    \begin{lstlisting}
    plot(ref.2016$red, ref.2016$nir)
    points(as.data.frame(datos[1])$red, as.data.frame(datos[1])$nir,col="green",
           pch = ".")
    \end{lstlisting}
    obteniendo como resultado
    \begin{figure}[h!]
    \begin{center}
        \includegraphics[scale=0.3]{plot-red-nir-zone.png}
    \end{center}
    \caption{Resultado del scatterplot para las bandas roja y nir. Se muestra en
        verde datos correspondientes a la selva paranaense.}
    \label{fig:}
    \end{figure}

\end{exa}

La funcion \texttt{extract} nos permite tambien aplicar una funcion a los datos
extraidos antes de entregarlos al usuario. Veamos como usarla para calcular
datos de interes sobre las coberturas y guardarlo en un archivo vectorial.

\begin{exa}
     Extraer los promedios y desvios standar de un raster y agregarlos a un
     vector. Primero extraemos los valores de promedio y desvio
     \begin{lstlisting}
     promedio <- extract(ref.2016,firmas,fun=mean)
     desvio <- extract(ref,firmas,fun=sd)
     \end{lstlisting}
     renombramos luego las columnas como promedio y devio seguido de la banda a
     la que pertenecen,
     \begin{lstlisting}
     colnames(promedio) <- paster("mean",colnames("promedio"),sep="_")
     colnames(desvio) <- paster("sd",colnames("desvio"),sep="_")
     \end{lstlisting}
     finalmente agregamos los archivos a un nuevo shapefile
     \begin{lstlisting}
     firmas@data <- cbind(firmas@data,promedio,desvio)
     writeOGR(firmas, sdn="vector_data/processed/,"firmas_datos",
              driver="ESRI Shapefile")
     \end{lstlisting}
\end{exa}

Por ultimo, veamos como usar una capa vectorial para graficar para extraer las
firmas espectrales y graficarlas para distintas coberturas

\begin{exa}
     Graficar las firmas espectrales en funcion de la longitud de onda para cada
     geometria de un vector. Utilizaremos en este caso dos nueva libreria,
     \texttt{reshape2} y \texttt{lattice}

     Comenzamos convirtiendo en dataframe a nuestros promedios donde cada
     columna corresponde a una firma espectral
     \begin{lstlisting}
     df <- t(promedio)
     colnames(df) <- vector@data$Comment
     \end{lstlisting}
     Agregamos luego una columna con las longitudes de onda en nanometros. Luego
     reformamos el dataframe para que podamos subsetearlo, poniendo finalmente
     los nombres a cada columna
     \begin{lstlisting}
     df$wl <- as.matrix(c(485,560,660,830,1650,2215))
     df <- melt(df,id.vars="wl", variable.name="cobertura")
     names(df) <- c("wl","Cobertura","Reflectancia")
     \end{lstlisting}
     si mostramos el dataframe, el resutado debería ser similar al siguiente
     \begin{Verbatim}[fontsize=\small]
          wl     Cobertura Reflectancia
     1   485          Alto  0.012926561
     2   560          Alto  0.034730646
     3   660          Alto  0.018491884
     4   830          Alto  0.354564681
     5  1650          Alto  0.133750642
     ...
     \end{Verbatim}
     repetimos el proceso para los desvios standar
     \begin{lstlisting}
     dfd <- t(desvio)
     colnames(dfd) <- vector@data$Comment
     dfd$wl <- as.matrix(c(485,560,660,830,1650,2215))
     dfd <- melt("wl","Cobertura","Desvio")
     df$desvio <- dfd$desvio
     df$MC_ID <- as.character(vector@data$MC_ID[match(df$Cobertura,
                              vector@data$Comment)])
     \end{lstlisting}
     el resultado sera ahora
     \begin{Verbatim}[fontsize=\small]
         wl     Cobertura Reflectancia       Desvio
     1   485          Alto  0.012926561 0.0007772473
     2   560          Alto  0.034730646 0.0018113004
     3   660          Alto  0.018491884 0.0011561294
     4   830          Alto  0.354564681 0.0166801398
     5  1650          Alto  0.133750642 0.0075157929
     ...
     \end{Verbatim}
     Veamos algunas opciones para generar ahora los graficos. En primer lugar
     pondremos todas las firmas juntas, separadas por color, usando la libreria
     \texttt{lattice}
     \begin{lstlisting}
     xyplot(Reflectancia~wl, data=df, groups = Cobertura,
            auto.key=list(space="top", columns=4),
            ty=c("l", "p"))
     \end{lstlisting}
     Aqui la primer linea dice que grafiquemos la reflectancia como funcion de
     la longitud de onda, obteniendo los datos del dataframe df y agrupandolos
     segun la columna cobertura. La siguiente linea agrega la leyenda en la
     parte superior de la figura y con 4 columnas. Por ultimo en la tercer linea
     pedimos que el grafico tenga lineas y puntos.
     \begin{figure}[h!]
     \begin{center}
         \includegraphics[scale=0.3]{spectra-1.png}
     \end{center}
     \caption{Firmas espectrales}
     \label{fig:spectra-1}
     \end{figure}
     Si queremos agruparlo por categoria de uso y cobertura cambiamos la formula
     \texttt{Reflectancia ~ wl} por \texttt{Reflectancia ~ wl | MC\_ID}
     \begin{lstlisting}
     xyplot(Reflectancia~wl | MC_ID, data=df, groups = Cobertura,
            auto.key=list(space="top", columns=4),
            ty=c("l", "p"))
     \end{lstlisting}
     y finalmente si queremos graficar solo un subset de los daatos
     \begin{lstlisting}
     xyplot(Reflectancia~wl | MC_ID, data=df, groups = Cobertura,
            auto.key=list(space="top", columns=4), ty=c("l", "p"),
            subset = Cobertura %in% c("Alto","Bajo"))
     \end{lstlisting}
     \begin{figure}[h!]
     \begin{center}
         \includegraphics[scale=0.3]{spectra-2.png}
     \end{center}
     \caption{Firmas espectrales}
     \label{fig:spectra-1}
     \end{figure}
 \end{exa}

\begin{act}
    Grafique la media y el desvio standar para las distintas coberturas que pudo
     identificar en el punto uno.
\end{act}


\chapter{Rebotando por la atm\'osfera}
\label{rebotando}

En esta segunda actividad practica nos centraremos en la correccion radiometrica
de imagenes satelitales. Son objetivos de la misma

\begin{itemize}
    \item Poder abrir una imagen satelital desde el metadato.
    \item Convertir los valores de la imagen a reflectancia tope de la
        atmosfera.
    \item Corregir la imagen satelital utilizando los metodos de \emph{dos} y
        \emph{cost}
    \item Corregir la imagen satelital utilizando el \emph{6S web}
\end{itemize}

\subsection{Calculo de reflectancia a tope de la atmosfera}

Para poder convertir una imagen a reflectancia a tope de la atmosfera vamos a
necesitar no solo la imagen sino tambien la informacion adicional que hallaremos
en su metadato.

Para abrir una imagen satelital desde el metadato utilizaremos las funciones
disponibles en \texttt{RStoolbox}. Este incluye diversas herramientas
para trabajar con imagenes satelitales.

\begin{exa}
   Comencemos analizando un ejemplo sencillo, abriremos la imagen Landsat 7
    del año 2000 desde el metadato y la mostraremos en combinacion de bandas de falso color
    compuesto, ademas de analizar las propiedades basicas de la misma.
    \begin{lstlisting}
    meta.2000 <- readMeta("raster_data/LE72240782000188EDC00/LE72240782000188EDC00_MTL.txt")
    \end{lstlisting}
    Podemos mostrar las distintas variables incluidas en el objeto usando el
    signo \$ y el nombre de la misma. Por ejemplo \verb|meta.2000$SOLAR_PARAMETERS|
    da como resultado
    \begin{Verbatim}[fontsize=\small]
     azimuth elevation  distance
    37.38251  31.14409   1.01670
    \end{Verbatim}
    A partir del metadato podemos cargar la imagen completa con el comando
    \texttt{stackMeta}. Ademas eliminaremos en este caso las bandas 6 y 7 por
    ser termicas.
    \begin{lstlisting}
    dn.2000 <- stackMeta(meta.2000)
    dn.2000 <- dn.2000[[-6:-7,]]
    dn.2000
    \end{lstlisting}
    obtenemos como resultado un objeto raster stack como el que sigue
    \begin{Verbatim}[fontsize=\small]
    class       : RasterStack
    dimensions  : 2412, 1834, 4423608, 6  (nrow, ncol, ncell, nlayers)
    resolution  : 30.00402, 30.00265  (x, y)
    extent      : 731118.6, 786146, 7101531, 7173897  (xmin, xmax, ymin, ymax)
    coord. ref. : +proj=utm +zone=21 +south +datum=WGS84 +units=m +no_defs
                  +ellps=WGS84 +towgs84=0,0,0
    names       : B1_dn, B2_dn, B3_dn, B4_dn, B5_dn, B7_dn
    min values  :     0,     0,     0,     0,     0,     0
    max values  :   255,   255,   255,   255,   255,   255
    \end{Verbatim}
    y podemos mostrar la iamgen como hicimos antes
    \begin{lstlisting}
    plotRGB(dn.2000, r=3, g=2, b=1, stretch="lin")
    \end{lstlisting}

     \begin{figure}[h!]
     \begin{center}
         \includegraphics[scale=0.6]{dn-l7-rgb}
     \end{center}
     \caption{Imagen en combinacion color real de la zona de interes sobre la
         imagen en DN Landsat 7.}
     \label{fig:dn-l7-rgb}
     \end{figure}
\end{exa}

De esta forma podemos tener el archivo cargado en DN con todos sus metadatos
para convertirlo a reflectancia reflectancia y realizar distintas correcciones.
Para pasar nuestra imagen a reflectancia a tope
de la atmosfera tenemos dos maneras de hacerlo. Podemos hacerlo a mano
utilizando las herramientas algebraicas de R o podemos hacerlo con la funcion
especifica de \texttt{RStoolbox}.

\begin{exa}
    Calculo de reflectancia a tope de la atmosfera
    utilizando el metadato paso por paso

    \begin{lstlisting}
    dn2ref.2000 <- meta.2000$CALREF[1:6,]
    elev.2000 <- pi*meta.2000$SOLAR_PARAMETERS['elevation']/180
    \end{lstlisting}
    extraemos primero del metadatos los parametros de calibracion en
    reflectancia y el angulo de elevacion solar.

    Convertimos luego la imagen a reflectancia y la dividimos luego por el
    angulo solar. Luego cambiamos los nombres de las bandas

    \begin{lstlisting}
    toam.2000 <- (dn.2000*dn2ref.2000$gain+dn2ref.2000$offset)/sin(elev.2000)
    names(toam.2000) <- c("blue","green","red","nir","swir1","swir2")
    \end{lstlisting}

    Otra forma forma de realizar este proceso es utilizando la funcion
    \texttt{radCor}. En este caso debemos dar la imagen en DN, el metadato y
    cual es la cantidad que queremos calcular.

    \begin{lstlisting}
    toa.2000 <- radCor(dn.2000, metaData = meta.2000, method = "apref")
    \end{lstlisting}

    podemos comparar los resultados de ambos metodos inspeccionando los objetos
    \texttt{toam.2000} y \texttt{toa.2000}.
    \begin{Verbatim}[fontsize=\small]
    class       : RasterBrick
    dimensions  : 2412, 1834, 4423608, 6  (nrow, ncol, ncell, nlayers)
    resolution  : 30.00402, 30.00265  (x, y)
    extent      : 731118.6, 786146, 7101531, 7173897  (xmin, xmax, ymin, ymax)
    coord. ref. : +proj=utm +zone=21 +south +datum=WGS84 +units=m +no_defs
                  +ellps=WGS84 +towgs84=0,0,0
    data source : in memory
    names       :        blue,       green,         red,         nir,    swir1,       swir2
    min values  : -0.01976113, -0.02181530, -0.02029439,  0.01934678,    -0.02781926, -0.02678077
    max values  :   0.6106812,   0.5609009,   0.6079443,   0.8696885,    0.8640919,   0.8263815
    \end{Verbatim}

    y

    \begin{Verbatim}[fontsize=\small]
    class       : RasterStack
    dimensions  : 2412, 1834, 4423608, 6  (nrow, ncol, ncell, nlayers)
    resolution  : 30.00402, 30.00265  (x, y)
    extent      : 731118.6, 786146, 7101531, 7173897  (xmin, xmax, ymin, ymax)
    coord. ref. : +proj=utm +zone=21 +south +datum=WGS84 +units=m +no_defs
                  +ellps=WGS84 +towgs84=0,0,0
    names       :     B1_tre,     B2_tre,     B3_tre,     B4_tre,     B5_tre,    B7_tre
    min values  : 0.00000000, 0.00000000, 0.00000000, 0.01934678, 0.00000000,    0.00000000
    max values  :  0.6106812,  0.5609009,  0.6079443,  0.8696885,  0.8640919,    0.8263815
    \end{Verbatim}

    \end{exa}

\begin{act}
    Inspeccione la reflectancia a tope de la atmosfera para todas las bandas.
    Para esto realice los histogramas, graficos de dispersion, calcule la media,
    el desvio standar y cualquier otra medida estadistica que le guste.
\end{act}
\subsection{Calculo de reflectancia corregida atmosfericamente por metodos
            estadisticos}

La funcion \texttt{radCor} dispone de un parametro para hacer distintas
tipos de correcciones atmosfericas. Ya vimos \emph{apref} que nos permitio
calcular la reflectancia a tope de la atmosfera. Veamos como aplicar el metodo
de substraccion de cuerpo obscuro.

\begin{exa}
    Apliquemos el metodo de \emph{simple dos} para corregir la imagen. En este caso
    solamente restaremos el minimo en cada banda a la imagen para las bandas
    donde existe haze, es decir en la zona del visible y del infrarrojo cercano.

    Estimamos el haze primero y corregimos la imagen luego haciendo
    \begin{lstlisting}
    haze.2000 <- estimateHaze(dn.2000,darkProp = 0.01, hazeBands = 1:4, plot=TRUE)
    sdos.2000 <- radCor(dn.2000, metaData = meta.2000,
                 hazeValues = haze.2000,
                 hazeBands = c("B1_dn","B2_dn","B3_dn","B4_dn"),
                 method="sdos")
    \end{lstlisting}
    en este caso los valores de haze estimados son
    \begin{Verbatim}[fontsize=\small]
    B1_dn B2_dn B3_dn B4_dn
       41    27    20    15
    \end{Verbatim}
    Para hacer un analisis de lo que pasa en la situacion, vamos a graficar los
    histogramas de cada banda para la imagen en reflectancia TOA y corregida por
    el metodo simple dos. Para esto usaremos el paquete \texttt{rasterVis}
    \begin{lstlisting}
    B1 <- densityplot(~B1_tre+B1_sre, data=toa.boa, xlab="Reflectancia",
                      ylab="", main="Banda azul", plot.points=FALSE, xlim=c(0,0.3),
                      key=simpleKey(text=c("Tope de la atmosfera",
                                           "Correccion Simple DOS"),
                                           lines=TRUE, points=FALSE))
    B2 <- densityplot(~B2_tre+B2_sre, data=toa.boa, xlab="Reflectancia",
                      ylab="", main="Banda verde", plot.points=FALSE, xlim=c(0,0.3),
                      key=simpleKey(text=c("Tope de la atmosfera",
                                           "Correccion Simple DOS"),
                                           lines=TRUE, points=FALSE))
    B3 <- densityplot(~B3_tre+B3_sre, data=toa.boa, xlab="Reflectancia",
                      ylab="", main="Banda roja", plot.points=FALSE, xlim=c(0,0.3),
                      key=simpleKey(text=c("Tope de la atmosfera",
                                           "Correccion Simple DOS"),
                                           lines=TRUE, points=FALSE))
    B4 <- densityplot(~B4_tre+B4_sre, data=toa.boa, xlab="Reflectancia",
                      ylab="", main="Banda nir", plot.points=FALSE, xlim=c(0,0.3),
                      key=simpleKey(text=c("Tope de la atmosfera",
                                           "Correccion Simple DOS"),
                                           lines=TRUE, points=FALSE))
     print(B1,split = c(1, 1, 2, 2),more=TRUE)
     print(B2,split = c(2, 1, 2, 2),more=TRUE)
     print(B3,split = c(1, 2, 2, 2),more=TRUE)
     print(B4,split = c(2, 2, 2, 2),more=FALSE)
    \end{lstlisting}
    En este caso las primeras 4 funciones crean los histogramas para cada banda
    corregida mientras que las ultimas 4 lineas los imprimen en una grilla.
    \begin{figure}[h!]
    \begin{center}
        \includegraphics[scale=0.4]{simpledos.png}
    \end{center}
    \caption{Graficos de los histogramas para las distintas bandas donde se
        muestra el nivel de correccion en cada una.}
    \label{fig:simpledos.png}
    Notamos en este caso que la correccion se vuelve menos importante a medida
        que crece la longitud de onda.
    \end{figure}

\end{exa}
\begin{act}
    Analice los valores de haze obtenidos por la funcion stimate haze y grafiquelos
    como funci\'on de la longitud de onda en escala logaritmica. ¿Que observa?
\end{act}

\begin{act}
    Utilice el metodo \emph{costz} para corregir la imagen a reflectancia a tope
    de la superficie.
\end{act}

\begin{act}
    Guarde los archivos raser generado por cada uno de los metodos de
    correccion. Abralos en qgis y comparelos visualmente. Obtenga firmas
    espectrales con los distintos metodos de correccion.
\end{act}


\subsection{6S}
\label{sub:corr:6S}

Veamos ahora como operar con el 6S para obtener una estimacion de los parametros
atmosfericos. Para esto utilizaremos la version web del 6S que se encuentra
disponible en http://6s.ltdri.org/pages/run6SV.html.

Para utilizarla ingresaremos a la pagina y haremos click en el boton
\menu{Submit query}. Iremos luego configurando paso a paso nuestro modelo de la
atmosfera haciendo siempre luego click en el boton \menu{submit query} para
pasar al paso siguiente.

Los parametros para nuestro modelo son

\begin{enumerate}
    \item Geometrical conditions
        \begin{itemize}
            \item TM (Landsat)
            \item Month: 4, Day:13, GTM decimal hour: 13.60, Longitude:
                -63.8606, Latitude: -24.9937.
        \end{itemize}
    \item Atmospheric Model
        \begin{itemize}
            \item Select Atmospheric Profile: Mid latitude summer
            \item Select aerosol model: Continental Model
            \item Visibility: 60
        \end{itemize}
    \item Target \& sensor altitude
        \begin{itemize}
            \item Select targe altitude: sea level
            \item Select sensor altitude: satellite level
        \end{itemize}
    \item Spectral conditions
        \begin{itemize}
            \item Select spectral conditions: choose band
            \item Select band: 1st band of tm (landat 5)
        \end{itemize}
    \item Ground reflectance
        \begin{itemize}
            \item Ground reflectance type: homogeneous surface
            \item Directional effect: no directional effect
            \item Specify surface reflectance: input constant value of ro
            \item input constant value for ro: 0
        \end{itemize}
    \item Signal
        \begin{itemize}
            \item Atmospheric correction mode: no atmospheric correction
        \end{itemize}
\end{enumerate}

Todos estos valores los encontramos en el metadato de la imagen.

En \menu{7.Results} podemos ver el resultado haciendo click en \emph{Output
file}. Del mismo debemos extraer los valores de
 \begin{itemize}
     \item \texttt{global gas. trans. - total}
     \item \texttt{total sca. trans. - total}
     \item \texttt{spherical albedo - total}
     \item \texttt{reflectance I - total}
 \end{itemize}

Una vez ejecutado el proceso puede usarse el siguiente codigo para corregir
todas las bandas utilizando R.

\begin{lstlisting}
    a <- c(0.98,0.90,...) # Global gas transmitance
    b <- c(0.81,0.90,...) # Total scatering transmitance
    g <- c(0.15,0.10,...) # Spherical albedo
    r <- c(0.08,0.05,...) # Reflectance I
    sss.2000 <- (toa.2000/(a*b)-r/b)/(1+g*(toa.2000/(a*b)-r/b))
\end{lstlisting}


\begin{act}
    Realice una extraccion de firmas espectrales para distintass coberturass de
    cada uno de los archivos raster obtenidos y grafiquelos en el mismo grafico.
    Comparela con la firma espectral obtenida a partir de la imagen corregida
    por el usgs.
\end{act}

\begin{act}
    Haga un grafico de densidades que muestre los distintos metodos de
    correccion atmosfericos para cada banda.
\end{act}

\begin{act}
    Calcule la diferencia promedio para cada banda entre las imagenes en
    reflectancia a tope de la atmosfera y las distintas correcciones
    y la imagen en reflectancia entregada por el USGS.
\end{act}


\chapter{Un \'abaco espectral}
\label{abaco}
Veamos ahora como realizar operaciones sencillas entre las bandas de una imagen.
Usaremos en esta practica los siguientes paquetes

\begin{lstlisting}
    library(raster)
    library(RStoolbox)
    library(RColorBrewer)
    library(rgdal)
    library(ggplot2)
    libyrary(GGally)
\end{lstlisting}

Comenzamos primer cargando la imagen desde el metadato y convirtiendola a
reflectancia como hicimos en la clase anterior

\begin{lstlisting}
    xml.2016 <- readMeta("raster_data/LC.../LC....xml")
    ref.2016 <- stackMeta(xml.2016, quantity = "sre")
    scaleF <- getMeta(ref.2015,xml.2016, what = "SCALE_FACTOR")
    ref.2016 <- ref.2016 * scaleF
    ref.2016 <- ref.2016[[-1,]]
    names(ref.2016) <- c("blue","green","red","nir","swir1","swir2")
\end{lstlisting}

una vez cargada la imagen podemos realizar operaciones entre las bandas llamando
a cada una por separado. Veamos como ejemplo el calculo de NDVI\@.

\begin{exa}
    Calculo de NDVI a mano y grafico del mismo
    \begin{lstlisting}
        ndvi.2016 <- (ref.2016$nir-ref.2016$red)/(nir.2016$nir+ref.2016$ref)
        cols = colorRampPalette(brewer.pal(9,"YlGn"))(16)
        plot(ndvi.2016, col=cols, zlim = c(0,1))
    \end{lstlisting}
    obteniendo una imagen como la que se ve debajo.
\end{exa}

El paquete \texttt{RStoolbox} tiene varias herramientas que nos ayudan a
calcular los indices espectrales. Veamos por ejemplo como calcular el NDVI y el
EVI utilizando dicho paquete

\begin{exa}
    Para calcular los indices mediante la funcion spectralIndices debemos
    especificar con que raster trabajamos y que bandas corresponden a cada
    longitud de onda
    \begin{lstlisting}
    indices.2016 <- spectralIndices(ref.2016, 
                                    blues="blue", red="red", nir="nir", 
                                    indices=c("NDVI","EVI"))
    plot(indices.2016,col=cols, zlim=c(0,1))
    \end{lstlisting}
    obtenemos una imagen como se muestra debajo.
\end{exa}

\begin{act}
    Calcule el NDVI para el año 2000 utilizando la imagen landsat 7.
\end{act}

\begin{act}
    Calcule y grafique todos los indices posibles que involucren a las bandas
    roja y nir de landsat 8. 
\end{act}


\begin{exa}
    Veamos ahora como calcular el tSAVI utilizando la linea de suelo obtenida a
    partir de la imagen. Para esto necesitaremos enmascarar las zonas con
    cobertura de agua y nubes. Veamos primer como hacer esto.
    \begin{lstlisting}
        mask.2016 <- raster("raster/.../...cfmask.tif")
        masked.2016 <- mask(ref.2016, mask=mask.2016, inverse=TRUE,
                            maskvalue=0, updatevalue=255)
        masked.2016[masked.2016<=0] <- 255
    \end{lstlisting}
    de esta forma enmascaramos todos los valores con nubes, agua y donde la
    reflectancia obtenida es cero.
    Calculamos ahora la linea de suelo y la mostramos en un scatterplot
    \begin{lstlisting}
        bsl.2016 <- BSL(as.matrix(masked.2016$red), as.matrix(masked.2016$nir),
                        method="quantile", ulimimt=0.99, llimit=0.001)
        plot(ref.2016$red, ref.2016$nir)
        abline(bsl.2016$BSL,col="red")
    \end{lstlisting}
\end{exa}

\begin{act}
    Calcule el tSAVI utilizando la linea de suelo obtenida arriba.
\end{act}

\begin{act}
    Vuelva a obtener la linea de suelo sin enmascarar la imagen y dibujo el
    scatterplot con la misma y la anterior. Que problema encuentra.
\end{act}

Finalmente, veamos como se puede obtener datos biofisicos a partir de los
indices de vegetacion calculados. De esta forma podremos generar mapas de
porcentaje de cobertura, productividad, etc.

\begin{act}
    Cargue la capa vectorial del muestreo de variables biofisicas
    \texttt{muestreo.shp} y haga una extraccion de los valores de NDVI
    correspondientes a dichos puntos. Guarde estos valores en un dataframe
    llamado \texttt{muestreo}.
\end{act}

\begin{exa}
    Veamos como ajustar con R un modelo lineal a nuestro modelo. Para esto
    comencemos haciendo un analisis visual con la funcion \texttt{ggpairs}.
    \begin{lstlisting}
        ggpairs(muestreo,diag=list(continuous="barDiag"))
    \end{lstlisting}
    Obtendremos un grafico que presenta los scatterplots entre las bandas, su
    correlacion e histogramas.
    Veamos en el mismo que la superficie cubierta por vegetacion varia
    linealmente con el NDVI\@. Por lo tanto utilizaremos estos para hacer un
    ajuste de nuestro modelo.
    \begin{lstlisting}
        lm.2016 <- lm(fcover~ndvi, data=muestreo)
        plot(muestreo$ndvi, mustreo$fcover)
        abline(lm.2016, col="red")
        summary(lm.2016)
    \end{lstlisting}
    de esta forma veremos los parametros de nuestro ajuste, y graficaremos al
    mismo en un scatterplot.

    Para aplicar el modelo a nuestro raster hacemos
    \begin{lstlisting}
        fcover.2016 <- predict(ndvi.2016,lm.2016)
        plot(fcover.2016)
    \end{lstlisting}
    Obteniendo el mapa de abajo.
\end{exa}

\begin{act}
    Genere los modelos de lai, fapar y fcover para el año 2016 y con los mismos
    realice mapas de dichas variables.
\end{act}

\begin{act}
    Utilizando los modelos obtenidos para 2016 aplique los mismos para obtener
    los mapas de lai, fapar y fcover del año 2000. Que suposicion esta
    haciendo?
\end{act}

\begin{act}
    * Utilizando la funcion spectralIndices y ggpairs, analice si hay otro indice
    que ajuste que correlacione mejor con las alguna de las varibles biofisicas
    medidas a campo.
\end{act}



\part{Variables discretas}

\chapter{Geometr\'ia espectral}
\label{rotaciones}
Durante la clase de hoy trabajaremos con rotaciones en el espacio espectral. A
diferencia del trabajo con indices las rotaciones pueden interpretarse no como
algebra entre las bandas sino como distintas formas de mirar al mismo espacio
espectral.

En este caso usaremos las librerias \texttt{raster} y \texttt{RStoolbox}.

\begin{lstlisting}
    library(raster)
    library(RStoolbox)
    library(bfastSpatial)
\end{lstlisting}


\begin{exa}
    Comencemos analizando la transformada por componentes principales de la
    imagen de 2016. Que podemos predecir?
    \begin{lstlisting}
        pairs(ref.2016)
    \end{lstlisting}

    Mirando el resumen de la imagen vemos que hay varias bandas muy
    correlacionadas entre si. Por ejemplo las del visible, mientras que otras lo
    estan poco, por el ejeplo el nir y el swir. Por lo tanto esperamos que no
    todas las bandas sean necesarias para explicar el comportamiento de la
    imagen


    \begin{lstlisting}
        pca.2016 <- rasterPCA(ref.2016)
        summary(pca.2016$model)
        loadings(pcs.2016$model)
        plot(pca.2016$map)
    \end{lstlisting}

\end{exa}

\begin{act}
    Calcule y analice la transformada por PCA de la imagen Landsat 7 del año
    2000.
\end{act}

\begin{exa}
    Otra aplicacion de la transformada por componentes principales por
    componentes principales. Veamos como realizarlo.
    \begin{lstlisting}
        ndvi.list <- list.files("raster_data/MOD13Q1/EVI/", pattern = "*.tif$",
                                 full.names = TRUE)
        ndvi.stack <- stack(ndvi.list)
    \end{lstlisting}
    una vez abierta la imagen la convertimos a valores entre -1 y 1 e
    interpolamos los valores que falten.
    \begin{lstlisting}
        ndvi.stack <- ndvi.stak/1e4
        ndvi.stack <- approxNA(ndvi.stack)
    \end{lstlisting}
    Una vez llenados los espacios donde no habia datos podemos aplicar la
    transformada por componentes principales y mostrarla
    \begin{lstlisting}
        ndvi.pca <- rasterPCA(ndvi.stack)
    \end{lstlisting}
\end{exa}

\begin{act}
    Grafique las primeras 4 componentes por de la transformada por componentes
    princiales de la imagen del stack de NDVI\@. Que zonas puede identificar en la
    primera? que zonas se distinguen en la segunda? que comportamiento encuentra
    en la tercera y cuarta.
\end{act}

\begin{act}
    Investigue la funcion \texttt{tasseledCap} y calcule la transformada
    tasseled cap para las imagenes landsat 7 y 8.
\end{act}

\begin{act}
    Grafique en el scatter-plot la imagen completa y marque en el mismo zonas
    con vegetacion, agua y suelo sin cobertura vegetal. Vea como cambian estas
    zonas frente a las transformadas por componentes principales y tasseled cap.
\end{act}



\chapter{Clasificacion no supervisada de imagenes}
\label{otrolado}
En esta clase vamos a trabajar con clasificaciones no supervisadas de imagenes
satelitales. Vamos a usar los paquetes

\begin{lstlisting}
    library(raster)
    library(RStoolbox)
\end{lstlisting}

Cargaremos primero la imagen landsat 8 y habilitaremos la opcion para escribir
el header de ENVI\@.

\begin{lstlisting}
    rasterOptions(addheader = "ENVI")
    set.seed(6)
    kmeans.2016 <- unsuperClass(ref.2016, nClasses = 5, nStarts = 100,
                                nSamples = 100)
    writeRaster(kmeans.2016, "raster_data/processed/kmeans2016",
                datatype="INT1U")
\end{lstlisting}

Podemos ahora graficar por separado cada una de las clases

\begin{lstlisting}
    classes.2016 <- layerize(kmeans.2016)
    plot(classes.2016)
\end{lstlisting}

Abriremos la imagen ahora en el qgis e identificaremos cada una de las clases
realiando interpretacion visual de la imagen. 

Para realizar la identificacion primero vamos al menu \menu{propiedades de la
imagen, Estilo, Tipo de renderizacion, Unibanda pseudocolor}. Elegimos de modo
Intervalo Igual y en numero de clases ponemos con el minimo en 1 y el maximo en
100. En estilo de color elegimos colores aleatorios. Iremos luego cambiando los
colores uno a uno por un color brillante e identificado a que cobertura
pertenece dicha clase espectral.

Construiremos con ella una tabla como la siguiente

\begin{verbatim}
    id  class
    1   1
    2   1
    3   2
    4   5
    5   7
\end{verbatim}

que guardaremos en un archivo de texto. El mismo lo utilizaremos para realizar
la fusion de clases.

Una vez conocidas las categorias de uso y cobertura correspondientes a cada
clase espectral podemos combinarlas

\begin{lstlisting}
    clases.2016 <- read.delim("class")
    reclas.2016 <- subs(kmeans.2016$map, clases.2016)
\end{lstlisting}

\begin{act}
    Clasifique por el metodo de kmeans la imagen en reflectancia con una
    cantidad de clases espectrales lo suficientemente altas para separar todas
    las clases espectrales.
\end{act}

\begin{act}
    Vuelva a repetir la clasificacion utilizando la imagen obtenida de la
    transformada por componentes principales descartando las bandas que aporten
    menos informacion.
\end{act}

Podemos ahora utilizar la clasificacion para separar zonas de la imagen en el
espacio espectral

\begin{lstlisting}
    ref.2016$kmeans <- reclas.2016
    xyplot(nir~red, groups=kmeans, data=ref.2016)
\end{lstlisting}

\begin{act}
    Grafique en los cortes del espacio espectral la imagen sin fusionar. Compare
    la diferencia entre clases espectrales y clases de informacion.
\end{act}



\chapter{Clasificacion supervisada de imagenes}
\label{educando}
En la sexta clase del curso, continuamos trabajando con algor\'itmos de clasificaci\'on
de im\'agenes, centrandonoces ne este caso en los no supervisados. Son nuestros objetivos:

\begin{itemize}
  \item Poder realizar clasificaciones no supervisadas utilizando los distintos
  algoritmos que se encuentran en R.
  \item Calcular la distancia espectral entre como forma de determinar la separabilidad
  de dos clases espectrales.
  \item Comparar utilizando la entropia de un pixel que coberturas presentan
  mayor confusion al momento de la clasificacion.
\end{itemize}

Cargaremos primero la imagen landsat 8 y habilitaremos la opcion para escribir
el header de ENVI\@. Usaremos en primer lugar los paquetes \texttt{raster},
\texttt{rgdal} y \texttt{RStoolbox}.

\begin{lstlisting}
    rasterOptions(addheader = "ENVI")
    xml.2016 <- readMeta("raster_data/LC82240782016304/LC82240782016304LGN00.xml")
    ref.2016 <- stackMeta(xml.2016, quantity = "sre")
    scaleF <- getMeta(ref.2016,xml.2016, what = "SCALE_FACTOR")
    ref.2016 <- ref.2016 * scaleF
    ref.2016 <- ref.2016[[-1,]]
    names(ref.2016) <- c("blue","green","red","nir","swir1","swir2")
    vector <- readOGR(dsn="vector_data/", layer="entrenamiento")
\end{lstlisting}

\subsection{Clasificador por m\'axima verosimilitud}

Empecemos con la clasificacion por el metodo de maxima verosimilitud, para esto
necesitamos del paquete

\begin{lstlisting}
    sup.2016 <- superClass(ref.2016, vector, responseCol = "MC_ID",
                           model = "mlc")
    plot(sup.2016, col=rainbow(8))
\end{lstlisting}

y realizar el scatterplot de dichas variables como.

\begin{lstlisting}
    ref.mlc <- stack(ref.2016,sup.2016$map)
    xyplot(nir~red, groups=MC_ID , data=ref.mlc)
\end{lstlisting}

Cambiando el algoritmo de clasificacion en el parametro \texttt{model} podemos
calcular distintas clasificaciones supervisadas. Algunas de las vistas en clase
son \texttt{rf}, \texttt{svmRadial}, \texttt{kNN}. Cada una de ellas usa alguna
libreria adicional de las cargadas antes.

\begin{exa}
  Una forma de mejorar las clasificaciones supervisadas basadas en el espacio
  espectral es clasificar por separado distintas clases espectrales y luego unirlas
  en la misma clase de informacion. Veams como hacerlo.
  \begin{lstlisting}
      sup.2016b <- superClass(ref.2016, vector, responseCol = "C_ID",
                             model = "mlc")
  \end{lstlisting}

  Una vez realizada la clasificacion, debemos substituir los valores de cada pixel
  por el de la clase de informacion correspondiente. Para ello hacemos

  \begin{lstlisting}
    subs.2016 = vector@data[c(3,1)]
    sub.2016 <- reclassify(sup.2016b$map, subs.2016)
    writeRaster(sub.2016, "raster_data/processed/mlc2016",
                datatype="INT1U")
    plot(sub.2016, col=rainbow(8))
  \end{lstlisting}


    Podemos finalmente comparar las dos imagenes clasificadas lado a lado ejecuntado el
    comando \verb|plot(stack(sup.2016$map,sub.2016),col=rainbow(8))|
\end{exa}

\begin{act}
    Realice clasificaciones por los distintos metodos y comparelas visualmente.
\end{act}

\begin{act}
    Agregue las bandas de textura y evolucion temporal del NDVI y vuelva a clasificar
    las imagenes.
\end{act}

\subsection{Entropia de la clasificacion}

Para poder comparar en que zonas los clasificadores presentan mas o menos
dispersion podemos calcular la entropia de las distintas clasificaciones en cada
pixel. Para esto utilizaremos la funcion \texttt{rasterEntropy}.

\begin{exa}
  Para esto comenzamos corriendo la clasificacion para distintos modelos, los apilados y
  despues calculamos la entropia de los mismos

  \begin{lstlisting}
      set.seed(42)
      sup.2016 <- superClass(ref.2016, vector, responseCol = "C_ID",
                           model = "mlc")
      mlc.2016 <- reclassify(sup.2016$map, subs.2016)

      library(randomForest)
      sup.2016 <- superClass(ref.2016, vector, responseCol = "C_ID",
                           model = "rf")
      rf.2016 <- reclassify(sup.2016$map, subs.2016)

      library(kernlab)
      sup.2016 <- superClass(ref.2016, vector, responseCol = "C_ID",
                           model = "svmLinear")
      svm.2016 <- reclassify(sup.2016$map, subs.2016)

      prediction_stack <- stack(mlc.2016, rf.2016, svm.2016)
      names(ensemble) <- c("mlc","rf","svm")

      model_entropy <- rasterEntropy(prediction_stack)
  \end{lstlisting}

  Podemos graficar la entropia de las clasificaciones como \verb|model_entropy|
  y ver que zonas presentan mas diferencias a la hora de la clasificacion y cuales no.
  \begin{lstlisting}
    plot(stack(prediction_stack, model_entropy),col=rainbow(8))
  \end{lstlisting}
\end{exa}

\begin{act}
  Repita las clasificaciones por los metodos de arriba agregando la banda
  textura y de variacion temporal del ndvi. Apilela junto con las clasificaciones por
  k-means de la clase anterior. ¿A que cobertura pertenecen las zonas con mayor
  variabilidad?
\end{act}


\chapter{Tecnicas pos-clasificacion}
\label{pos}
Veamos ahora algunas tecnicas de que permiten mejorar las clasificaciones y nos
ayudaran a validar y extraer datos de las imagenes clasificadas.

Comenzamos viendo como aplicar un filtro a una imagen. Comenzamos cargando las
librerias utilizar.

\begin{lstlisting}
    library(RStoolbox)
    library(rgdal)
    library(raster)
    library(rasterVis)
\end{lstlisting}

Para aplicar un filtro a una imagen monobanda, debemos primero definir cual es
la ventana en la que trabajaremos y luego cual es la operacion que desamos
realizr en dicha ventana.

\begin{lstlisting}
    window <- matrix(1,nrow=3, ncol=3)
    clasification.3x3<-focal(clasification.2016,w=window,fun=modal)
\end{lstlisting}

En el caso de un filtro por moda, estaremos dejando el mayor que mas veces
aparezca entre los que rodean al pixel.

\begin{act}
    Aplique filtros de 5x5 y 7x7 para filtrar la imagen. Que problemas
    desaparecen? que dificultad introducen.
\end{act}

\begin{act}
    Aplique el filtro de 3x3 a la imagen correspondiente al año 2000.
\end{act}

\begin{act}
    Utilice la funcion raster sieve de qgis para realizar un filtrado espacial.
    Que diferencias encuentra.
\end{act}

Una vez filtrada la imagen podemos obtener de la misma la matriz de confusion.
Para esto debemos cargar el poligono de validacion y la calculamos con la
funcion \texttt{validateMap}.

\begin{lstlisting}
    valid.2016 <- readOGR(dsn="vector_data/", layer="validacion")
    val.unsup.2016 <- validateMap(sup.2016$map,valData = valid, 
                              responseCol = "MC_ID")
\end{lstlisting}

\begin{act}
    Construya la matriz de confusion y obtenga la presicion global para todas
    las clasificaciones. Que algoritmo funciona mejor con la imagen?
\end{act}

Otra forma de construir incorporar contexto espacial a las clasificaciones es
contruir una capa de textura. Veamos como construir una banda de textura
utilizando la banda pancromatica de Landsat degradada a 30m.

\begin{exa}
    Comenzamos cargando la imagen pancromatica
    \begin{lstlisting}
        pan.2016 <- raster("raster_data/LE72240782000188EDC00/LE72240782000188EDC00_B8.TIF")
    \end{lstlisting}
    Una vez cargada podemos visualizarla como 
    \begin{lstlisting}
        plot(pan.2016)
    \end{lstlisting}
    calculamos ahora el estimador mas sencillo de la textura mediante el calculo
    del desvio standar en una ventana de 3x3
    \begin{lstlisting}
        windows <- matrix(1,nrow=3,ncol=3)
        sd.2016 <- focal(pan.2016,window,fun=sd)
    \end{lstlisting}
    desvio que pondemos graficar como
    \begin{lstlisting}
        plot(sd.2016)
    \end{lstlisting}
    finalmente, degradamos el mapa obtenido a 30m para poder utilizarlo con la
    imagen multiespectral.
    \begin{lstlisting}
        sd.agregate.2016 <- aggregate(sd.2016,fact=2,fun=mean)
    \end{lstlisting}
    finalmente usamos la funcion \texttt{stact} para juntar todas las bandas y
    proceder a la clasificacion.
    \begin{lstlisting}
        context.2016 <- stack{ref.2016,sd.agregate.2016}
        class.2016 <- supClas{}
    \end{lstlisting}
\end{exa}

\newpage

\end{document}
