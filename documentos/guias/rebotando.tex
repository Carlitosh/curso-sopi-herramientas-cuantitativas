
En esta segunda actividad practica nos centraremos en la correccion radiometrica
de imagenes satelitales. Son objetivos de la misma

\begin{itemize}
    \item Poder abrir una imagen satelital desde el metadato.
    \item Convertir los valores de la imagen a reflectancia tope de la
        atmosfera.
    \item Corregir la imagen satelital utilizando los metodos de \emph{dos} y
        \emph{cost}
    \item Corregir la imagen satelital utilizando el \emph{6S web}
\end{itemize}

\subsection{Correccion de imagenes en R}
\label{sub:corr:r}
Para abrir una imagen satelital desde el metadato utilizaremos las funciones
disponibles en \texttt{RStoolbox}. Dicho paquete incluye diversar herramientas
para trabajar con sensores remotos y ya lo utilizamos antes para graficar
imagenes satelitales.

\begin{exa}
   Comencemos analizando un ejemplo sencillo, abriremos una imagen landsat 5
    desde el metadato y la mostraremos en combinacion de bandas de falso color
    compuesto, ademas de analizar las propiedades basicas de la misma.
    \begin{lstlisting}
        library(raster)
        library(RStoolbox)
        meta.1992 <- readMeta("raster_data/LT52300771992104CUB00/LT52300771992104CUB00_MTL.txt")
        dn.1992 <- stackMeta(meta.1992)
        dn.1992 <- dn.1992[[-6,]]
        ggRGB(dn.1992, 
              r=4, g=3, b=2, 
              geom_raster = TRUE, 
              stretch = "lin")
    \end{lstlisting}
    
    Cargaremos de esta forma el metadato de la imagen landat 5 del año 1992,
    abriremos las bandas de la misma y la mostraremos en combinacion de falso
    color compuesto. Analicemos punto por punto que esta pasando.
    \begin{enumerate}
        \item Las lineas 1 y 2 del script leen las librerias que necesitamos
            para trabajar con la imagen.
        \item La linea 3 crea la variable \texttt{meta.1992} con los metadatos
            correspondientes a la imagen de interes.
        \item La linea 4 crea la variable \texttt{dn.1992} con las bandas
            estaqueadas para poder utilizarlas en R. Podemos inspeccionar el
            elemento poniendo su nombre en la consola.
        \item La linea 5 elimina la banda termica de nuestra imagen.
        \item Las lineas 6 a 9 nos permiten mostrar la imagen en combinacion
            falso color compuesto. De ella la linea 6 se refiere a la imagen a
            mostrar, la linea 7 a la combinacion de colores elegida y la 9 al
            tipo de realce aplicado.
    \end{enumerate}
\end{exa}

De esta forma podemos tener el archivo cargado en DN con todos sus metadatos
para convertirlo a reflectancia. Para pasar nuestra imagen a reflectancia a tope
de la atmosfera tenemos dos maneras de hacerlo. Podemos hacerlo a mano
utilizando las herramientas algebraicas de R o podemos hacerlo con la funcion
especifica de \texttt{RStoolbox}.

Veamos ambas. A mano

\begin{lstlisting}
    calref.1992 <- meta.1992$calref
    elev.1992 <- pi*meta.1992$SOLAR_PARAMETERS['elevation']/180
    dn2ref.1992 <- meta.1992$CALREF
    toa.1992 <- (dn.1992*dn2ref.1992$gain+dn2ref.1992$offset)/sin(elev.1992)
    names(toa.1992) <- c("B1_toa","B2_toa","B3_toa","B4_toa","B5_toa","B7_toa")
\end{lstlisting}

de forma automaica

\begin{lstlisting}
    toa.1992b <- radCor(dn.1992, metaData = meta.1992, method = "apref")
\end{lstlisting}

podemos comparar los resultados de ambos metodos inspeccionando los objetos.

\begin{act}
    Inspeccione la reflectancia a tope de la atmosfera para todas las bandas.
    Para esto realice los histogramas, graficos de dispersion, calcule la media,
    el desvio standar y cualquier otra medida estadistica que le guste.
\end{act}

La funcion \texttt{radCor} dispone distintos parametros para hacer distintos
tipos de correcciones atmosfericas. Ya vimos \emph{apref} que nos permitio
calcular la reflectancia a tope de la atmosfera. Veamos como aplicar el metodo
de substraccion de cuerpo obscuro.

\begin{lstlisting}
    haze.1992 <- estimateHaze(dn.1992,darkProp = 0.01, hazeBands = 1:4, plot=TRUE)
    sdos.1992 <- radCor(dn.1992, metaData = meta.1992, 
                 hazeValues = haze.1992,
                 hazeBands = c("B1_dn","B2_dn","B3_dn","B4_dn"), 
                 method="sdos")
\end{lstlisting}

\begin{act}
    Analice los valores de haze obtenidos por la funcion stimate hace y en caso
    de que sea necesario, corrijalos para la banda indicada.
\end{act}

\begin{act}
    Utilice el metodo \emph{costz} para corregir la imagen a reflectancia a tope
    de la superficie.
\end{act}

\begin{act}
    Guarde los archivos raser generado por cada uno de los metodos de
    correccion. Abralos en qgis y comparelos visualmente.    
\end{act}


\subsection{6S}
\label{sub:corr:6S}

Veamos ahora como operar con el 6S para obtener una estimacion de los parametros
atmosfericos. Para esto utilizaremos la version web del 6S que se encuentra
disponible en http://6s.ltdri.org/pages/run6SV.html.

Para utilizarla ingresaremos a la pagina y haremos click en el boton
\menu{Submit query}. Iremos luego configurando paso a paso nuestro modelo de la
atmosfera haciendo siempre luego click en el boton \menu{submit query} para
pasar al paso siguiente.

Los parametros para nuestro modelo son

\begin{enumerate}
    \item Geometrical conditions
        \begin{itemize}
            \item TM (Landsat)
            \item Month: 4, Day:13, GTM decimal hour: 13.60, Longitude:
                -63.8606, Latitude: -24.9937.
        \end{itemize}
    \item Atmospheric Model
        \begin{itemize}
            \item Select Atmospheric Profile: Mid latitude summer
            \item Select aerosol model: Continental Model
            \item Visibility: 60
        \end{itemize}
    \item Target \& sensor altitude
        \begin{itemize}
            \item Select targe altitude: sea level
            \item Select sensor altitude: satellite level
        \end{itemize}
    \item Spectral conditions
        \begin{itemize}
            \item Select spectral conditions: choose band
            \item Select band: 1st band of tm (landat 5)
        \end{itemize}
    \item Ground reflectance
        \begin{itemize}
            \item Ground reflectance type: homogeneous surface
            \item Directional effect: no directional effect
            \item Specify surface reflectance: input constant value of ro
            \item input constant value for ro: 0
        \end{itemize}
    \item Signal
        \begin{itemize}
            \item Atmospheric correction mode: no atmospheric correction
        \end{itemize}
\end{enumerate}

En \menu{7.Results} podemos ver el resultado haciendo click en \emph{Output
file}

Una vez ejecutado el proceso puede usarse el siguiente codigo para corregir
todas las bandas utilizando R.

\begin{lstlisting}
    a <- c(0.98,0.90,...)
    b <- c(0.81,0.90,...)
    g <- c(0.15,0.10,...)
    r <- c(0.08,0.05,...)
    sss.1992 <- (toa.1992/(a*b)-r/b)/(1+g*(toa.1992/(a*b)-r/b))
\end{lstlisting}


\begin{act}
    Realice una extraccion de firmas espectrales para distintass coberturass de
    cada uno de los archivos raster obtenidos y grafiquelos en el mismo grafico.
    Comparela con la firma espectral obtenida a partir de la imagen corregida
    por el usgs.
\end{act}

