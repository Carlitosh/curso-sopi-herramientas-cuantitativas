
En esta segunda actividad practica nos centraremos en la correccion radiometrica
de imagenes satelitales. Son objetivos de la misma

\begin{itemize}
    \item Poder abrir una imagen satelital desde el metadato.
    \item Convertir los valores de la imagen a reflectancia tope de la
        atmosfera.
    \item Corregir la imagen satelital utilizando los metodos de \emph{dos} y
        \emph{cost}
    \item Corregir la imagen satelital utilizando el \emph{6S web}
\end{itemize}

\subsection{Calculo de reflectancia a tope de la atmosfera}

Para poder convertir una imagen a reflectancia a tope de la atmosfera vamos a
necesitar no solo la imagen sino tambien la informacion adicional que hallaremos
en su metadato.

Para abrir una imagen satelital desde el metadato utilizaremos las funciones
disponibles en \texttt{RStoolbox}. Este incluye diversas herramientas
para trabajar con imagenes satelitales.

\begin{exa}
   Comencemos analizando un ejemplo sencillo, abriremos la imagen Landsat 7
    del año 2000 desde el metadato y la mostraremos en combinacion de bandas de falso color
    compuesto, ademas de analizar las propiedades basicas de la misma.
    \begin{lstlisting}
    meta.2000 <- readMeta("raster_data/LE72240782000188EDC00/LE72240782000188EDC00_MTL.txt")
    \end{lstlisting}
    Podemos mostrar las distintas variables incluidas en el objeto usando el
    signo \$ y el nombre de la misma. Por ejemplo \verb|meta.2000$SOLAR_PARAMETERS|
    da como resultado
    \begin{Verbatim}[fontsize=\small]
     azimuth elevation  distance
    37.38251  31.14409   1.01670
    \end{Verbatim}
    A partir del metadato podemos cargar la imagen completa con el comando
    \texttt{stackMeta}. Ademas eliminaremos en este caso las bandas 6 y 7 por
    ser termicas.
    \begin{lstlisting}
    dn.2000 <- stackMeta(meta.2000)
    dn.2000 <- dn.2000[[-6:-7,]]
    dn.2000
    \end{lstlisting}
    obtenemos como resultado un objeto raster stack como el que sigue
    \begin{Verbatim}[fontsize=\small]
    class       : RasterStack
    dimensions  : 2412, 1834, 4423608, 6  (nrow, ncol, ncell, nlayers)
    resolution  : 30.00402, 30.00265  (x, y)
    extent      : 731118.6, 786146, 7101531, 7173897  (xmin, xmax, ymin, ymax)
    coord. ref. : +proj=utm +zone=21 +south +datum=WGS84 +units=m +no_defs
                  +ellps=WGS84 +towgs84=0,0,0
    names       : B1_dn, B2_dn, B3_dn, B4_dn, B5_dn, B7_dn
    min values  :     0,     0,     0,     0,     0,     0
    max values  :   255,   255,   255,   255,   255,   255
    \end{Verbatim}
    y podemos mostrar la iamgen como hicimos antes
    \begin{lstlisting}
    plotRGB(dn.2000, r=3, g=2, b=1, stretch="lin")
    \end{lstlisting}

     \begin{figure}[h!]
     \begin{center}
         \includegraphics[scale=0.6]{dn-l7-rgb}
     \end{center}
     \caption{Imagen en combinacion color real de la zona de interes sobre la
         imagen en DN Landsat 7.}
     \label{fig:dn-l7-rgb}
     \end{figure}
\end{exa}

De esta forma podemos tener el archivo cargado en DN con todos sus metadatos
para convertirlo a reflectancia reflectancia y realizar distintas correcciones.
Para pasar nuestra imagen a reflectancia a tope
de la atmosfera tenemos dos maneras de hacerlo. Podemos hacerlo a mano
utilizando las herramientas algebraicas de R o podemos hacerlo con la funcion
especifica de \texttt{RStoolbox}.

\begin{exa}
    Calculo de reflectancia a tope de la atmosfera
    utilizando el metadato paso por paso

    \begin{lstlisting}
    dn2ref.2000 <- meta.2000$CALREF[1:6,]
    elev.2000 <- pi*meta.2000$SOLAR_PARAMETERS['elevation']/180
    \end{lstlisting}
    extraemos primero del metadatos los parametros de calibracion en
    reflectancia y el angulo de elevacion solar.

    Convertimos luego la imagen a reflectancia y la dividimos luego por el
    angulo solar. Luego cambiamos los nombres de las bandas

    \begin{lstlisting}
    toam.2000 <- (dn.2000*dn2ref.2000$gain+dn2ref.2000$offset)/sin(elev.2000)
    names(toam.2000) <- c("blue","green","red","nir","swir1","swir2")
    \end{lstlisting}

    Otra forma forma de realizar este proceso es utilizando la funcion
    \texttt{radCor}. En este caso debemos dar la imagen en DN, el metadato y
    cual es la cantidad que queremos calcular.

    \begin{lstlisting}
    toa.2000 <- radCor(dn.2000, metaData = meta.2000, method = "apref")
    \end{lstlisting}

    podemos comparar los resultados de ambos metodos inspeccionando los objetos
    \texttt{toam.2000} y \texttt{toa.2000}.
    \begin{Verbatim}[fontsize=\small]
    class       : RasterBrick
    dimensions  : 2412, 1834, 4423608, 6  (nrow, ncol, ncell, nlayers)
    resolution  : 30.00402, 30.00265  (x, y)
    extent      : 731118.6, 786146, 7101531, 7173897  (xmin, xmax, ymin, ymax)
    coord. ref. : +proj=utm +zone=21 +south +datum=WGS84 +units=m +no_defs
                  +ellps=WGS84 +towgs84=0,0,0
    data source : in memory
    names       :        blue,       green,         red,         nir,    swir1,       swir2
    min values  : -0.01976113, -0.02181530, -0.02029439,  0.01934678,    -0.02781926, -0.02678077
    max values  :   0.6106812,   0.5609009,   0.6079443,   0.8696885,    0.8640919,   0.8263815
    \end{Verbatim}

    y

    \begin{Verbatim}[fontsize=\small]
    class       : RasterStack
    dimensions  : 2412, 1834, 4423608, 6  (nrow, ncol, ncell, nlayers)
    resolution  : 30.00402, 30.00265  (x, y)
    extent      : 731118.6, 786146, 7101531, 7173897  (xmin, xmax, ymin, ymax)
    coord. ref. : +proj=utm +zone=21 +south +datum=WGS84 +units=m +no_defs
                  +ellps=WGS84 +towgs84=0,0,0
    names       :     B1_tre,     B2_tre,     B3_tre,     B4_tre,     B5_tre,    B7_tre
    min values  : 0.00000000, 0.00000000, 0.00000000, 0.01934678, 0.00000000,    0.00000000
    max values  :  0.6106812,  0.5609009,  0.6079443,  0.8696885,  0.8640919,    0.8263815
    \end{Verbatim}

    \end{exa}

\begin{act}
    Inspeccione la reflectancia a tope de la atmosfera para todas las bandas.
    Para esto realice los histogramas, graficos de dispersion, calcule la media,
    el desvio standar y cualquier otra medida estadistica que le guste.
\end{act}
\subsection{Calculo de reflectancia corregida atmosfericamente por metodos
            estadisticos}

La funcion \texttt{radCor} dispone de un parametro para hacer distintas
tipos de correcciones atmosfericas. Ya vimos \emph{apref} que nos permitio
calcular la reflectancia a tope de la atmosfera. Veamos como aplicar el metodo
de substraccion de cuerpo obscuro.

\begin{exa}
    Apliquemos el metodo de \emph{simple dos} para corregir la imagen. En este caso
    solamente restaremos el minimo en cada banda a la imagen para las bandas
    donde existe haze, es decir en la zona del visible y del infrarrojo cercano.

    Estimamos el haze primero y corregimos la imagen luego haciendo
    \begin{lstlisting}
    haze.2000 <- estimateHaze(dn.2000,darkProp = 0.01, hazeBands = 1:4, plot=TRUE)
    sdos.2000 <- radCor(dn.2000, metaData = meta.2000,
                 hazeValues = haze.2000,
                 hazeBands = c("B1_dn","B2_dn","B3_dn","B4_dn"),
                 method="sdos")
    \end{lstlisting}
    en este caso los valores de haze estimados son
    \begin{Verbatim}[fontsize=\small]
    B1_dn B2_dn B3_dn B4_dn
       41    27    20    15
    \end{Verbatim}
    Para hacer un analisis de lo que pasa en la situacion, vamos a graficar los
    histogramas de cada banda para la imagen en reflectancia TOA y corregida por
    el metodo simple dos. Para esto usaremos el paquete \texttt{rasterVis}
    \begin{lstlisting}
    B1 <- densityplot(~B1_tre+B1_sre, data=toa.boa, xlab="Reflectancia",
                      ylab="", main="Banda azul", plot.points=FALSE, xlim=c(0,0.3),
                      key=simpleKey(text=c("Tope de la atmosfera",
                                           "Correccion Simple DOS"),
                                           lines=TRUE, points=FALSE))
    B2 <- densityplot(~B2_tre+B2_sre, data=toa.boa, xlab="Reflectancia",
                      ylab="", main="Banda verde", plot.points=FALSE, xlim=c(0,0.3),
                      key=simpleKey(text=c("Tope de la atmosfera",
                                           "Correccion Simple DOS"),
                                           lines=TRUE, points=FALSE))
    B3 <- densityplot(~B3_tre+B3_sre, data=toa.boa, xlab="Reflectancia",
                      ylab="", main="Banda roja", plot.points=FALSE, xlim=c(0,0.3),
                      key=simpleKey(text=c("Tope de la atmosfera",
                                           "Correccion Simple DOS"),
                                           lines=TRUE, points=FALSE))
    B4 <- densityplot(~B4_tre+B4_sre, data=toa.boa, xlab="Reflectancia",
                      ylab="", main="Banda nir", plot.points=FALSE, xlim=c(0,0.3),
                      key=simpleKey(text=c("Tope de la atmosfera",
                                           "Correccion Simple DOS"),
                                           lines=TRUE, points=FALSE))
     print(B1,split = c(1, 1, 2, 2),more=TRUE)
     print(B2,split = c(2, 1, 2, 2),more=TRUE)
     print(B3,split = c(1, 2, 2, 2),more=TRUE)
     print(B4,split = c(2, 2, 2, 2),more=FALSE)
    \end{lstlisting}
    En este caso las primeras 4 funciones crean los histogramas para cada banda
    corregida mientras que las ultimas 4 lineas los imprimen en una grilla.
    \begin{figure}[h!]
    \begin{center}
        \includegraphics[scale=0.4]{simpledos.png}
    \end{center}
    \caption{Graficos de los histogramas para las distintas bandas donde se
        muestra el nivel de correccion en cada una.}
    \label{fig:simpledos.png}
    Notamos en este caso que la correccion se vuelve menos importante a medida
        que crece la longitud de onda.
    \end{figure}

\end{exa}
\begin{act}
    Analice los valores de haze obtenidos por la funcion stimate haze y grafiquelos
    como funci\'on de la longitud de onda en escala logaritmica. ¿Que observa?
\end{act}

\begin{act}
    Utilice el metodo \emph{costz} para corregir la imagen a reflectancia a tope
    de la superficie.
\end{act}

\begin{act}
    Guarde los archivos raser generado por cada uno de los metodos de
    correccion. Abralos en qgis y comparelos visualmente. Obtenga firmas
    espectrales con los distintos metodos de correccion.
\end{act}


\subsection{6S}
\label{sub:corr:6S}

Veamos ahora como operar con el 6S para obtener una estimacion de los parametros
atmosfericos. Para esto utilizaremos la version web del 6S que se encuentra
disponible en http://6s.ltdri.org/pages/run6SV.html.

Para utilizarla ingresaremos a la pagina y haremos click en el boton
\menu{Submit query}. Iremos luego configurando paso a paso nuestro modelo de la
atmosfera haciendo siempre luego click en el boton \menu{submit query} para
pasar al paso siguiente.

Los parametros para nuestro modelo son

\begin{enumerate}
    \item Geometrical conditions
        \begin{itemize}
            \item TM (Landsat)
            \item Month: 4, Day:13, GTM decimal hour: 13.60, Longitude:
                -63.8606, Latitude: -24.9937.
        \end{itemize}
    \item Atmospheric Model
        \begin{itemize}
            \item Select Atmospheric Profile: Mid latitude summer
            \item Select aerosol model: Continental Model
            \item Visibility: 60
        \end{itemize}
    \item Target \& sensor altitude
        \begin{itemize}
            \item Select targe altitude: sea level
            \item Select sensor altitude: satellite level
        \end{itemize}
    \item Spectral conditions
        \begin{itemize}
            \item Select spectral conditions: choose band
            \item Select band: 1st band of tm (landat 5)
        \end{itemize}
    \item Ground reflectance
        \begin{itemize}
            \item Ground reflectance type: homogeneous surface
            \item Directional effect: no directional effect
            \item Specify surface reflectance: input constant value of ro
            \item input constant value for ro: 0
        \end{itemize}
    \item Signal
        \begin{itemize}
            \item Atmospheric correction mode: no atmospheric correction
        \end{itemize}
\end{enumerate}

Todos estos valores los encontramos en el metadato de la imagen.

En \menu{7.Results} podemos ver el resultado haciendo click en \emph{Output
file}. Del mismo debemos extraer los valores de
 \begin{itemize}
     \item \texttt{global gas. trans. - total}
     \item \texttt{total sca. trans. - total}
     \item \texttt{spherical albedo - total}
     \item \texttt{reflectance I - total}
 \end{itemize}

Una vez ejecutado el proceso puede usarse el siguiente codigo para corregir
todas las bandas utilizando R.

\begin{lstlisting}
    a <- c(0.98,0.90,...) # Global gas transmitance
    b <- c(0.81,0.90,...) # Total scatering transmitance
    g <- c(0.15,0.10,...) # Spherical albedo
    r <- c(0.08,0.05,...) # Reflectance I
    sss.2000 <- (toa.2000/(a*b)-r/b)/(1+g*(toa.2000/(a*b)-r/b))
\end{lstlisting}


\begin{act}
    Realice una extraccion de firmas espectrales para distintass coberturass de
    cada uno de los archivos raster obtenidos y grafiquelos en el mismo grafico.
    Comparela con la firma espectral obtenida a partir de la imagen corregida
    por el usgs.
\end{act}

\begin{act}
    Haga un grafico de densidades que muestre los distintos metodos de
    correccion atmosfericos para cada banda.
\end{act}

\begin{act}
    Calcule la diferencia promedio para cada banda entre las imagenes en
    reflectancia a tope de la atmosfera y las distintas correcciones
    y la imagen en reflectancia entregada por el USGS.
\end{act}
