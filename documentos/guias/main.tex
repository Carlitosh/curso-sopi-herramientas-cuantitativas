\documentclass[hidelinks,12pt]{article}

% Preambulo por defecto
% Paquetes para usar bien el idioma español
\usepackage[spanish,es-tabla]{babel}
\selectlanguage{spanish}
\usepackage[utf8]{inputenc}

% Paquetes para usar mejores imagenes
\usepackage{graphicx}

% Paquetes para links y tabla de contenidos en el PDF
\usepackage{hyperref}
\hypersetup{colorlinks=true,allcolors=blue}
%\usepackage{hypcap}

% Paquetes para mejores tablas
\usepackage{booktabs}

% Mejor matematica
\usepackage{amsmath}

% Fuentes de las imagenes
\usepackage[absolute,overlay]{textpos}

% Paquete captions
\usepackage[justification=centering,labelformat=empty,labelsep=none]{caption}

% Opciones para ticks
\usepackage{tikz}
\usetikzlibrary{shapes,arrows,positioning}

\tikzstyle{decision} = [diamond, draw, fill=blue!20, text width=4em, text badly centered, node distance=2cm, inner sep=0pt,on grid]
\tikzstyle{block} = [rectangle, draw, fill=blue!20, text width=8em, text centered, rounded corners, minimum height=2em,on grid]
\tikzstyle{line} = [draw, -latex]

% Citas bibliograficas
\usepackage[backend=biber]{biblatex}
\renewcommand{\footnotesize}{\tiny}
\addbibresource{biblio.bib}

% Mejoro las captions
\setbeamertemplate{caption}{\raggedright\insertcaption\par}

\setbeamertemplate{caption}{%
\begin{beamercolorbox}[wd=0.85\paperwidth, sep=.2ex]{block body}\insertcaption%
\end{beamercolorbox}%
}


% Sacar barra de navegacion
\setbeamertemplate{navigation symbols}{}%remove navigation symbols

% Transparencias en items
\setbeamercovered{transparent}

% Estilo de diapositivas
% \usetheme{Boadilla}
\usecolortheme{whale}
\usecolortheme{orchid}


\definecolor{A11}{HTML}{B2DF8A}
\definecolor{A12}{HTML}{33A02C}
\definecolor{A23}{HTML}{FDBF6F}
\definecolor{A24}{HTML}{FF7F00}
\definecolor{B15}{HTML}{FB9A99}
\definecolor{B16}{HTML}{E31A1C}
\definecolor{B27}{HTML}{A6CEE3}
\definecolor{B28}{HTML}{1F78B4}

\title{SoPI II \- Herramientas de Teledetecci\'on Cuantitativa \\
\emph{Gu\'{\i}a de actividades: Uso del suelo en el departamento de Iguaz\'u,
provincia de Misiones}}
\author{Francisco
Nemi\~na\thanks{\href{mailto:fnemina@conae.govar}{fnemina@conae.gov.ar}}}
\affil{Unidad de Educaci\'on y Formaci\'on Masiva\\
    Comisi\'on Nacional de Actividades Espaciales}
\date{\today}

\begin{document}

\maketitle

\section*{Introducci\'on}

La utilización de imágenes satelitales permite analizar grandes extensiones del
territorio, contando con un registro histórico con el cual realizar
comparaciones.

En la provincia de Misiones, el departamento de Iguaz\'u es lindante a Brasil y
Paraguay siendo parte de la zona conocida como triple frontera
perteneciente a la ecorregi\'on conocida como \emph{selva paranaense}. Dentro
del mismo podemos encontrar la Represa de Urugua-\'{\i} y el Parque Nacional
Iguaz\'u.

Tomaremos entonces al departamento como \'area de estudio durante este curso
con el objetivo de obtener un mapa de uso y cobertura dentro del mismo que nos
permita estimar y validar  las \'areas correspondients a los mismos.

Utilizaremos para esto imagenes satelitales de los satelites Landsat 8, SPOT-5 y
el producto de MOD13Q1 obtenido de los satelites TERRA y AQUA obtenidas durante
el periodo que va de agosto de 2013 a agosto de 2014.

\newpage
\section{Transformaciones en el dominio espectral}
\subsection{Análisis de Firmas Espectrales}
Es objetivo de esta pr\'actica es familiarizarse con la zona de inter\'es,
estudiar y caracterizar el comportamiento espectral de distintas las
categor\'{\i}as de uso y cobertura y comprender como se relacionan las mismas
con los valores obtenidos a partir de im\'agenes  satelitales.

\begin{enumerate}
    \item Abra la imagen \texttt{l8\_oli\_20140211.tif}. Realice distintas
        combinaciones de banda y seleccione aquella que le permita distinguir
        mas detalles de la vegetación.
    \item Encuentre dentro de la escena parches de coberturas uniformes y
        digitalícelos utilizando la herramienta de edición vectorial. Tome
        ejemplos de distintas coberturas correspondientes a los casos A11, A12,
        B15, B27 y B28 de la tabla~\ref{tabla1} del ap\'endice~\ref{apcate}
    \item Utilizando la herramienta de extracción de estadísticas en \'area de
        clasificaci\'on calcule la media correspondiente a cada cobertura analizada
        en el punto 2.
    \item Grafique la media obtenida en el punto 3
\end{enumerate}
\subsection{Corrección Radiométrica de Imágenes satelitales}
Es objetivo de esta practica es conocer como afecta la interacción entre la luz
y la atmósfera a la radiometr\'ia de una imagen satelital y la respuesta
espectral de los distintos usos y coberturas  y estudiar distintos métodos empíricos
y estadísticos para corregirla.
\begin{enumerate}
    \item Convierta la imagen  \texttt{l1\_l8\_oli\_20130819.tif} a
    reflectancia a tope de la atmósfera utilizando los parámetros de
    calibración que se encuentran dentro del metadato de la misma y la
    corrección por el ángulo solar.
    \item Grafique el histograma para cada banda de la imagen anterior y
     utilícelos  para corregirla por el método de substracción de cuerpo
     obscuro (DOS1).
    \item Utilice la web de 6S\footnote{http://6s.ltdri.org/} para ejecutarlo
    según los parámetros geométricos de la imagen. Utilice estos resultados para
    corregir la imagen.
    \item Utilice los vectores obtenidos en la clase anterior y realice las
    firmas espectrales de cada clase de uso y cobertura para las distintas
    correcciones.
\end{enumerate}

\subsection{Calculo de \'{\i}ndices espectrales}
El objetivo de esta práctica generar e interpretar \'{\i}ndices espectrales
a partir de  imágenes satelitales y sus distintos usos como un caso
particular de reducci\'on de la dimensionalidad.

\begin{enumerate}
    \item Abra las imagenes \texttt{l8\_oli\_20130819.tif} y
        \texttt{l8\_oli\_20140211.tif} y calcule los indices NDVI y EVI
        para ambas.
    \item Estime la pendiente de la linea de suelo a partir de la
    interpretaci\'on de un scatter-plot de las bandas del infrarrojo cercano
    y rojo para ambas imágenes y utilicelas para calcular el \'{\i}ndice
    TSAVI\@.
    \item Apile cada par de  índices y visualice la imagen en la combinación
    que muestre ambos en simultáneo.
    \item Para cada uno de los índices ajuste su valor a los datos medidos a
    campo y estime el el LAI a partir de ellos.
\end{enumerate}

\subsection{Rotaciones y transformaciones \label{rot:pra}}
El objetivo de esta práctica es  profundizar los conceptos de reducción
de dimensionalidad en el trabajo con imágenes satelitales a través de la
utilizacion de rotaciones y transformaciones espectrales.

\begin{enumerate}
    \item Con la imagen \texttt{l8\_oli\_2014\-02\-11.tif} calcule la
        transformada por componentes principales y la transformada tasseled-cap.
    \item \label{pca} Apile las imagenes
        \texttt{l8\_oli\_2013\-08\-19.tif}
        y \texttt{l8\_oli\_2014\-02\-11.tif}.
        Aplique la transformaci\'on por componentes principales.
        Analice las distintas compotentes.
    \item Abra la imagen \texttt{mod13q1\_ndvi\_2013\-07\-27\_2014\-08\-28.tif}
        y grafique su variación temporal para distintas coberturas.
    \item Utilizando la herramienta análisis por componentes principales
        encuentre la rotación que diagonaliza la matriz de correlación para la
        imagen \texttt{mod13q1\_ndvi\_2013\-07\-27\_2014\-08\-28.tif}. Analice
        por componentes principales y diga que información puede distinguir en
        la misma.
\end{enumerate}
\newpage
\section{Clasificación de imágenes en la práctica}
\subsection{Métodos supervisados de clasificación}
El objetivo de esta práctica es comenzar a realizar clasificaciones sobre las imágenes, utilizando métodos de clasificación supervisada para obtener
mapas de uso y cobertura con sus respectivas áreas.

\begin{enumerate}
    \item Digitalice un parche homogeneo  para las categorias A11, A12, B15, B27, B28 de la
    tabla~\ref{tabla1} del ap\'endice~\ref{apcate} creando una capa vectorial para
    cada tipo de cobertura. Grafique la firma espectral y el desvío de cada una.
    \item Clasifique la imagen \texttt{l8\_oli\_2014\-02\-11.tif} utilizando las
    áreas de entrenamiento creadas en el punto anterior. Con la herramienta
    de estadísticas globales, encuentre el área correspondiente a cada tipo de uso y
    cobertura.
    \item Cargue las capas vectoriales de la carpeta \texttt{entrenamiento} y
        vuelva a clasificar la imagen \texttt{l8\_oli\_2014\-02\-11.tif}.
    \item Fusione la imagen en las clases de uso y cobertura deseada, y utilice la
    imagen obtenida para calcular nuevamente el área correspondiente a cada uso y
    cobertura del suelo.
\end{enumerate}

\subsection{Métodos no supervisados de clasificación}
El objetivo de esta práctica es continuar estudiando la clasificacion de imágenes, utilizando métodos de clasificación no supervisada para obtener
mapas de uso y cobertura con sus respectivas áreas.

\begin{enumerate}
    \item Clasifique la imagen \texttt{l8\_oli\_2014\-02\-11.tif} por el método
        k-means, asignando un número total
        de 5 clases. Analice y asigne a estas clases las categorias A11, A12,
        B15, B27, B28 de la tabla~\ref{tabla1} del ap\'endice~\ref{apcate}.
    \item Clasifique la imagen \texttt{l8\_oli\_2014\-02\-11.tif} por el método de
        k-means, pero ahora utilice 50
        clases. Analice y asigne a estas clases las categorias A11, A12,
        B15, B27, B28 de la tabla~\ref{tabla1} del ap\'endic~\ref{apcate}.
    \item Clasifique las primeras 6 la bandas de imagen obtenida en el punto~\ref{pca}
        de la secci\'on~\ref{rot:pra} utilizando el m\'etodo k-means
        asignando un total de 50 clases. Analice y asigne a estas clases las categorias A11, A12,
        B15, B27, B28 de la tabla~\ref{tabla1} del ap\'endice~\ref{apcate}.
    \item Utilice las herramientas de calcular estadísticas globales para
        estimar las áreas correspondientes a cada tipo de uso y cobertura en
        las tres clasificaciones.
\end{enumerate}

\subsection{Validación de clasificaciones}
El objetivo de esta práctica es analizar la precisión de las clasificaciones
realizadas en clases anteriores haciendo hincapié en la importancia del muestreo
y los problemas que pueden presentarse.

\begin{enumerate}
    \item Aplique un filtro por mayoría a las clasificaciones obtenidas en las
        clases anteriores
    \item Cargue los polígonos de la carpeta \texttt{validacion} correspondientes a
        cada clase de uso y cobertura del suelo.
    \item Realice la matriz de confusión para cada una de las clasificaciones.
    \item A partir de las matrices obtenidas calcule la precisión global,
        precisiones del usuario el productor y el índice kappa para cada una de
        ellas. Utilizando ademas los datos de \'area de cada imagen, obtenga las
        areas y errores correspondientes a cada categoria de uso y cobertura.
\end{enumerate}
\newpage
\appendix

\section{Categorias de uso y cobertura del suelo}\label{apcate}
Categorias de uso y cobertuar segun el esquema LCCS2 de la FAO\@. Los colores son
sugerencias por categoria.
\begin{table}[hbt]
    \centering
    \begin{tabular}{p{11cm}cc}
        \toprule
        Nombre & Codigo & Color \\
        \midrule
        Áreas terrestres cultivadas y manejada & A11 & \textcolor{A11}{$\blacksquare$}\texttt{\#b2df8a}
        \\
        Vegetación natural y semi-natural & A12 & \textcolor{A12}{$\blacksquare$}\texttt{\#33a02c}\\
        Áreas acuáticas o regularmente inundadas cultivadas & A23  &
        \textcolor{A23}{$\blacksquare$}\texttt{\#fdbf6f}\\
        Vegetación natural y semi-natural acuática o
	regularmente inundadas & A24 & \textcolor{A24}{$\blacksquare$}\texttt{\#ff7f00}\\
        Superficies artificiales y áreas asociadas & B15  &
        \textcolor{B15}{$\blacksquare$}\texttt{\#fb9a99}\\
        Áreas descubiertas o desnudas & B16 & \textcolor{B16}{$\blacksquare$}\texttt{\#e31a1c}\\
        Cuerpos artificiales de agua, nieve y hielo & B27 &
        \textcolor{B27}{$\blacksquare$}\texttt{\#a6cee3}\\
        Cuerpos naturales de agua, nieve y hielo & B28&
        \textcolor{B28}{$\blacksquare$}\texttt{\#1f78b4}\\
        \bottomrule
    \end{tabular}
\caption{\label{tabla1}Categorias usos del suelo segun el esquema LCCS2 de la
FAO.}
\end{table}

\section{Transformada tasseled-cap}
Transformada tasseled cap para Landsat 8-OLI.

\begin{table}[htb]
    \centering
    \begin{tabular}{@{}lr@{}lr@{}lr@{}lr@{}lr@{}lr@{}l@{}}
        \toprule
                   & \multicolumn{2}{c}{(Azul)}  & \multicolumn{2}{c}{(Verde)} &
        \multicolumn{2}{c}{(Rojo)}  & \multicolumn{2}{c}{(NIR)}   &
        \multicolumn{2}{c}{(SWIR1)} & \multicolumn{2}{c}{(SWIR2)} \\
                   & \multicolumn{2}{c}{Banda 2} & \multicolumn{2}{c}{Banda 3} &
        \multicolumn{2}{c}{Banda 4} & \multicolumn{2}{c}{Banda 5} &
        \multicolumn{2}{c}{Banda 6} & \multicolumn{2}{c}{Banda 7} \\
        \midrule
        Brightness & 0,           & 3029         & 0,           & 2786         &
        0,           & 4733         & 0,           & 5599         & 0,
        & 5080         & 0,           & 1872         \\
        Greenness  & -0,          & 2941         & -0,          & 2430         &
        -0,          & 5424         & 0,           & 7276         & 0,
        & 0713         & -0,          & 1608         \\
        Wetness    & 0,           & 1511         & 0,           & 1973         &
        0,           & 3283         & 0,           & 3407         & -0,
        & 7117         & -0,          & 4559         \\
        TCT4       & -0,          & 8239         & 0,           & 0849         &
        0,           & 4396         & -0,          & 0580         & 0,
        & 2013         & -0,          & 2773         \\
        TCT5       & -0,          & 3294         & 0,           & 0557         &
        0,           & 1056         & 0,           & 1855         & -0,
        & 4349         & 0,           & 8085         \\
        TCT6       & 0,           & 1079         & -0,          & 9023         &
        0,           & 4119         & 0,           & 0575         & 0,
        & 0259         & 0,           & 0252         \\ \bottomrule
    \end{tabular}
    \caption{Transformada tasseled-cap para landsat 8.}
\end{table}
\pagebreak
\section{Ecuaciones}
Ecuaciones utiles para el curso. Todas las magnitudes entre 0 y 1.

\begin{table}[hbt]
    \centering
    \begin{tabular}{lcl}
        \toprule
        Nombre & Ecuaci\'on & Observaciones\\
        \midrule
        Reflectancia TOA & $\rho'_\lambda =
        (G_\lambda*DN+B_\lambda)/\sin(\theta_e)$ &
        $G_\lambda$ : ganancia\\
        & & $B_\lambda$ : desvio\\
        & & $\theta_e$ : angulo de elevaci\'on\\
        & & \\
        Corrección por DOS1 & $\rho^*_\lambda = \rho_\lambda -
        \rho_{\lambda,min}$ & \\
        & & \\
        Corrección por 6S & $\rho_\lambda^* = [A\rho+B] / [1+\gamma (A\rho+B)]$& $A = 1/\alpha\beta$\\
        & & $B = -\rho_I/\beta$\\
        & & $\alpha$: global gas transmitance\\
        & & $\beta$: total scattering transmittance\\
        & & $\gamma$: spherical albedo\\
        & & $\rho_I$: reflectance I\\
        & & \\
        NDVI & $\frac{\rho_{nir} -
    \rho_{rojo}}{\rho_{nir}+\rho_{rojo}}$ & \\
            & & \\
        EVI & $G\frac{\rho_{nir} -
    \rho_{rojo}}{\rho_{nir}+C_1\rho_{rojo}-C_2\rho_{azul}+L}$ &
        $G=2.5$, $L=1.0$ \\
        & & $C_1=6.0$, $C_2 = 7.5$\\
                & & \\
        SAVI & $(1+L)\frac{\rho_{nir} -
    \rho_{rojo}}{\rho_{nir}+\rho_{rojo}+L}$ & $L = 0.5$\\
                    & & \\
        TSAVI & $m\frac{\rho_{nir} -
        m\rho_{rojo}-b}{m\rho_{nir}+\rho_{rojo}-mb}\times10000$ & $m$ :
        Pendiente \\
        & & $b$ : Ordenada al origen\\
        & & de la linea de suelo.\\
                & & \\
        \bottomrule
    \end{tabular}
    \caption{Ecuaciones escaladas para utilizar con tipo de dato entero de 16
    bits}
\end{table}
\pagebreak
\section{Datos sobre las im\'agenes}
Datos espectrales sobre las im\'agenes utilizadas.
\begin{table}[htb]
    \centering
    \begin{tabular}{@{}llcclcc@{}}
        \toprule
        \multicolumn{1}{c}{Satelite} & \multicolumn{1}{c}{Sensor} &
        \multicolumn{2}{c}{Banda} & \multicolumn{1}{c}{Designacion} & $\lambda$ &
        \multicolumn{1}{l}{$\Delta \lambda$} \\
        \multicolumn{1}{c}{}         & \multicolumn{1}{c}{}       & Satelite
        & Imagen     & \multicolumn{1}{c}{}            & [nm]    &
        \multicolumn{1}{l}{[nm]}           \\ \midrule
        Landsat 8                    & OLI                        & 2
        & 1          & Azul                            & 482     & 60
        \\
        Landsat 8                    & OLI                        & 3
        & 2          & Verde                           & 561     & 57
        \\
        Landsat 8                    & OLI                        & 4
        & 3          & Rojo                            & 655     & 37
        \\
        Landsat 8                    & OLI                        & 5
        & 4          & NIR                             & 864     & 28
        \\
        Landsat 8                    & OLI                        & 6
        & 5          & SWIR 1                          & 1608    & 84
        \\
        Landsat 8                    & OLI                        & 7
        & 6          & SWIR 2                          & 2200    & 187
        \\
        SPOT 5                       & HRG-2                      & 3
        & 1          & NIR                             & 840     & 100
        \\
        SPOT 5                       & HRG-2                      & 2
        & 2          & Rojo                            & 645     & 70
        \\
        SPOT 5                       & HRG-2                      & 1
        & 3          & Verde                           & 545     & 90
        \\
        SPOT 5                       & HRG-2                      & 4
        & 4          & SWIR                            & 1665    & 170
        \\ \bottomrule
    \end{tabular}
    \caption{Datos espectrales sobre las im\'agenes utilizadas.}
    \label{my-label}
\end{table}

%\printbibliography\
\end{document}
