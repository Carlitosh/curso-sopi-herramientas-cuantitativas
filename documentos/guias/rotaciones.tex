Durante la clase de hoy trabajaremos con rotaciones en el espacio espectral. A
diferencia del trabajo con indices las rotaciones pueden interpretarse no como
algebra entre las bandas sino como distintas formas de mirar al mismo espacio
espectral.

En este caso usaremos las librerias \texttt{raster} y \texttt{RStoolbox}.

\begin{lstlisting}
    library(raster)
    library(RStoolbox)
    library(bfastSpatial)
\end{lstlisting}


\begin{exa}
    Comencemos analizando la transformada por componentes principales de la
    imagen de 2016. Que podemos predecir?
    \begin{lstlisting}
        pairs(ref.2016)
    \end{lstlisting}

    Mirando el resumen de la imagen vemos que hay varias bandas muy
    correlacionadas entre si. Por ejemplo las del visible, mientras que otras lo
    estan poco, por el ejeplo el nir y el swir. Por lo tanto esperamos que no
    todas las bandas sean necesarias para explicar el comportamiento de la
    imagen


    \begin{lstlisting}
        pca.2016 <- rasterPCA(ref.2016)
        summary(pca.2016$model)
        loadings(pcs.2016$model)
        plot(pca.2016$map)
    \end{lstlisting}

\end{exa}

\begin{act}
    Calcule y analice la transformada por PCA de la imagen Landsat 7 del año
    2000.
\end{act}

\begin{exa}
    Otra aplicacion de la transformada por componentes principales por
    componentes principales. Veamos como realizarlo.
    \begin{lstlisting}
        ndvi.list <- list.files("raster_data/MOD13Q1/EVI/", pattern = "*.tif$",
                                 full.names = TRUE)
        ndvi.stack <- stack(ndvi.list)
    \end{lstlisting}
    una vez abierta la imagen la convertimos a valores entre -1 y 1 e
    interpolamos los valores que falten.
    \begin{lstlisting}
        ndvi.stack <- ndvi.stak/1e4
        ndvi.stack <- approxNA(ndvi.stack)
    \end{lstlisting}
    Una vez llenados los espacios donde no habia datos podemos aplicar la
    transformada por componentes principales y mostrarla
    \begin{lstlisting}
        ndvi.pca <- rasterPCA(ndvi.stack)
    \end{lstlisting}
\end{exa}

\begin{act}
    Grafique las primeras 4 componentes por de la transformada por componentes
    princiales de la imagen del stack de NDVI\@. Que zonas puede identificar en la
    primera? que zonas se distinguen en la segunda? que comportamiento encuentra
    en la tercera y cuarta.
\end{act}

\begin{act}
    Investigue la funcion \texttt{tasseledCap} y calcule la transformada
    tasseled cap para las imagenes landsat 7 y 8.
\end{act}

\begin{act}
    Grafique en el scatter-plot la imagen completa y marque en el mismo zonas
    con vegetacion, agua y suelo sin cobertura vegetal. Vea como cambian estas
    zonas frente a las transformadas por componentes principales y tasseled cap.
\end{act}

