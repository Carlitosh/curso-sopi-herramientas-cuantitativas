En esta clase vamos a trabajar con clasificaciones no supervisadas de imagenes
satelitales. Vamos a usar los paquetes

\begin{lstlisting}
    library(raster)
    library(RStoolbox)
\end{lstlisting}

Cargaremos primero la imagen landsat 8 y habilitaremos la opcion para escribir
el header de ENVI\@.

\begin{lstlisting}
    rasterOptions(addheader = "ENVI")
    set.seed(6)
    kmeans.2016 <- unsuperClass(ref.2016, nClasses = 5, nStarts = 100,
                                nSamples = 100)
    writeRaster(kmeans.2016, "raster_data/processed/kmeans2016",
                datatype="INT1U")
\end{lstlisting}

Podemos ahora graficar por separado cada una de las clases

\begin{lstlisting}
    classes.2016 <- layerize(kmeans.2016)
    plot(classes.2016)
\end{lstlisting}

Abriremos la imagen ahora en el qgis e identificaremos cada una de las clases
realiando interpretacion visual de la imagen. 

Para realizar la identificacion primero vamos al menu \menu{propiedades de la
imagen, Estilo, Tipo de renderizacion, Unibanda pseudocolor}. Elegimos de modo
Intervalo Igual y en numero de clases ponemos con el minimo en 1 y el maximo en
100. En estilo de color elegimos colores aleatorios. Iremos luego cambiando los
colores uno a uno por un color brillante e identificado a que cobertura
pertenece dicha clase espectral.

Construiremos con ella una tabla como la siguiente

\begin{verbatim}
    id  class
    1   1
    2   1
    3   2
    4   5
    5   7
\end{verbatim}

que guardaremos en un archivo de texto. El mismo lo utilizaremos para realizar
la fusion de clases.

Una vez conocidas las categorias de uso y cobertura correspondientes a cada
clase espectral podemos combinarlas

\begin{lstlisting}
    clases.2016 <- read.delim("class")
    reclas.2016 <- subs(kmeans.2016$map, clases.2016)
\end{lstlisting}

\begin{act}
    Clasifique por el metodo de kmeans la imagen en reflectancia con una
    cantidad de clases espectrales lo suficientemente altas para separar todas
    las clases espectrales.
\end{act}

\begin{act}
    Vuelva a repetir la clasificacion utilizando la imagen obtenida de la
    transformada por componentes principales descartando las bandas que aporten
    menos informacion.
\end{act}

Podemos ahora utilizar la clasificacion para separar zonas de la imagen en el
espacio espectral

\begin{lstlisting}
    ref.2016$kmeans <- reclas.2016
    xyplot(nir~red, groups=kmeans, data=ref.2016)
\end{lstlisting}

\begin{act}
    Grafique en los cortes del espacio espectral la imagen sin fusionar. Compare
    la diferencia entre clases espectrales y clases de informacion.
\end{act}

