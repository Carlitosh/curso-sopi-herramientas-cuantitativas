\documentclass[]{beamer}
%\documentclass[handout]{beamer}
%\documentclass[handout,draft]{beamer}
%\documentclass[]{article}



% Preambulo
% Paquetes de la ams
\usepackage{amsmath,amsthm,amssymb,amsfonts}
% Posibilidad de mover la pagina
\usepackage[a4paper]{geometry}
% Saco la indentacion en todos los parrafos.
%\usepackage{parskip}
% Codificacion UTF-8
\usepackage[utf8]{inputenc}
% Tablas e imagenes en espaniol
\usepackage[spanish,es-tabla]{babel}
% Mejores graficos
\usepackage{graphicx}
% tablas mas lindas
\usepackage{booktabs}
% Posibilidad de tocar los encabezados
\usepackage{fancyhdr}
%\pagestyle{fancy}
% Posibilidad de meter subfiguras
\usepackage[font=footnotesize, labelfont=it]{subcaption}
% Links a urls
\usepackage{url}
% Linkear referencias en pdfs
\usepackage{hyperref}
% Texto mas lindo para los pie de figura
\usepackage[margin=10pt,font=small,labelfont=bf, labelsep=endash]{caption}
% Mejores autores
\usepackage[affil-it]{authblk}
% Compatibilidad con PDF/A
\usepackage{xmpincl}
% Hoja a4 mas ancha
\usepackage{a4wide}
% Citas
\usepackage[backend=biber,style=ieee]{biblatex}
\addbibresource{biblio.bib}
% Cambio and por y
\renewcommand\Authand{y }
\renewcommand\Authands{, y }

% Codigo
\usepackage{listings}

% Coloreo los links
\usepackage[usenames,dvipsnames]{xcolor}
\hypersetup{colorlinks,
     linkcolor={red!50!black},
     citecolor={blue!50!black},
     urlcolor={blue!80!black} }
% Graficos con tikz
\usepackage{tikz}

% Dir tree
\usepackage{dirtree}

% Configuracion de listings para R
\lstset{%
  language=R,                     % the language of the code
  basicstyle=\footnotesize,       % the size of the fonts that are used for the code
  numbers=left,                   % where to put the line-numbers
  numberstyle=\tiny\color{gray},  % the style that is used for the line-numbers
  stepnumber=1,                   % the step between two line-numbers. If it's 1, each line
                                  % will be numbered
  numbersep=5pt,                  % how far the line-numbers are from the code
  backgroundcolor=\color{white},  % choose the background color. You must add \usepackage{color}
  showspaces=false,               % show spaces adding particular underscores
  showstringspaces=false,         % underline spaces within strings
  showtabs=false,                 % show tabs within strings adding particular underscores
  %frame=single,                   % adds a frame around the code
  rulecolor=\color{black},        % if not set, the frame-color may be changed on line-breaks within not-black text (e.g. commens (green here))
  tabsize=2,                      % sets default tabsize to 2 spaces
  captionpos=b,                   % sets the caption-position to bottom
  breaklines=true,                % sets automatic line breaking
  breakatwhitespace=false,        % sets if automatic breaks should only happen at whitespace
  title=\lstname,                 % show the filename of files included with \lstinputlisting;
                                  % also try caption instead of title
  keywordstyle=\color{blue},      % keyword style
  commentstyle=\color{OliveGreen},   % comment style
  stringstyle=\color{Plum}       % string literal style
}

\definecolor{A11}{HTML}{B2DF8A}
\definecolor{A12}{HTML}{33A02C}
\definecolor{A23}{HTML}{FDBF6F}
\definecolor{A24}{HTML}{FF7F00}
\definecolor{B15}{HTML}{FB9A99}
\definecolor{B16}{HTML}{E31A1C}
\definecolor{B27}{HTML}{A6CEE3}
\definecolor{B28}{HTML}{1F78B4}

%\usepackage{beamerarticle}

% Titulo
\title{Herramientas de Teledetección Cuantitativa\\{\small Educando al clasificador}}
\author{Francisco Nemiña}

\institute{Unidad de Educación y Formación Masiva \\
Comisión Nacional de Actividades Espaciales}

\logo{\includegraphics[height=0.7cm]{imagenes/sopi.png}}

\AtBeginSection[]
{
\begin{frame}
\frametitle{Esquema de presentación}
\tableofcontents[currentsection]
\end{frame}
}


\begin{document}
\begin{frame}
    \maketitle
\end{frame}

\section{Escenas del capítulo anterior}
\begin{frame}{La vez pasada vimos}
  \begin{itemize}[<+->]
    \item Que a partir de esto podiamos definir la $\rho_\lambda$ la firma espectral como una característica de cada cuerpo.
    \item La necesidad de definir categorías de uso y cobertura de forma concisa.
    \item La diferencia entre el concepto de categorias de uso y cobertura y clases espectrales.
    \item La importante del espacio espectral para comprender los métodos de clasificación.
    \item El funcionamiento del algoritmo k--means de segmentación.
  \end{itemize}
\end{frame}
%--- Next Frame ---%
\section{Matemática}
\subsection{Estadística}
\begin{frame}{\subsecname}
  \begin{block}{Notación}
    Notamos a la media para la clase $\omega_i$ como $$m_i = \frac{1}{q_i-1} \sum_j^{q_i} x_j$$ donde $q_i$ es la cantidad de píxeles de la clase.
  \end{block}\pause
  \begin{block}{Notación}
    La varianza como $$\sigma_i^2 = \frac{1}{q_i-1} \sum_j^{q_i} (x_j-m_i)^2$$ dónde los $x_j$ pertenecen a la clase $i$.
  \end{block}
\end{frame}
%--- Next Frame ---%

\begin{frame}{\subsecname}
  \begin{block}{Probabilidad condicional}
    Recordamos a la probabilidad condicional como $$p(x|\omega_i)$$ como la probabilidad de encontrar a un píxel en el punto $x$ del espacio espectral dado que sabemos que pertenece a la clase $\omega_i$.
  \end{block}
\end{frame}
%--- Next Frame ---%

\begin{frame}{\subsecname}
  \begin{block}{Teorema de Bayes}
    $$p(\omega_i|x) = \frac{p(x|\omega_i) p(\omega_i)}{p(x)}$$
    Es decir, la probabilidad de que un píxel pertenezca a la clase $\omega_i$ dado que se encuentra en el punto del espacio espectral $x$.
  \end{block}
\end{frame}
%--- Next Frame ---%

\begin{frame}{\subsecname}
  \begin{block}{Distribución de Gauss multidimensional}
    Si definimos a la matriz de covarianza como $$C_i = \frac{1}{q_i-1} \sum_j^{q_i} (x_j-m_i)(x_j-m_i)^T$$ podemos definir la distribución de Gauss en un espacio multidimensional como $$p(x|\omega_i) \sim \exp (\frac{-1}{2} (x-m_i)^T C_i^{-1} (x-m_i) )$$
  \end{block}
\end{frame}
%--- Next Frame ---%

\section{Clasificación supervisada}
\subsection{Idea}
\begin{frame}{\subsecname}
  \begin{alertblock}{Importante}
    Ahora tenemos que definir apriori cuales son las clases que queremos y como encontrarlas.
  \end{alertblock}
\end{frame}
%--- Next Frame ---%

\begin{frame}{\subsecname}
  \begin{figure}
  \includegraphics[width=0.6\textwidth]{imagenes/vector-3.png}
  \caption{Espacio vectorial.\footfullcite{richards2013remote}}
  \end{figure}
\end{frame}
%--- Next Frame ---%

\begin{frame}{\subsecname}
  \begin{figure}
  \includegraphics[width=0.6\textwidth]{imagenes/vector-2.png}
  \caption{Clasificación del espacio vectorial a partir de clases de entrenamiento.\footfullcite{richards2013remote}}
  \end{figure}
\end{frame}

\begin{frame}{\subsecname}
  \begin{block}{Esquema general}
    \begin{enumerate}[<+>]
      \item Decidir cuales son las clases de intereés.
      \item Elegir píxeles conocidos y representativos para cada clase a utilizar como áreas de entrenamiento.
      \item Estimar los parámetros del método de clasificación.
      \item Usar el clasificador para clasificar los pixeles.
      \item Producir mapas temáticos para extraer información.
      \item Corroborar la precisión de la clasificación con datos de campo
    \end{enumerate}
  \end{block}
\end{frame}
%--- Next Frame ---%

%--- Next Frame ---%
\subsection{Métodos}
\begin{frame}{\subsecname}
  \begin{block}{Generales}
    \begin{itemize}
      \item<.> Paralelepípedos
      \item<.> Distancia mínima
      \item<1> Máxima verosimilitud
      \item<.> Ángulo espectral
    \end{itemize}
  \end{block}
\end{frame}
%--- Next Frame ---%

\subsection{Máxima verosimilitud}

\begin{frame}{\subsecname}
  \begin{block}{Clasificador Bayesiano}
    Si conocemos las probabilidades condicionales $p(\omega_i|x)$ entonces un píxel $x$ pertenece a la clase $\omega_i$ si $$p(\omega_i|x)>p(\omega_j|x)$$ si $i \neq j$.
  \end{block}
  \pause
  \begin{alertblock}{Problema}
    No conocemos $p(\omega_i|x)$.
  \end{alertblock}
\end{frame}
%--- Next Frame ---%

\begin{frame}{\subsecname}
  \begin{block}{Solución}
    Usamos el teorema de Bayes y podemos escribir que un píxel $x$ pertenece a la clase $\omega_i$ si $$p(x|\omega_i)p(\omega_i)>p(x|\omega_j)p(\omega_j)$$ si $i \neq j$.
  \end{block}
  \pause
  \begin{block}{Función discriminante}
    Si definimos $g_i(x) = \log (p(x|\omega_i)p(\omega_i))$ entonces lo anterior se convierte en $x$ pertenece a la clase $\omega_i$ si $$g_i(x)>g_j(x)$$ si $i \neq j$.
  \end{block}
\end{frame}
%--- Next Frame ---%

\begin{frame}{\subsecname}
  \begin{block}{Caso Gaussiano}
    Si suponemos que la distribución $p$ es normal y que, apriori la probabilidad de pertenecer a una clase es equiprobable, tenemos que
    $$g_i(x) = -\log |C_i| - (x-m_i)^T C_i^{-1} (x-m_i)$$.
  \end{block}
  \pause
  \begin{alertblock}{Observaciones:}
    Como la distribución de Gauss no se anula nunca, esto puede clasificar a lo largo de todo el espacio
  \end{alertblock}
\end{frame}
%--- Next Frame ---%

\begin{frame}{\subsecname}
  \begin{block}{Superficies de equiprobabilidad}
    Si buscamos la superficies de $$g_i = g_j$$ ese espacio queda dividido en distintos sectores donde es siempre mayor la probabilidad de pertenecer a una clase.
    \pause
    Son
    \begin{itemize}[<+>]
      \item Elipses
      \item Parábolas
      \item Hipérbolas
    \end{itemize}
  \end{block}
\end{frame}
%--- Next Frame ---%

\begin{frame}{\subsecname}
  \begin{figure}
  \includegraphics[width=0.6\textwidth]{imagenes/areas.png}
  \caption{Vista en el espacio vectorial.\footfullcite{clasif1}}
  \end{figure}
\end{frame}
%--- Next Frame ---%

\begin{frame}{\subsecname}
  \begin{figure}
  \includegraphics[width=0.6\textwidth]{imagenes/max.png}
  \caption{Vista en el espacio vectorial.\footfullcite{clasif1}}
  \end{figure}
\end{frame}
%--- Next Frame ---%

\begin{frame}{\subsecname}
  \begin{block}{Número de píxeles necesarios}
    Para estimar la matriz de covarianza se necesitan al menos $N(N+1)$ elementos. Es decir, al menos $N+1$ píxeles.
  \end{block}
\end{frame}
%--- Next Frame ---%

\begin{frame}{\subsecname}
  \begin{figure}
  \includegraphics[width=0.8\textwidth]{imagenes/train.png}
  \caption{Clasificación supervisada incrementando el número de píxeles de entrenamiento.\footfullcite{richards2013remote}}
  \end{figure}
\end{frame}
%--- Next Frame ---%

\begin{frame}{\subsecname}
  \begin{alertblock}{Número de píxeles necesarios}
    En la práctica, se necesitan entre $10N$ y $100N$ píxeles.
  \end{alertblock}
\end{frame}
%--- Next Frame ---%

\begin{frame}{\subsecname}
  \begin{figure}
  \includegraphics[width=0.6\textwidth]{imagenes/thresh.png}
  \caption{Problemas de clasificación y umbral.\footfullcite{richards2013remote}}
  \end{figure}
\end{frame}
%--- Next Frame ---%

\begin{frame}{\subsecname}
  \begin{figure}
  \includegraphics[width=0.8\textwidth]{imagenes/hughes.png}
  \caption{Otro problema, fenómeno de Hughes.\footfullcite{richards2013remote}}
  \end{figure}
\end{frame}
%--- Next Frame ---%

\begin{frame}{\subsecname}
  \begin{figure}
  \includegraphics[width=0.6\textwidth]{imagenes/t_area.png}
  \caption{Imagen con áreas de entrenamineto.\footfullcite{richards2013remote}}
  \end{figure}
\end{frame}
%--- Next Frame ---%

\begin{frame}{\subsecname}
  \begin{figure}
  \includegraphics[width=0.6\textwidth]{imagenes/t_map.png}
  \caption{Imagen clasificada.\footfullcite{richards2013remote}}
  \end{figure}
\end{frame}
%--- Next Frame ---%

\subsection{Otros métodos}
\begin{frame}{\subsecname}
  \begin{block}{Pocos píxeles}
    Si contamos con pocos píxeles de entrenamiento, podemos caer en otros metodos.
    \begin{itemize}
      \item<.> Paralelepípedos
      \item<1> Distancia mínima
      \item<.> Máxima verosimilitud
      \item<1> Ángulo espectral
    \end{itemize}
  \end{block}
\end{frame}
%--- Next Frame ---%

\begin{frame}{\subsecname}
  \begin{block}{Distancia mínima}
    Si buscamos la superficies de $g_i = g_j$ con $g_i = 2 m_i x - m_i m_i$
    y me divide a mi espacio por hiperplanos.
  \end{block}
\end{frame}
%--- Next Frame ---%

\begin{frame}{\subsecname}
  \begin{figure}
  \includegraphics[width=0.6\textwidth]{imagenes/min.png}
  \caption{Vista en el espacio vectorial.\footfullcite{clasif1}}
  \end{figure}
\end{frame}
%--- Next Frame ---%

\begin{frame}{\subsecname}
  \begin{block}{Angulo espectral}
    Dividimos en este caso al espacio utilizando el ángulo correspondiente a los píxeles de entrenamiento.
  \end{block}
\end{frame}
%--- Next Frame ---%

\begin{frame}{\subsecname}
  \begin{figure}
  \includegraphics[width=0.6\textwidth]{imagenes/angle.png}
  \caption{Vista en el espacio vectorial.\footfullcite{richards2013remote}}
  \end{figure}
\end{frame}
%--- Next Frame ---%

\section{Práctica}

\begin{frame}{\secname}
  \begin{exampleblock}{Actividades prácticas de la cuarta clase}
    \begin{enumerate}[<+>]
      \item Abra las imágenes Landsat 8 y digitalice las coberturas de interés.
      \item Clasifique la imagen utilizando un vector de entrenamiento por clase.
      \item Clasifique la imagen utilizando varios vectores de entrenamiento por clase.
      \item Utilizce la herramienta de estadísticas globales para estimar las áreas correspondientes a cada uso y cobertura.
    \end{enumerate}
  \end{exampleblock}
\end{frame}
%--- Next Frame ---%
\section{Práctica}

\begin{frame}{Práctica}
  \begin{exampleblock}{Actividades prácticas de la primera clase}
    \begin{enumerate}
      \item Clasifique la imagen por el método de máxima verosimilitud con una sola clase de entrenamiento por categoría de uso y cobertura.
      \item Clasifique la imagen por el método de máxima verosimilitud con varias clases de entrenamiento por categoría de uso y cobertura.
      \item Utilizar la herramienta de estadísticas globales para estimar las áreas correspondientes a cada uso y cobertura.
    \end{enumerate}
  \end{exampleblock}
\end{frame}
%--- Next Frame ---%

\end{document}
