Buen día. Este es el apunte del curso SoPI 2 de herramientas de teledetección
cuantitativa. La idea es que este apunte reproduzca el material de las clases
basado en el curso del segundo cuatrimestre de 2016.

El apunte va a ser lo mas incluso posible pero claramente incompleto por lo
tanto es bueno complementarlo con otros libros/guías/papers/etc. Al final de
cada capítulo citaremos las fuentes que sean releveantes para el mismo. Sin
embargo, citar todas las fuentes es claramente imposible por lo tanto
incentivamos a los alumnos a llevar adelante su propia busqueda y ser crítico de
las mismas.

El curso va a trabajar utilizando el concepto de \emph{vector pixel} para
estudiar distintas herramientas de teledeteccion cuantitativa. La forma de tocar
estas herramientas es desde un punto de vista introductorio. Intentando siempre
motivar el porque se introducen y que aplicacion tienen, pero no siempre
demostrando su validez matematica.

A grandes rasgos el curso se divide en dos partes. La primera llamada
\emph{transformaciones en el dominio espectral} incluye los capitulos 1 a 4 y
toca los conceptos basicos de la teledeteccion, firmas espectrales, vectores
pixeles, correcciones radimetricas y atmosfericas, indices espectrales y
transformaciones en el espacio espectral. La segunda llama \emph{clasificaciones
supervisadas en la practica} incluye los capitulos 5 a 7 y cubre distintos
metodos de clasificacion supervisada con su correspondiente validacion.

Durante el curso que funciono de manera teorico-practica se trabajo con un caso
de estudio de la region de Triple Frontera, en Misiones. Al final de cada
capitulo se incluira las actividades sugeridas para el mismo mas algunos
ejercicios que pueden aportar a la compresion de los conceptos espuestos.

Como concentrarnos en un tema permite siplificar las cosas, en el curso
trabajamos en la parte optica del espectro. Es decir, de los $0.4\mu mu$ a $14.0
\mu m$ incluyendo el visible, infrarrojo cercano, medio y termico. Las imagenes
que se utilizaron fueron en su mayoria Landsat 8 del sensor OLI, pero los mismos
conceptos pueden aplicarse a imagenes de otras fuentes.

Como cualquier producción de un texto, este no va a estar libre de defectos. Les
pedimos por favor que ante cualquier problema escriban a
\url{fnemina@conae.gov.ar} con los comentarios pertinentes para dar una solución
lo mas pronta posible al problema.
