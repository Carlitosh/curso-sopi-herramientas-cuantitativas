\documentclass[hidelinks,12pt]{article}

% Preambulo por defecto
% Paquetes para usar bien el idioma español
\usepackage[spanish,es-tabla]{babel}
\selectlanguage{spanish}
\usepackage[utf8]{inputenc}

% Paquetes para usar mejores imagenes
\usepackage{graphicx}

% Paquetes para links y tabla de contenidos en el PDF
\usepackage{hyperref}
\hypersetup{colorlinks=true,allcolors=blue}
%\usepackage{hypcap}

% Paquetes para mejores tablas
\usepackage{booktabs}

% Mejor matematica
\usepackage{amsmath}

% Fuentes de las imagenes
\usepackage[absolute,overlay]{textpos}

% Paquete captions
\usepackage[justification=centering,labelformat=empty,labelsep=none]{caption}

% Opciones para ticks
\usepackage{tikz}
\usetikzlibrary{shapes,arrows,positioning}

\tikzstyle{decision} = [diamond, draw, fill=blue!20, text width=4em, text badly centered, node distance=2cm, inner sep=0pt,on grid]
\tikzstyle{block} = [rectangle, draw, fill=blue!20, text width=8em, text centered, rounded corners, minimum height=2em,on grid]
\tikzstyle{line} = [draw, -latex]

% Citas bibliograficas
\usepackage[backend=biber]{biblatex}
\renewcommand{\footnotesize}{\tiny}
\addbibresource{biblio.bib}

% Mejoro las captions
\setbeamertemplate{caption}{\raggedright\insertcaption\par}

\setbeamertemplate{caption}{%
\begin{beamercolorbox}[wd=0.85\paperwidth, sep=.2ex]{block body}\insertcaption%
\end{beamercolorbox}%
}


% Sacar barra de navegacion
\setbeamertemplate{navigation symbols}{}%remove navigation symbols

% Transparencias en items
\setbeamercovered{transparent}

% Estilo de diapositivas
% \usetheme{Boadilla}
\usecolortheme{whale}
\usecolortheme{orchid}


\definecolor{A11}{HTML}{B2DF8A}
\definecolor{A12}{HTML}{33A02C}
\definecolor{A23}{HTML}{FDBF6F}
\definecolor{A24}{HTML}{FF7F00}
\definecolor{B15}{HTML}{FB9A99}
\definecolor{B16}{HTML}{E31A1C}
\definecolor{B27}{HTML}{A6CEE3}
\definecolor{B28}{HTML}{1F78B4}

\title{SoPI II \- Herramientas de Teledetecci\'on Cuantitativa \\ 
\emph{Tutorial de instalaci\'on de python}}
\author{Francisco
Nemi\~na\thanks{\href{mailto:fnemina@conae.govar}{fnemina@conae.gov.ar}}}
\affil{Unidad de Educaci\'on y Formaci\'on Masiva\\ 
    Comisi\'on Nacional de Actividades Espaciales}
\date{\today}

\begin{document}

\maketitle

\section*{Introducci\'on}
Este tutorial lo guiar\'a en los pasos necesarios para instalar y utilizar
python en windows.

Los notebook hechos con dicho lenguaje seran utilizados como herramientas
complementarias durante el curso.

\section{Linux}
En esta seccion veremos como instalar y ejecutar python en Linux.

\subsection{Instalar miniconda}

De la p\'agina de conda, \url{http://conda.pydata.org/miniconda.html}, descargue 
el instalador de miniconda de la versi\'on 3.5 correspondiente a su sistema operativo. 

Abra una terminal y ejecute el comando
\begin{quote}
\begin{verbatim}
bash ~/Downloads/Anaconda3-2.4.0-Linux-x86_64.sh
\end{verbatim}
\end{quote}
Siga el proceso de instalacion aceptando la licencia. Cuando le pregunte si
desea agregar miniconda a la variable path, responda \texttt{yes} y precione
enter. Cierre luego la terminal.

\subsection{Configuraci\'on de paquetes de python}

Abra una terminal. Ejecute luego los siguientes comandos, presionando enter
luego de cada uno. Responda \texttt{y} cuando el instalador le pregunte 
\begin{quote}
\begin{verbatim}
conda create -n py34 python=3.4
source activate py34
conda install -n py34 numpy scipy matplotlib scikit-learn jupyter gdal
\end{verbatim}
\end{quote}
Cierre luego la terminal.

\subsection{Ejecutar jupyter notebook}
Para ejecutar los notebooks de pythons abra una terminal. Ejecute luego los 
siguientes comandos, presionando enter luego de cada uno

\begin{quote}
\begin{verbatim}
source activate py34
jupyter notebook
\end{verbatim}
\end{quote}

\section{Windows}
En esta seccion veremos como instalar y ejecutar python en windows.

\subsection{Instalar miniconda}

De la p\'agina de conda, \url{http://conda.pydata.org/miniconda.html}, descargue 
el instalador de miniconda de la versi\'on 3.5 correspondiente a su sistema
operativo. Ejec\'utelo. Cuando le mismo le pregunte para que usuario instalarlo,
seleccione la opcion \emph{Just me}.

\subsection{Configuraci\'on de paquetes de python}

Abra una terminal haciendo click en \emph{inicio} y en el campo de \emph{Buscar
programas y archivos} escriba 
\begin{quote}
\begin{verbatim}
cmd.exe
\end{verbatim}
\end{quote}
y precione enter.

Ejecute luego los siguientes comandos, presionando enter luego de cada uno. Responda 
\texttt{y} cuando el instalador le pregunte.

\begin{quote}
\begin{verbatim}
conda create -n py34 python=3.4
activate py34
conda install -n py34 numpy scipy matplotlib scikit-learn jupyter gdal
\end{verbatim}
\end{quote}

Cierre luego la terminal.

\subsection{Ejecutar jupyter notebook}
Para ejecutar los notebooks de pythons abra una terminal haciendo click en 
\emph{inicio} y en el campo de \emph{Buscar programas y archivos} escriba 
\begin{quote}
\begin{verbatim}
cmd.exe
\end{verbatim}
\end{quote}
y precione enter.

Ejecute luego los siguientes comandos, presionando enter luego de cada uno

\begin{quote}
\begin{verbatim}
activate py34
jupyter notebook
\end{verbatim}
\end{quote}

Recuerde que para ejecutarlos debe tener configurado \emph{firefox} o
\emph{google-chrome} como su navegador predeterminado.

%\printbibliography\
\end{document}
