% Paquetes para usar bien el idioma español
\usepackage[spanish,es-tabla]{babel}
\selectlanguage{spanish}
\usepackage[utf8]{inputenc}

% Paquetes para usar mejores imagenes
\usepackage{graphicx}

% Paquetes para links y tabla de contenidos en el PDF
\usepackage{hyperref}
\hypersetup{colorlinks=true,allcolors=blue}
%\usepackage{hypcap}

% Paquetes para mejores tablas
\usepackage{booktabs}

% Mejor matematica
\usepackage{amsmath}

% Fuentes de las imagenes
\usepackage[absolute,overlay]{textpos}

% Paquete captions
\usepackage[justification=centering,labelformat=empty,labelsep=none]{caption}

% Opciones para ticks
\usepackage{tikz}
\usetikzlibrary{shapes,arrows,positioning}

\tikzstyle{decision} = [diamond, draw, fill=blue!20, text width=4em, text badly centered, node distance=2cm, inner sep=0pt,on grid]
\tikzstyle{block} = [rectangle, draw, fill=blue!20, text width=8em, text centered, rounded corners, minimum height=2em,on grid]
\tikzstyle{line} = [draw, -latex]

% Citas bibliograficas
\usepackage[backend=biber]{biblatex}
\renewcommand{\footnotesize}{\tiny}
\addbibresource{biblio.bib}

% Mejoro las captions
\setbeamertemplate{caption}{\raggedright\insertcaption\par}

\setbeamertemplate{caption}{%
\begin{beamercolorbox}[wd=0.85\paperwidth, sep=.2ex]{block body}\insertcaption%
\end{beamercolorbox}%
}


% Sacar barra de navegacion
\setbeamertemplate{navigation symbols}{}%remove navigation symbols

% Transparencias en items
\setbeamercovered{transparent}

% Estilo de diapositivas
% \usetheme{Boadilla}
\usecolortheme{whale}
\usecolortheme{orchid}
