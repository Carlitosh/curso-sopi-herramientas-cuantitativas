%\documentclass[]{beamer}
\documentclass[handout]{beamer}
%\documentclass[handout,draft]{beamer}
%\documentclass[]{article}


% Preambulo
% Paquetes de la ams
\usepackage{amsmath,amsthm,amssymb,amsfonts}
% Posibilidad de mover la pagina
\usepackage[a4paper]{geometry}
% Saco la indentacion en todos los parrafos.
%\usepackage{parskip}
% Codificacion UTF-8
\usepackage[utf8]{inputenc}
% Tablas e imagenes en espaniol
\usepackage[spanish,es-tabla]{babel}
% Mejores graficos
\usepackage{graphicx}
% tablas mas lindas
\usepackage{booktabs}
% Posibilidad de tocar los encabezados
\usepackage{fancyhdr}
%\pagestyle{fancy}
% Posibilidad de meter subfiguras
\usepackage[font=footnotesize, labelfont=it]{subcaption}
% Links a urls
\usepackage{url}
% Linkear referencias en pdfs
\usepackage{hyperref}
% Texto mas lindo para los pie de figura
\usepackage[margin=10pt,font=small,labelfont=bf, labelsep=endash]{caption}
% Mejores autores
\usepackage[affil-it]{authblk}
% Compatibilidad con PDF/A
\usepackage{xmpincl}
% Hoja a4 mas ancha
\usepackage{a4wide}
% Citas
\usepackage[backend=biber,style=ieee]{biblatex}
\addbibresource{biblio.bib}
% Cambio and por y
\renewcommand\Authand{y }
\renewcommand\Authands{, y }

% Codigo
\usepackage{listings}

% Coloreo los links
\usepackage[usenames,dvipsnames]{xcolor}
\hypersetup{colorlinks,
     linkcolor={red!50!black},
     citecolor={blue!50!black},
     urlcolor={blue!80!black} }
% Graficos con tikz
\usepackage{tikz}

% Dir tree
\usepackage{dirtree}

% Configuracion de listings para R
\lstset{%
  language=R,                     % the language of the code
  basicstyle=\footnotesize,       % the size of the fonts that are used for the code
  numbers=left,                   % where to put the line-numbers
  numberstyle=\tiny\color{gray},  % the style that is used for the line-numbers
  stepnumber=1,                   % the step between two line-numbers. If it's 1, each line
                                  % will be numbered
  numbersep=5pt,                  % how far the line-numbers are from the code
  backgroundcolor=\color{white},  % choose the background color. You must add \usepackage{color}
  showspaces=false,               % show spaces adding particular underscores
  showstringspaces=false,         % underline spaces within strings
  showtabs=false,                 % show tabs within strings adding particular underscores
  %frame=single,                   % adds a frame around the code
  rulecolor=\color{black},        % if not set, the frame-color may be changed on line-breaks within not-black text (e.g. commens (green here))
  tabsize=2,                      % sets default tabsize to 2 spaces
  captionpos=b,                   % sets the caption-position to bottom
  breaklines=true,                % sets automatic line breaking
  breakatwhitespace=false,        % sets if automatic breaks should only happen at whitespace
  title=\lstname,                 % show the filename of files included with \lstinputlisting;
                                  % also try caption instead of title
  keywordstyle=\color{blue},      % keyword style
  commentstyle=\color{OliveGreen},   % comment style
  stringstyle=\color{Plum}       % string literal style
}

\definecolor{A11}{HTML}{B2DF8A}
\definecolor{A12}{HTML}{33A02C}
\definecolor{A23}{HTML}{FDBF6F}
\definecolor{A24}{HTML}{FF7F00}
\definecolor{B15}{HTML}{FB9A99}
\definecolor{B16}{HTML}{E31A1C}
\definecolor{B27}{HTML}{A6CEE3}
\definecolor{B28}{HTML}{1F78B4}

%\usepackage{beamerarticle}

% Titulo
\title{Herramientas de Teledetección Cuantitativa\\{\small Segmentando el espacio mirando para otro lado}}
\author{Francisco Nemiña}

\institute{Unidad de Educación y Formación Masiva \\
Comisión Nacional de Actividades Espaciales}

\logo{\includegraphics[height=0.7cm]{imagenes/sopi.png}}

\AtBeginSection[]
{
\begin{frame}
\frametitle{Esquema de presentación}
\tableofcontents[currentsection]
\end{frame}
}


\begin{document}
\begin{frame}
    \maketitle
\end{frame}

\section{Escenas del capítulo anterior}
\begin{frame}{La vez pasada vimos}
  \begin{itemize}[<+->]
    \item Que a partir de esto podiamos definir la $\rho_\lambda$ la firma espectral como una característica de cada cuerpo.
    \item Definimos 3 tipos de firmas espectrales \emph{patrón} y como se comportaba cada una.
    \item Que es importante corregir a las imágenes atmosfericamente para obtener el valor de reflectancia del píxel.
    \item Que podemos definir índices a partir de hacer operaciones entre los valores de los píxeles como si fueran números.
    \item Que a partir del concepto de espacio espectral podemos hacer rotaciones que cambian la forma de mostrar la información.
  \end{itemize}
\end{frame}
%--- Next Frame ---%

\section{Clases y categorías}

\begin{frame}{Categorías de uso y cobertura}
  \begin{block}{Definición}
    Hablamos de \emph{categorías de uso y cobertura} cuando hablamos de nuestras categorías de interés para nuestro estudio.
  \end{block}
  \begin{block}{Propiedades}
    Un esquema de clasificación debe ser
    \begin{itemize}[<+.>]
      \item Exaustivo.
      \item Mutuamente excluyente.
      \item Mismo nivel gerarquico.
    \end{itemize}
  \end{block}
\end{frame}
%--- Next Frame ---%

\begin{frame}{Clase espectral}
  \begin{block}{Definición}
    Hablamos de una \emph{clase espectral} cuando hablamos de un conjunto de píxeles que podemos agrupar espectralmente.
  \end{block}
\end{frame}
%--- Next Frame ---%

\begin{frame}{Clave vs. Categoría}
  \begin{block}{Observación}
    Clase espectral y categoría de uso y cobertura \emph{NO} son lo mismo.
  \end{block}
\end{frame}
%--- Next Frame ---%

\begin{frame}{De la imagen...}

    Queremos cambiar de información espectral a categorías. \pause
    \begin{figure}
  \includegraphics[width=0.6\textwidth]{imagenes/imagen.png}
  \caption{Imagen de la zona de interés en combinación RGB.}
  \end{figure}
\end{frame}
%--- Next Frame ---%

\begin{frame}{...al espacio espectral...}
  \begin{figure}
    \includegraphics[width=0.5\textwidth]{imagenes/cluster.png}
    \caption{Clustering en $R^2$.\footfullcite{cluster}}
  \end{figure}
\end{frame}
%--- Next Frame ---%

\begin{frame}{... a las categorías}
  \begin{figure}
  \includegraphics[width=0.6\textwidth]{imagenes/im_class.png}
  \caption{Clasificación de la zona de interés.}
\end{figure}
\end{frame}
%--- Next Frame ---%

\begin{frame}{Objetivo}
  \begin{exampleblock}{¿Cómo?}
    Realizando clasificaciones en el espacio vectorial de la imagen. Estos algoritmos se van a basar en los valores individuales vector píxel.
  \end{exampleblock}
\end{frame}
%--- Next Frame ---%

\section{Clustering}

\subsection{Introducción}

\begin{frame}{Introducción}
  \begin{block}{Distancia}
    Para poder trabajar cómodos en el espacio vectorial vamos a tener que definir la distancia entre dos vectores
    \begin{equation}
    d(\rho_1, \rho_2) = \left( | \rho_{1,i} - \rho_{2,i} |^p \right)^{1/p}
    \end{equation}
  \end{block}
\end{frame}

\begin{frame}{Introducción}
  \begin{exampleblock}{Taxisita}
    Cuando $p=1$ tenemos
    \begin{equation}
      d(\rho_1, \rho_2) = \left( | \rho_{1,i} - \rho_{2,i} | \right)
    \end{equation}
  \end{exampleblock}\pause
  \begin{exampleblock}{Euclídea}
    Cuando $p=2$ tenemos
    \begin{equation}
      d(\rho_1, \rho_2) = \left( | \rho_{1,i} - \rho_{2,i} |^2 \right)^{1/2}
    \end{equation}
  \end{exampleblock}
\end{frame}
%--- Next Frame ---%

\begin{frame}{Introducción}
  \begin{block}{Criterio habitual}
    Encontrar clases $c_i$ que minimice
      \begin{equation}
          SSE = \sum_{c_i} \sum_{x \in c_i} (x-x_i)^2
      \end{equation}
       donde $x_i$ es el promedio de todos los valores de cada clase.
  \end{block}
\end{frame}
%--- Next Frame ---%

\begin{frame}{Introducción}
  \begin{exampleblock}{Una solución}
    Si $c_i = {x_i}$ esto da cero y es mínimo. Entonces tiene al menos una solución.
  \end{exampleblock}\pause
  \begin{block}{Otras soluciones}
    Tenemos que encontrar N categorías $c_i$ que minimicen esto.
  \end{block}
\end{frame}
%--- Next Frame ---%

\begin{frame}{Introducción}
  \begin{block}{Cuentas}
    Esto son MUCHAS cuentas y tomaría mucho tiempo. \pause Tenemos que buscar otra manera más eficiente de hacerlo.
  \end{block}
\end{frame}
%--- Next Frame ---%

\subsection{k-means}
\begin{frame}{k-means}
  \begin{exampleblock}{Ejemplo en 1-D}
    Edades.
  \end{exampleblock}
\end{frame}
%--- Next Frame ---%

\begin{frame}{k-means}
  \begin{figure}
    \includegraphics[width=0.6\textwidth]{imagenes/byn.png}
    \caption{Imagen a clasificar.\footfullcite{kmean}}
  \end{figure}
\end{frame}

\begin{frame}{k-means}
  \begin{figure}
    \includegraphics[width=0.9\textwidth]{imagenes/step1.png}
    \caption{Proceso paso a paso.\footfullcite{kmean}}
  \end{figure}
\end{frame}

\begin{frame}{k-means}
  \begin{figure}
    \includegraphics[width=0.9\textwidth]{imagenes/step2.png}
    \caption{Proceso paso a paso.\footfullcite{kmean}}
  \end{figure}
\end{frame}

\begin{frame}{k-means}
  \begin{figure}
    \includegraphics[width=0.9\textwidth]{imagenes/step.png}
    \caption{Proceso paso a paso.\footfullcite{kmean}}
  \end{figure}
\end{frame}

\begin{frame}{k-means}
  \begin{block}{Descripción del algoritmo}
    \begin{enumerate}[<+>]
      \item Selecciono N clases iniciales
      \item Asigno los píxeles a estas clases
      \item Calculo los centroides de las clases clasificadas
      \item Repito 2 - 4 con los nuevos centroides hasta converger
    \end{enumerate}
  \end{block}
\end{frame}
%--- Next Frame ---%

\begin{frame}{k-means}
  \begin{figure}
    \includegraphics[width=0.8\textwidth]{imagenes/sse.png}
    \caption{Y que pasa con la función de SSE a minimizar. \footfullcite{richards2013remote}}
  \end{figure}
\end{frame}

\subsection{Problemas}

\begin{frame}{Problemas}
  \begin{figure}
    \includegraphics[width=0.8\textwidth]{imagenes/minimo.png}
    \caption{Mínimo local vs. mínimo global en 1-D.\footfullcite{minimo}}
  \end{figure}
\end{frame}
%--- Next Frame ---%

\begin{frame}{Problemas}
  \begin{alertblock}{Selección inicial de clases}
    Lo que determina a que mínimo converge es la selección inicial de clases. Además no siempre me garantizo generar N clases, puedo generar menos de las deseadas.
  \end{alertblock}\pause
  \begin{block}{Como elijo las medias iniciales}
    \begin{itemize}[<+>]
      \item De forma estocástica
      \item Con algún criterio estadístico
    \end{itemize}
  \end{block}
\end{frame}

\subsection{isodata}

\begin{frame}{isodata}
  \begin{block}{Diferencias con respecto a kmeans}
    El algoritmo es básicamente el mismo, pero implementa tres condiciones adicionales.
    \begin{itemize}[<+>]
      \item Eliminar cluster si no son estadísticamente relevantes.
      \item Fusionar cluster si espectralmente son similares.
      \item Partir clusters que son muy alargados.
    \end{itemize}
  \end{block}
\end{frame}

\begin{frame}{isodata}
  \begin{figure}
    \includegraphics[width=0.6\textwidth]{imagenes/isodata.png}
    \caption{Clasificación no supervisada por isodata.\footfullcite{isodata}}
  \end{figure}
\end{frame}
%--- Next Frame ---%

\section{Práctica}

\begin{frame}{Práctica}
  \begin{exampleblock}{Actividades prácticas de la primera clase}
    \begin{enumerate}
      \item Clasifique la imagen por el método k-means con 7 clases.
      \item Clasifique la imagen por el método k-means con 70 clases.
      \item Utilizar la herramienta de estadísticas globales para estimar las áreas correspondientes a cada uso y cobertura.
    \end{enumerate}
  \end{exampleblock}
\end{frame}
%--- Next Frame ---%

\end{document}
