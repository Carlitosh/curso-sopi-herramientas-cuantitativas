\documentclass[hidelinks,12pt]{article}

% Preambulo por defecto
% Paquetes de la ams
\usepackage{amsmath,amsthm,amssymb,amsfonts}
% Posibilidad de mover la pagina
\usepackage[a4paper]{geometry}
% Saco la indentacion en todos los parrafos.
%\usepackage{parskip}
% Codificacion UTF-8
\usepackage[utf8]{inputenc}
% Tablas e imagenes en espaniol
\usepackage[spanish,es-tabla]{babel}
% Mejores graficos
\usepackage{graphicx}
% tablas mas lindas
\usepackage{booktabs}
% Posibilidad de tocar los encabezados
\usepackage{fancyhdr}
%\pagestyle{fancy}
% Posibilidad de meter subfiguras
\usepackage[font=footnotesize, labelfont=it]{subcaption}
% Links a urls
\usepackage{url}
% Linkear referencias en pdfs
\usepackage{hyperref}
% Texto mas lindo para los pie de figura
\usepackage[margin=10pt,font=small,labelfont=bf, labelsep=endash]{caption}
% Mejores autores
\usepackage[affil-it]{authblk}
% Compatibilidad con PDF/A
\usepackage{xmpincl}
% Hoja a4 mas ancha
\usepackage{a4wide}
% Citas
\usepackage[backend=biber,style=ieee]{biblatex}
\addbibresource{biblio.bib}
% Cambio and por y
\renewcommand\Authand{y }
\renewcommand\Authands{, y }

% Codigo
\usepackage{listings}

% Coloreo los links
\usepackage[usenames,dvipsnames]{xcolor}
\hypersetup{colorlinks,
     linkcolor={red!50!black},
     citecolor={blue!50!black},
     urlcolor={blue!80!black} }
% Graficos con tikz
\usepackage{tikz}

% Dir tree
\usepackage{dirtree}

% Configuracion de listings para R
\lstset{%
  language=R,                     % the language of the code
  basicstyle=\footnotesize,       % the size of the fonts that are used for the code
  numbers=left,                   % where to put the line-numbers
  numberstyle=\tiny\color{gray},  % the style that is used for the line-numbers
  stepnumber=1,                   % the step between two line-numbers. If it's 1, each line
                                  % will be numbered
  numbersep=5pt,                  % how far the line-numbers are from the code
  backgroundcolor=\color{white},  % choose the background color. You must add \usepackage{color}
  showspaces=false,               % show spaces adding particular underscores
  showstringspaces=false,         % underline spaces within strings
  showtabs=false,                 % show tabs within strings adding particular underscores
  %frame=single,                   % adds a frame around the code
  rulecolor=\color{black},        % if not set, the frame-color may be changed on line-breaks within not-black text (e.g. commens (green here))
  tabsize=2,                      % sets default tabsize to 2 spaces
  captionpos=b,                   % sets the caption-position to bottom
  breaklines=true,                % sets automatic line breaking
  breakatwhitespace=false,        % sets if automatic breaks should only happen at whitespace
  title=\lstname,                 % show the filename of files included with \lstinputlisting;
                                  % also try caption instead of title
  keywordstyle=\color{blue},      % keyword style
  commentstyle=\color{OliveGreen},   % comment style
  stringstyle=\color{Plum}       % string literal style
}

\definecolor{A11}{HTML}{B2DF8A}
\definecolor{A12}{HTML}{33A02C}
\definecolor{A23}{HTML}{FDBF6F}
\definecolor{A24}{HTML}{FF7F00}
\definecolor{B15}{HTML}{FB9A99}
\definecolor{B16}{HTML}{E31A1C}
\definecolor{B27}{HTML}{A6CEE3}
\definecolor{B28}{HTML}{1F78B4}


\title{SoPI II - Herramientas de Teledetecci\'on Cuantitativa \\ 
\emph{Gu\'{\i}a de actividades: Uso del suelo en el departamento de Iguaz\'u,
provincia de Misiones}}
\author{Francisco
Nemi\~na\thanks{\href{mailto:fnemina@conae.govar}{fnemina@conae.gov.ar}}}
\affil{Unidad de Educaci\'on y Formaci\'on Masiva\\ 
    Comisi\'on Nacional de Actividades Espaciales}
\date{\today}

\begin{document}

\maketitle

\section*{Introducci\'on}

La utilización de imágenes satelitales permite analizar grandes extensiones del
territorio, contando con un registro histórico con el cual realizar
comparaciones.

En la provincia de Misiones, el departamento de Iguazo es lindante a Brasil y 
Paraguay conformandose la zona conocida como \emph{triple frontera}. Dentro del 
mismo podemos encontrar la Represa de Urugua-\'{\i} y el parque nacional
Iguaz\'u.

Tomaremos entonces a dicha regi\'on como \'area de estudio durante este curso
con el objetivo de obtener una estimaci\'on confiable de los distintos usos del
suelo dentro de la provincia.

Utilizaremos para esto imagenes satelitales de los satelites Landsat 8, SPOT-5 y
el producto de NDVI obtenido de los satelites TERRA y AQUA obtenidas durante el
a\~no 2014. 

\newpage
\section{Análisis de Firmas Espectrales}
Es objetivo de esta practica familiarizarse con la zona de interes, estudiar la
firma espectral de distintas categorias de uso del suelo y comprender como se
relacionan las mismas con la interpretaci\'on visual.

\subsection{Actividades pr\'acticas}
\begin{enumerate}
    \item Abra la imagen \texttt{l8\_oli\_2014-08-06.tif}. Realice distintas
        combinaciones de banda y seleccione la que mejor le permita distinguir 
        detalles de la vegetación.
    \item Encuentre dentro de la escena parches de coberturas uniformes y 
        digitalícelos utilizando la herramienta de edición vectorial. Tome
        ejemplos de distintas coberturas correspondientes a los casos de la
        tabla~\ref{tabla1}
    \item Utilizando la herramienta de extracción de estadísticas globales
        calcule y grafique la media correspondiente a cada cobertura analizada 
        en el punto 2.
    \item Repita el proceso para coberturas del mismo tipo en la imagen 
        correspondiente a \texttt{l1\_l8\_oli\_2013-08-19.tif}.
\end{enumerate}
\subsection{Preguntas}
\begin{enumerate}
    \item Justifique desde el punto de vista de la firma espectral por qué una
        combinación de bandas que incluya zonas del espectro visible, el
        infrarrojo cercano y el infrarrojo medio aporta más información sobre la
        vegetación.
    \item Compare las respuestas espectrales de la vegetación en la zona del
        espectro visible en agosto de 2013 y 2014 y explique qué diferencias
        encuentra. ¿De qué color debe verse la vegetación si su respuesta fuese
        como la de la firma espectral del mes de agosto?
    \item Compare las respuestas espectrales obtenidas para el agua en los meses
        de agosto de 2013 y 2014. En caso de encontrar diferencias, diga en que zona
        del espectro se encuentran las mismas y a qué supone que se debe esta
        diferencia.
    \item Seleccione una de las dos imágenes en la cual considere que las
        respuestas espectrales son más similares a la firma espectral que espera
        observar. A partir de dicha imagen genere un gráfico que incluya las
        firmas espectrales de 4 tipos de cobertura de vegetación. Describa
        brevemente el comportamiento de la respuesta espectral de la vegetación
        en la zona entre el rojo y el infrarrojo cercano.
\end{enumerate}

\newpage
\section{Corrección Radiométrica de Imágenes satelitales}
Es objetivo de esta practica es conocer como afecta la interacción entre la luz
y la atmósfera a la toma de firmas espectrales en el terreno a partir de una
imagen satelital y estudiar distintos métodos empíricos y estadísticos para
corregirlas.
\subsection{Actividades pr\'acticas}
\begin{enumerate}
    \item Convierta la imagen  \texttt{l1\_l8\_oli\_2013-08-19.tif} a reflectancia a 
        tope de la atmósfera utilizando los parámetros de calibración que se 
        encuentran dentro del metadato de la misma.
    \item Corrija la imagen del punto anterior utilizando el coseno del ángulo solar 
        como única corrección.
    \item Grafique el histograma para cada banda de la imagen anterior y utilícelos 
        para corregirla por el método de substracción de cuerpo obscuro (DOS1).
    \item Compare las firmas espectrales para los distintos métodos de
        corrección radiométrica con la obtenida para la imagen 
        \texttt{l8\_oli\_2013-08-19.tif}.
\end{enumerate}

\subsection{Preguntas}
\begin{enumerate}
    \item ¿Encuentra alguna diferencia visual entre las imágenes corregidas por los
        distintos métodos?
    \item Compare las firmas espectrales obtenidas a partir de la imagen
        corregida a tope de la atmosfera  la corregida por el coseno del angulo
        solar y diga si la correccion depende del valor de reflectancia o no.
    \item A partir de la comparación entre la firma espectral de la imagen
        corregida por el coseno del angulo y la imagen corregida por el método DOS1, 
        diga en que zonas del espectro se encuentra las mayor diferencia. Incluya 
        un gráfico de una firma espectral para alguna cobertura de su interés que 
        lo muestre.
    \item Grafique el menor valor de reflectancia para cada banda en función de la
        longitud de onda. Describa el comportamiento encontrado. ¿Este efecto se
        relaciona con el scattering atmosférico o con la absorción atmosférica?
\end{enumerate}
\newpage
\section{Calculo de \'indices espectrales}
El objetivo de esta práctica es estudiar los \'{\i}indices espectrales
calculados a partir de  imágenes satelitales y sus distintos usos como un caso
particular de reducci\'on de la dimensionalidad.

\subsection{Actividades pr\'acticas}
\begin{enumerate}
    \item Abra las escenas Landsat 8 correspondientes a los meses de
        \texttt{l8\_oli\_2013-08-19.tif} y \texttt{l8\_oli\_2013-08-11.tif} y
        calcule los indices NDVI y EVI en ambas fechas.
    \item Apile cada par de  índices y visualice la imagen en la combinación que 
        muestre ambos en simultáneo.
    \item Abra la imagen \texttt{mod13q1\_ndvi\_2013-07-27\_2014-08-28.tif} y 
        estudie su variaci\'on temporal a los largo del a\~no para distintas 
        coberturas. Repita el procedimiento con la imagen 
        \texttt{mod13q1\_evi\_2013-07-27\_2014-08-28.tif}.
    \item Estudie la tendencia lineal de cada sector de la imagen a lo largo del
        periodo de estudio.
\end{enumerate}

\subsection{Preguntas}
\begin{enumerate}
    \item ¿Cómo se observa, en la imagen de índices apilados las distintas
        coberturas? Explique brevemente a que se debe dicha diferencia.
    \item Muestre en un gráfico de variación anual del NDVI para distintas coberturas.
        ¿Cuál es la que presenta la estacionalidad más marcada? Identifique, si es
        posible, la temporada de crecimiento para la misma.
    \item Grafique en simultaneo los \'{\i}ndices EVI y NDVI.?`Encuentra alguna
        diferencia entre el comportamiento temporal de ambos? 
    \item Calculo el \'indice NDVI a partir de la imagen \texttt{l1\_l8\_oli\_2013-08-19.tif}
        y comparelos con el obtenido a partir de la
        imagen\texttt{8\_oli\_2013-08-19.tif} estudiando el valor para distintos
        tipos de cobertura.?`Coinciden los mismos??`A que se debe la
        diferencia?
\end{enumerate}
\newpage
\section{Rotaciones y transformaciones}
El objetivo de esta práctica es  profundizar los conceptos de reducción
de dimensionalidad en el trabajo con imágenes satelitales a través de la
utilizacion de rotaciones y transformaciones espectrales.

\subsection{Actividades pr\'acticas}\label{rot:pra}
\begin{enumerate}
    \item Utilizando la herramienta Análisis por componentes principales
        encuentre la rotación que diagonaliza la matriz de correlación para la
        imagen MODIS\@. Análise por componentes principales y diga que información puede
        distinguir en la misma. 
    \item \label{pca} Apile las imagenes
        \begin{itemize}
        \item \texttt{l8\_oli\_2013-08-19.tif}
        \item \texttt{l8\_oli\_2013-11-07.tif}
        \item \texttt{l8\_oli\_2014-02-11.tif}
        \item \texttt{l8\_oli\_2014-05-18.tif}
        \end{itemize}
        y aplique la transformaci\'on por componentes principales a la misma.
        Analise las distintas compotentes.
    \item Utilizando la imagen \texttt{l8\_oli\_2014-08-06.tif} calcule la 
        transformada tasseled-cap y compare la componentes de \emph{verdor} con el NDVI.
    \item Utilizando la imagen \texttt{l8\_oli\_2014-08-06.tif} calcule la 
        transformada por componentes principales y compare sus componentes con la
        componente de verdor de la transformada tasseled-cap.
\end{enumerate}

\subsection{Preguntas}
\begin{enumerate}
    \item ¿Cómo se relacionan las primeras 3 componentes de la transformada por
        componentes principales con las curvas de variación del NDVI antes
        estudiadas? 
    \item En la imagen obtenida a partir del proceso de calculo de componentes
        principales sobre el apilado de imagenes Landsat,?`Cuantas componentes
        se necesitan para explicar el 95\% de la variabilidad de la imagen?
    \item Describa, brevemente, como se interpretan las primeras 3 componentes
        de la imagen obtenida en el punto \ref{pca}?`Como se relaciona esto con la
        firma espectral de la vegetaci\'on?
    \item Interprete brevemente la componente de brillo de la transformada
        tasseled cap. Con que componente de una transformada por PCA esta mas
        asociada.
\end{enumerate}

\section{Métodos no supervisados de clasificación}

El objetivo de esta práctica es continuar estudiado el concepto de reducción de
dimensionalidad, utilizando métodos de clasificación no supervisada para obtener
mapas de uso y cobertura con sus respectivas áreas.

\subsection{Actividades pr\'acticas}
\begin{enumerate}
    \item Clasifique la imagen \texttt{l8\_oli\_2014-08-06.tif}por el método 
        k-means, asignando un número total
        de 8 clases. Analice y asigne a estas clases los usos y coberturas de la
        tabla~\ref{tabla1}.
    \item Clasifique la imagen \texttt{l8\_oli\_2014-08-06.tif}por el método de 
        k-means, pero ahora utilice 50
        clases. Analice y asigne a estas clases los usos y coberturas de la
        tabla~\ref{tabla1}.
    \item Clasifique las primeras 6 bandas de  imagen obtenida en el punto~\ref{pca} 
        de la secci\'on~\ref{rot:pra} utilizando el m\'etodo k-means
        asignando un total de 50 clases. Analice y asigne a estas clases los 
        usos y coberturas de la tabla~\ref{tabla1}.
    \item Utilice las herramientas de calcular estadísticas globales para
        estimar las áreas correspondientes a cada tipo de uso y cobertura en
        las tres clasificaciones.
\end{enumerate}

\subsection{Preguntas}
\begin{enumerate}
    \item ¿Qué diferencia encuentra entre las imágenes clasificadas? Justifique
        las diferencias desde el punto de vista del método de k-means.
    \item Utilice la imagen  \texttt{mod13q1\_ndvi\_2013-07-27\_2014-08-28.tif }para 
        generar una nueva clasificación por el método k-means. Confeccione una 
        tabla comparativa de las áreas de los distintos usos y coberturas 
        obtenidas de las cuatro formas.
    \item ¿Considera que el método de k-means da resultados satisfactorios para
        la clasificación de áreas urbanas? ¿A qué se debe este problema?
    \item Utilice los filtros por mayoría en las imágenes clasificadas y
        describa brevemente que sucede con los bordes de las distintas clases.
\end{enumerate}

\newpage
\section{Métodos supervisados de clasificación}
El objetivo de esta práctica es continuar estudiado el concepto de reducción de 
dimensionalidad, utilizando métodos de clasificación supervisada para obtener 
mapas de uso y cobertura con sus respectivas áreas. 

\subsection{Actividades pr\'acticas}
\begin{enumerate}
    \item Digitalice un parche en cada categoria de uso y cobertura de la
    tabla\ref{tabla1} creando una capa vectorial para cada tipo de cobertura. 
    Grafique la firma espectral y el desvío de cada una. 
    \item Clasifique la imagen \texttt{l8\_oli\_2014-08-06.tif} las
    áreas de entrenamiento creadas en el punto anterior. Utilizando la herramienta
    de estadísticas globales, encuentre el área correspondiente a cada tipo de uso y
    cobertura.
    \item Cargue las capas vectoriales de la carpeta \texttt{entrenamiento} clasificar 
    la imagen.
    \item Fusione la imagen en las clases de uso y cobertura deseada, y utilice la
    imagen obtenida para calcular nuevamente el área correspondiente a cada uso y
    cobertura del suelo.
\end{enumerate}

\subsection{Preguntas}
\begin{enumerate}
    \item Clasifique la imagen obtenida en el punto~\ref{pca} de la secci\'on~\ref{rot:pra}
        utilizando la herramienta de clasificación supervisada.
        Repita este proceso utilizando las 6 primeras bandas de la imagen. 
        Compare y describa visualmente ambas clasificaciones.
    \item Explique brevemente, desde el punto de vista de la firma espectral, la
        necesidad de tomar distintas clases de entrenamiento para cada uso y
        cobertura encontrado en la imagen. ¿Cómo se determina la homogeneidad de
        cada uno de estos parches?
    \item ¿Cuál es el problema de aumentar el número de bandas del satélite
        utilizadas para la clasificación sin aumentar el número de píxeles de
        entrenamiento?
    \item Compare las áreas obtenidas a partir del método k-means, utilizando la
        imagen fusionada, y la clasificación por máxima verosimilitud y diga si
        las mismas son comparables o no.
\end{enumerate}

\newpage
\section{Validación de clasificaciones}
El objetivo de esta práctica es analizar la precisión de las clasificaciones
realizadas en clases anteriores haciendo hincapié en la importancia del muestreo
y los problemas que pueden presentarse. 

\subsection{Actividades pr\'acticas}
\begin{enumerate}
    \item Abra la imagen \texttt{spot5\_hrg2\_2013-12-25\_702\_402.tif} y
        identifique coberturas de la tabla~\ref{tabla1}, digitalice un poligono
        para cada una de las mismas.
    \item Cargue los polígonos de la carpeta \texttt{validacion} correspondientes a cada clase de
        uso y cobertura del suelo. 
    \item Realice la matriz de confusión para cada una de las clasificaciones.
    \item A partir de las matrices obtenidas calcule la precisión global,
        precisiones del usuario el productor y el índice kappa para cada una de
        ellas. Utilizando ademas los datos de area de cada imagen obtenga las
        areas y errores correspondientes a cada categoria de uso y cobertura.
\end{enumerate}
\subsection{Preguntas}
\begin{enumerate}
    \item Calcule la precisión de las clasificaciones obtenidas a partir de la
    imagen de NDVI y del apilado de las imágenes Landsat 8. Para ello
    construya la matriz de confusión para cada una y calcule  las precisiones
    globales, del usuario y del productor. Muestre las matrices de confusión
    obtenidas y sus precisiones globales para la clasificación del NDVI, la
    clasificación de las imágenes apiladas, la clasificación por el método
    k-means de 50 clases fusionadas y la clasificación supervisada.
    \item ¿Cuál de los métodos elegidos muestra mejores resultados para la
    clasificación de areas de vegetaci\'on natural? ¿Qué clasificaci\'on elegiría, desde
    el punto de vista del usuario, para estimar el área del embalse
    Urugua-\'{\i}? Justifique su respuesta.
    \item Utilice el método de mayor precisión global para estimar el área  de cada
    uso y cobertura. Encuentre el error correspondiente a los mismos.
    \item ¿Es representativo el muestreo realizado? En qué zonas de la imagen
    propone tomar más puntos de validación y a que clases pertenecerían.
\end{enumerate}

\appendix

\section{Categorias de uso y cobertura del suelo}
\begin{table}
    \centering
    \begin{tabular}{lll}
        \toprule
        Nombre & Codigo & Descripcion\\
        \midrule 
        A & B & C \\
        \bottomrule
    \end{tabular}
\caption{\label{tabla1}Categorias usos del suelo segun el esquema LCCS2 de la
FAO.}
\end{table}

%\printbibliography\
\end{document}
