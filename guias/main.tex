\documentclass[hidelinks,12pt]{article}

% Preambulo por defecto
% Paquetes para usar bien el idioma español
\usepackage[spanish,es-tabla]{babel}
\selectlanguage{spanish}
\usepackage[utf8]{inputenc}

% Paquetes para usar mejores imagenes
\usepackage{graphicx}

% Paquetes para links y tabla de contenidos en el PDF
\usepackage{hyperref}
\hypersetup{colorlinks=true,allcolors=blue}
%\usepackage{hypcap}

% Paquetes para mejores tablas
\usepackage{booktabs}

% Mejor matematica
\usepackage{amsmath}

% Fuentes de las imagenes
\usepackage[absolute,overlay]{textpos}

% Paquete captions
\usepackage[justification=centering,labelformat=empty,labelsep=none]{caption}

% Opciones para ticks
\usepackage{tikz}
\usetikzlibrary{shapes,arrows,positioning}

\tikzstyle{decision} = [diamond, draw, fill=blue!20, text width=4em, text badly centered, node distance=2cm, inner sep=0pt,on grid]
\tikzstyle{block} = [rectangle, draw, fill=blue!20, text width=8em, text centered, rounded corners, minimum height=2em,on grid]
\tikzstyle{line} = [draw, -latex]

% Citas bibliograficas
\usepackage[backend=biber]{biblatex}
\renewcommand{\footnotesize}{\tiny}
\addbibresource{biblio.bib}

% Mejoro las captions
\setbeamertemplate{caption}{\raggedright\insertcaption\par}

\setbeamertemplate{caption}{%
\begin{beamercolorbox}[wd=0.85\paperwidth, sep=.2ex]{block body}\insertcaption%
\end{beamercolorbox}%
}


% Sacar barra de navegacion
\setbeamertemplate{navigation symbols}{}%remove navigation symbols

% Transparencias en items
\setbeamercovered{transparent}

% Estilo de diapositivas
% \usetheme{Boadilla}
\usecolortheme{whale}
\usecolortheme{orchid}


\definecolor{A11}{HTML}{B2DF8A}
\definecolor{A12}{HTML}{33A02C}
\definecolor{A23}{HTML}{FDBF6F}
\definecolor{A24}{HTML}{FF7F00}
\definecolor{B15}{HTML}{FB9A99}
\definecolor{B16}{HTML}{E31A1C}
\definecolor{B27}{HTML}{A6CEE3}
\definecolor{B28}{HTML}{1F78B4}

\title{SoPI II \- Herramientas de Teledetecci\'on Cuantitativa \\ 
\emph{Gu\'{\i}a de actividades: Uso del suelo en el departamento de Iguaz\'u,
provincia de Misiones}}
\author{Francisco
Nemi\~na\thanks{\href{mailto:fnemina@conae.govar}{fnemina@conae.gov.ar}}}
\affil{Unidad de Educaci\'on y Formaci\'on Masiva\\ 
    Comisi\'on Nacional de Actividades Espaciales}
\date{\today}

\begin{document}

\maketitle

\section*{Introducci\'on}

La utilización de imágenes satelitales permite analizar grandes extensiones del
territorio, contando con un registro histórico con el cual realizar
comparaciones.

En la provincia de Misiones, el departamento de Iguaz\'u es lindante a Brasil y 
Paraguay siendo parte de la zona conocida como triple frontera
perteneciente a la ecorregi\'on conocida como \emph{selva paranaense}. Dentro del 
mismo podemos encontrar la Represa de Urugua-\'{\i} y el parque nacional
Iguaz\'u.

Tomaremos entonces al departamento como \'area de estudio durante este curso
con el objetivo de obtener un mapa de uso y cobertura dentro del mismo que nos
permita estimar y validar  las \'areas correspondients a los mismos. 

Utilizaremos para esto imagenes satelitales de los satelites Landsat 8, SPOT-5 y
el producto de MOD13Q1 obtenido de los satelites TERRA y AQUA obtenidas durante el
periodo que va de agosto de 2013 a agosto de 2014. 

\newpage
\section{Análisis de Firmas Espectrales}
Es objetivo de esta pr\'actica es familiarizarse con la zona de inter\'es, estudiar
y caracterizar el comportamiento espectral de distintas las categor\'{\i}as de uso y
cobertura y comprender como se relacionan las mismas con los valores obtenidos 
por a partir de im\'agenes  satelitales.

\subsection{Actividades pr\'acticas}
\begin{enumerate}
    \item Abra la imagen \texttt{l8\_oli\_20140211.tif}. Realice distintas
        combinaciones de banda y seleccione aquella que le permita distinguir 
        detalles mas de la vegetación.
    \item Encuentre dentro de la escena parches de coberturas uniformes y 
        digitalícelos utilizando la herramienta de edición vectorial. Tome
        ejemplos de distintas coberturas correspondientes a los casos A11, A12,
        B15, B27 y B28 de la tabla~\ref{tabla1} del ap\'endice~\ref{apcate}
    \item Utilizando la herramienta de extracción de estadísticas globales
        calcule y grafique la media correspondiente a cada cobertura analizada 
        en el punto 2. Compararla con la firma espectral disponible para dicha
        cobertura.
    \item Repita el proceso para coberturas del mismo tipo en la imagen 
        correspondiente a \texttt{l1\_l8\_oli\_20130819.tif}.
\end{enumerate}
\subsection{Preguntas}
\begin{enumerate}
    \item Justifique desde el punto de vista de la firma espectral por qué una
        combinación de bandas que incluya zonas del espectro visible, el
        infrarrojo cercano y el infrarrojo medio aporta más información sobre la
        vegetación.
    \item Compare las respuestas espectrales de la vegetación en la zona del
        espectro visible en de las im\'agenes \texttt{l8\_oli\_20140211.tif}
        y \texttt{l1\_l8\_oli\_20130819.tif} y explique qué diferencias
        encuentra. ?`Le parece que las mismas pueden asociarse a un
        comportamiento biof\'isico de la vegetaci\'on?
    \item Compare las respuestas espectrales obtenidas para el agua en los meses
        de las im\'agenes \texttt{l8\_oli\_20140211.tif}
        y \texttt{l1\_l8\_oli\_20130819.tif}. En caso de encontrar diferencias, 
        diga en que zona del espectro se encuentran las mismas y a qué supone que 
        se debe esta diferencia.
    \item A partir de dicha la imagen \texttt{l8\_oli\_20140211.tif} genere un 
        gráfico que incluya las firmas espectrales de 4 tipos de cobertura de 
        vegetación. Describa cualitativamente las caracteristicas comunes de las
        cuatro firmas. ?`Donde presentan mayor diferencia? ?`A que se debe?
\end{enumerate}

\newpage
\section{Corrección Radiométrica de Imágenes satelitales}
Es objetivo de esta practica es conocer como afecta la interacción entre la luz
y la atmósfera a la radiometr\'ia de una imagen satelital y la respuesta
espectral de los distintos usos y coberturas  y estudiar distintos métodos empíricos 
y estadísticos para corregirla.
\subsection{Actividades pr\'acticas}
\begin{enumerate}
    \item Convierta la imagen  \texttt{l1\_l8\_oli\_20130819.tif} a reflectancia a 
        tope de la atmósfera utilizando los parámetros de calibración que se 
        encuentran dentro del metadato de la misma.
    \item Corrija la imagen del punto anterior utilizando el coseno del ángulo solar 
        como única corrección.
    \item Grafique el histograma para cada banda de la imagen anterior y utilícelos 
        para corregirla por el método de substracción de cuerpo obscuro (DOS1).
    \item Compare las firmass espectrales obtenidas en el punto anterior con las
        obtenidas a partir de la imagen \texttt{l8\_oli\_20130819.tif} para
        distintos usos y coberturas.
\end{enumerate}

\subsection{Preguntas}
\begin{enumerate}
    \item ¿Encuentra alguna diferencia visual entre las imágenes corregidas por los
        distintos métodos?
    \item Compare las firmas espectrales obtenidas a partir de la imagen
        corregida a tope de la atmosfera  la corregida por el coseno del angulo
        solar y diga si la correccion depende del valor de reflectancia o de la
        longitud de onda.
    \item A partir de la comparación entre la firma espectral de la imagen
        corregida por el coseno del angulo y la imagen corregida por el método DOS1, 
        diga en que zonas del espectro se encuentra las mayor diferencia. Incluya 
        un gráfico de una firma espectral para alguna cobertura de su interés que 
        lo muestre.
    \item Grafique el menor valor de reflectancia para cada banda en función de la
        longitud de onda. Describa el comportamiento encontrado. ¿Este efecto se
        relaciona con el scattering atmosférico o con la absorción atmosférica?
\end{enumerate}
\newpage
\section{Calculo de \'{\i}ndices espectrales}
El objetivo de esta práctica generar e interpretar \'{\i}ndices espectrales
a partir de  imágenes satelitales y sus distintos usos como un caso
particular de reducci\'on de la dimensionalidad.

\subsection{Actividades pr\'acticas}
\begin{enumerate}
    \item Abra las imagenes \texttt{l8\_oli\_20130819.tif} y 
        \texttt{l8\_oli\_20140211.tif} y calcule los indices NDVI y EVI en 
        para ambas.
    \item Apile cada par de  índices y visualice la imagen en la combinación que 
        muestre ambos en simultáneo.
    \item Abra la imagen \texttt{mod13q1\_ndvi\_2013\-07\-27\_2014\-08\-28.tif} y 
        estudie su variaci\'on temporal a los largo del a\~no para distintas 
        coberturas. Repita el procedimiento con la imagen 
        \texttt{mod13q1\_evi\_2013\-07\-27\_2014\-08\-28.tif}.
    \item Estime la pendiente de la linea de suelo a partir de la
        interpretaci\'on de un scatter-plot de las bandas del infrarrojo cercano
        y rojo de la imagen \texttt{l8\_oli\_2014\-02\-11.tif} y utilicela para
        calcular el \'{\i}ndice TSAVI\@.
\end{enumerate}

\subsection{Preguntas}
\begin{enumerate}
    \item ¿Cómo se observa, en la imagen de índices apilados las distintas
        coberturas? Explique brevemente a que se debe dicha diferencia.
    \item Muestre en un gráfico de variación anual del NDVI para distintas coberturas.
        ¿Cuál es la que presenta mayor variaci\'on en el tiempo? Identifique, si es
        posible, la temporada de crecimiento para la misma.
    \item Grafique en simultaneo los \'{\i}ndices EVI y NDVI. ?`Encuentra alguna
        diferencia entre el comportamiento temporal de ambos? 
    \item Calcule el \'{\i}ndice NDVI a partir de la imagen
        \texttt{l1\_l8\_oli\_2013\-08\-19.tif}
        y comparelos con el obtenido a partir de la
        imagen \texttt{8\_oli\_2013\-08\-19.tif} estudiando el valor para distintos
        tipos de cobertura. ?`Coinciden los mismos? ?`Es biologicamente relevante
        este cambio?
\end{enumerate}
\newpage
\section{Rotaciones y transformaciones}
El objetivo de esta práctica es  profundizar los conceptos de reducción
de dimensionalidad en el trabajo con imágenes satelitales a través de la
utilizacion de rotaciones y transformaciones espectrales.

\subsection{Actividades pr\'acticas}\label{rot:pra}
\begin{enumerate}
    \item Utilizando la herramienta Análisis por componentes principales
        encuentre la rotación que diagonaliza la matriz de correlación para la
        imagen \texttt{mod13q1\_ndvi\_2013\-07\-27\_2014\-08\-28.tif}. Análise 
        por componentes principales y diga que información puede distinguir en la 
        misma. 
    \item \label{pca} Apile las imagenes
        \texttt{l8\_oli\_2013\-08\-19.tif}
        ; \texttt{l8\_oli\_2013\-11\-07.tif}
        ; \texttt{l8\_oli\_2014\-02\-11.tif}
        y \texttt{l8\_oli\-\_2014\-05\-18.tif}.
        Aplique la transformaci\'on por componentes principales.
        Analice las distintas compotentes.
    \item Con imagen \texttt{l8\_oli\_2014\-02\-11.tif} calcule la 
        transformada tasseled-cap. Compare la componente de \emph{verdor} con
        el NDVI\@.
    \item Con la imagen \texttt{l8\_oli\_2014\-02\-11.tif} calcule la 
        transformada por componentes principales y compare las primeras
        componentes con las de la transformada tasseled-cap.
\end{enumerate}

\subsection{Preguntas}
\begin{enumerate}
    \item ¿Cómo se relacionan las primeras 2 componentes de la transformada por
        componentes principales con el promedio y el comportamiento anual de la
        variaci\'on temporal del NDVI\@? 
    \item En la imagen obtenida a partir del proceso de calculo de componentes
        principales sobre el apilado de imagenes Landsat, ?`Cuantas componentes
        se necesitan para explicar el 75\% de la variabilidad de la imagen?
    \item Describa, brevemente, como se interpretan las primeras 3 componentes
        de la imagen obtenida en el punto~\ref{pca}. ?`Como se relaciona esto con la
        firma espectral de la vegetaci\'on?
    \item Interprete brevemente la componente de brillo y verdorde la transformada
        tasseled cap.
\end{enumerate}

\section{Métodos no supervisados de clasificación}

El objetivo de esta práctica es continuar estudiado el concepto de reducción de
dimensionalidad, utilizando métodos de clasificación no supervisada para obtener
mapas de uso y cobertura con sus respectivas áreas.

\subsection{Actividades pr\'acticas}
\begin{enumerate}
    \item Clasifique la imagen \texttt{l8\_oli\_2014\-02\-11.tif} por el método 
        k-means, asignando un número total
        de 5 clases. Analice y asigne a estas clases las categorias A11, A12,
        B15, B27, B28 de la tabla~\ref{tabla1} del ap\'endice~\ref{apcate}.
    \item Clasifique la imagen \texttt{l8\_oli\_2014\-02\-11.tif} por el método de 
        k-means, pero ahora utilice 50
        clases. Analice y asigne a estas clases las categorias A11, A12,
        B15, B27, B28 de la tabla~\ref{tabla1} del ap\'endic~\ref{apcate}.
    \item Clasifique las primeras 6 la bandas de imagen obtenida en el punto~\ref{pca} 
        de la secci\'on~\ref{rot:pra} utilizando el m\'etodo k-means
        asignando un total de 50 clases. Analice y asigne a estas clases las categorias A11, A12,
        B15, B27, B28 de la tabla~\ref{tabla1} del ap\'endice~\ref{apcate}.
    \item Utilice las herramientas de calcular estadísticas globales para
        estimar las áreas correspondientes a cada tipo de uso y cobertura en
        las tres clasificaciones.
\end{enumerate}

\subsection{Preguntas}
\begin{enumerate}
    \item ¿Qué diferencia encuentra entre las imágenes clasificadas? Justifique
        las diferencias desde el punto de vista de la variaci\'on del parametro
        de numero de clases del método de k-means.
    \item Clasifique la imagen  \texttt{mod13q1\_ndvi\_2013\-07\-27\_2014\-08\-28.tif }para 
        generar una nueva clasificación por el método k-means. Confeccione una 
        tabla comparativa de las áreas de los distintos usos y coberturas.
    \item ¿Considera que el método de k-means da resultados satisfactorios para
        la clasificación de áreas urbanas? ¿A qué se debe este problema? ?`Como
        lo corroborar\'ia?
    \item Utilice los filtros por mayoría en las imágenes clasificadas y
        describa brevemente que sucede con los bordes de las distintas clases.
\end{enumerate}

\newpage
\section{Métodos supervisados de clasificación}
El objetivo de esta práctica es continuar estudiado el concepto de reducción de 
dimensionalidad, utilizando métodos de clasificación supervisada para obtener 
mapas de uso y cobertura con sus respectivas áreas. 

\subsection{Actividades pr\'acticas}
\begin{enumerate}
    \item Digitalice un parche homogeneo  para las categorias A11, A12, B15, B27, B28 de la 
    tabla~\ref{tabla1} del ap\'endice~\ref{apcate} creando una capa vectorial para 
    cada tipo de cobertura. Grafique la firma espectral y el desvío de cada una. 
    \item Clasifique la imagen \texttt{l8\_oli\_2014\-02\-11.tif} utilizando las
    áreas de entrenamiento creadas en el punto anterior. Con la herramienta
    de estadísticas globales, encuentre el área correspondiente a cada tipo de uso y
    cobertura.
    \item Cargue las capas vectoriales de la carpeta \texttt{entrenamiento} y
        vuelva a clasificar la imagen \texttt{l8\_oli\_2014\-02\-11.tif}.
    \item Fusione la imagen en las clases de uso y cobertura deseada, y utilice la
    imagen obtenida para calcular nuevamente el área correspondiente a cada uso y
    cobertura del suelo. 
\end{enumerate}

\subsection{Preguntas}
\begin{enumerate}
    \item Clasifique la imagen obtenida en el punto~\ref{pca} de la secci\'on~\ref{rot:pra}
        utilizando la herramienta de clasificación supervisada.
        Repita este proceso utilizando las 6 primeras bandas de la imagen. 
        Compare y describa visualmente ambas clasificaciones.
    \item Explique brevemente, desde el punto de vista de la firma espectral, la
        necesidad de tomar distintas clases de entrenamiento para cada uso y
        cobertura encontrado en la imagen. ¿Cómo se determina la homogeneidad de
        cada uno de estos parches?
    \item Compare las clasificaciones obtenidas al utilizar distintos umbrales. 
        ¿Que pasa con las \'areas clasificadas como sin clase cuando el umbral
        vale cero?
    \item Compare las áreas obtenidas a partir del método k-means, utilizando la
        imagen fusionada, y la clasificación por máxima verosimilitud y diga si
        las mismas son comparables o no.
\end{enumerate}

\newpage
\section{Validación de clasificaciones}
El objetivo de esta práctica es analizar la precisión de las clasificaciones
realizadas en clases anteriores haciendo hincapié en la importancia del muestreo
y los problemas que pueden presentarse. 

\subsection{Actividades pr\'acticas}
\begin{enumerate}
    \item Abra la imagen \texttt{spot5\_hrg2\_2013\-12\-25\_702\_402.tif} e
        identifique coberturas de las categorias A11, A12, B15, B27, B28 de la 
        tabla~\ref{tabla1} del ap\'endic~\ref{apcate}. Digitalice un poligono
        para cada una.
    \item Cargue los polígonos de la carpeta \texttt{validacion} correspondientes a cada clase de
        uso y cobertura del suelo. 
    \item Realice la matriz de confusión para cada una de las clasificaciones.
    \item A partir de las matrices obtenidas calcule la precisión global,
        precisiones del usuario el productor y el índice kappa para cada una de
        ellas. Utilizando ademas los datos de \'area de cada imagen, obtenga las
        areas y errores correspondientes a cada categoria de uso y cobertura.
\end{enumerate}
\subsection{Preguntas}
\begin{enumerate}
    \item Calcule la precisión de las clasificaciones obtenidas a partir de la
    imagen de NDVI y del apilado de las imágenes Landsat 8. ?`Cual es el
    m\'etodo con mayor presici\'on global
    \item Utilice el método de mayor precisión global para estimar el área  de cada
    uso y cobertura. Encuentre el error correspondiente a los mismos.
    \item ¿Cuál de los métodos elegidos muestra mejores resultados para la
    clasificación de areas de vegetaci\'on natural? ¿Qué clasificaci\'on elegiría, desde
    el punto de vista del usuario, para estimar el área del embalse
    Urugua-\'{\i}? Justifique su respuesta.
    \item ?`Cuales son las categorias con menor precisi\'on del productor? ?`Que
        le dice esto sobre los m\'etodos de clasificaci\'on? ?`Cual es la
        utilidad de la presici\'on del productor y del usuario?
\end{enumerate}
\newpage
\appendix

\section{Categorias de uso y cobertura del suelo}\label{apcate}
Categorias de uso y cobertuar segun el esquema LCCS2 de la FAO\@. Los colores son
sugerencias por categoria.
\begin{table}[hbt]
    \centering
    \begin{tabular}{p{11cm}cc}
        \toprule
        Nombre & Codigo & Color \\
        \midrule 
        Áreas Terrestres Cultivadas y Manejada & A11 & \textcolor{A11}{$\blacksquare$}\texttt{\#b2df8a}
        \\
        Vegetación natural y semi-natural & A12 & \textcolor{A12}{$\blacksquare$}\texttt{\#33a02c}\\
        Áreas Acuáticas o Regularmente Inundadas Cultivadas & A23  &
        \textcolor{A23}{$\blacksquare$}\texttt{\#fdbf6f}\\
        Vegetación Natural y Semi-Natural Acuática o
	Regularmente Inundadas & A24 & \textcolor{A24}{$\blacksquare$}\texttt{\#ff7f00}\\
        Superficies Artificiales y Áreas Asociadas & B15  &
        \textcolor{B15}{$\blacksquare$}\texttt{\#fb9a99}\\
        Áreas descubiertas o desnudas & B16 & \textcolor{B16}{$\blacksquare$}\texttt{\#e31a1c}\\
        Cuerpos Artificiales de Agua, Nieve y Hielo & B27 &
        \textcolor{B27}{$\blacksquare$}\texttt{\#a6cee3}\\
        Cuerpos Naturales de Agua, Nieve y Hielo & B28&
        \textcolor{B28}{$\blacksquare$}\texttt{\#1f78b4}\\
        \bottomrule
    \end{tabular}
\caption{\label{tabla1}Categorias usos del suelo segun el esquema LCCS2 de la
FAO.}
\end{table}

\section{Ecuaciones}
Ecuaciones utiles para el curso. Todas las magnitudes entre 0 y 1 estan
escaladas entre 0 y 10,000. Las mismas estan pensadas para usar como tipo de
dato entero de 16bits.

\begin{table}[hbt]
    \centering
    \begin{tabular}{lcl}
        \toprule
        Nombre & Ecuaci\'on & Observaciones\\
        \midrule
        Reflectancia TOA & $\rho'_\lambda =
        (G_\lambda*DN+B_\lambda)\times10000$ &
        $G_\lambda$ : ganancia\\
        & & $B_\lambda$ : desvio\\
        Correccion por COS & $\rho_\lambda = \rho'_\lambda /
        \sin(\theta_e)$ & $\theta_e$ : angulo de elevaci\'on \\
        Correccion por DOS1 & $\rho^*_\lambda = \rho_\lambda -
        \rho_{\lambda,min}$ & \\
        NDVI & $\frac{\rho_{nir} -
    \rho_{rojo}}{\rho_{nir}+\rho_{rojo}}\times10000$ & \\
        EVI & $G\frac{\rho_{nir} -
    \rho_{rojo}}{\rho_{nir}+C_1\rho_{rojo}-C_2\rho_{azul}+L}\times10000$ &
        $G=2.5$, $L=1.0$ \\
        & & $C_1=6.0$, $C_2 = 7.5$\\
        SAVI & $(1+L)\frac{\rho_{nir} -
    \rho_{rojo}}{\rho_{nir}+\rho_{rojo}+L}\times10000$ & $L = 0.5$\\
        TSAVI & $m\frac{\rho_{nir} -
        m\rho_{rojo}-b}{m\rho_{nir}+\rho_{rojo}-mb}\times10000$ & $m$ :
        Pendiente \\
        & & $b$ : Ordenada al origen\\
        & & de la linea de suelo.\\
        \bottomrule
    \end{tabular}
    \caption{Ecuaciones escaladas para utilizar con tipo de dato entero de 16
    bits}
\end{table}

\section{Transformada tasseled-cap}
Transformada tasseled cap para Landsat 8-OLI.

\begin{table}[htb]
    \centering
    \begin{tabular}{@{}lr@{}lr@{}lr@{}lr@{}lr@{}lr@{}l@{}}
        \toprule
                   & \multicolumn{2}{c}{(Azul)}  & \multicolumn{2}{c}{(Verde)} &
        \multicolumn{2}{c}{(Rojo)}  & \multicolumn{2}{c}{(NIR)}   &
        \multicolumn{2}{c}{(SWIR1)} & \multicolumn{2}{c}{(SWIR2)} \\ 
                   & \multicolumn{2}{c}{Banda 2} & \multicolumn{2}{c}{Banda 3} &
        \multicolumn{2}{c}{Banda 4} & \multicolumn{2}{c}{Banda 5} &
        \multicolumn{2}{c}{Banda 6} & \multicolumn{2}{c}{Banda 7} \\
        \midrule
        Brightness & 0,           & 3029         & 0,           & 2786         &
        0,           & 4733         & 0,           & 5599         & 0,
        & 5080         & 0,           & 1872         \\
        Greenness  & -0,          & 2941         & -0,          & 2430         &
        -0,          & 5424         & 0,           & 7276         & 0,
        & 0713         & -0,          & 1608         \\
        Wetness    & 0,           & 1511         & 0,           & 1973         &
        0,           & 3283         & 0,           & 3407         & -0,
        & 7117         & -0,          & 4559         \\
        TCT4       & -0,          & 8239         & 0,           & 0849         &
        0,           & 4396         & -0,          & 0580         & 0,
        & 2013         & -0,          & 2773         \\
        TCT5       & -0,          & 3294         & 0,           & 0557         &
        0,           & 1056         & 0,           & 1855         & -0,
        & 4349         & 0,           & 8085         \\
        TCT6       & 0,           & 1079         & -0,          & 9023         &
        0,           & 4119         & 0,           & 0575         & 0,
        & 0259         & 0,           & 0252         \\ \bottomrule
    \end{tabular}
    \caption{Transformada tasseled-cap para landsat 8.}
\end{table}

\section{Datos sobre las im\'agenes}
Datos espectrales sobre las im\'agenes utilizadas.
\begin{table}[htb]
    \centering
    \begin{tabular}{@{}llcclcc@{}}
        \toprule
        \multicolumn{1}{c}{Satelite} & \multicolumn{1}{c}{Sensor} &
        \multicolumn{2}{c}{Banda} & \multicolumn{1}{c}{Designacion} & $\lambda$ &
        \multicolumn{1}{l}{$\Delta \lambda$} \\
        \multicolumn{1}{c}{}         & \multicolumn{1}{c}{}       & Satelite
        & Imagen     & \multicolumn{1}{c}{}            & [nm]    &
        \multicolumn{1}{l}{[nm]}           \\ \midrule
        Landsat 8                    & OLI                        & 2
        & 1          & Azul                            & 482     & 60
        \\
        Landsat 8                    & OLI                        & 3
        & 2          & Verde                           & 561     & 57
        \\
        Landsat 8                    & OLI                        & 4
        & 3          & Rojo                            & 655     & 37
        \\
        Landsat 8                    & OLI                        & 5
        & 4          & NIR                             & 864     & 28
        \\
        Landsat 8                    & OLI                        & 6
        & 5          & SWIR 1                          & 1608    & 84
        \\
        Landsat 8                    & OLI                        & 7
        & 6          & SWIR 2                          & 2200    & 187
        \\
        SPOT 5                       & HRG-2                      & 3
        & 1          & NIR                             & 840     & 100
        \\
        SPOT 5                       & HRG-2                      & 2
        & 2          & Rojo                            & 645     & 70
        \\
        SPOT 5                       & HRG-2                      & 1
        & 3          & Verde                           & 545     & 90
        \\
        SPOT 5                       & HRG-2                      & 4
        & 4          & SWIR                            & 1665    & 170
        \\ \bottomrule
    \end{tabular}
    \caption{Datos espectrales sobre las im\'agenes utilizadas.}
    \label{my-label}
\end{table}

%\printbibliography\
\end{document}
